\chapter{Estudio del Problema}

Como se mencionó al final del capítulo anterior, la investigación vigente del fenómeno estudiado en el presente trabajo ha concluido en generar distintas hipótesis para explicar el mal rendimiento presentado por los sockets de internet en escenarios de concurrencia. Para ser exactos, y reiterando lo anterior, las sospechas se pueden agrupar en tres aristas distintas. En el presente capítulo se estudian las tres líneas hipotéticas responsables del problema, inspeccionando en distintos niveles la operación y funcionamiento del sistema, indagando en las operaciones teóricas de funcionamiento y contrastando dicho planteamiento a resultados experimentales obtenidos en cada caso. Finalmente, es fruto de cada una de las siguientes subsecciones un análisis profundo del aspecto problemático estudiado, junto con una conclusión sobre el mismo. Así también, es fruto del presente capítulo un veredicto sobre cuales -Y en qué medida- de los problemas estudiados tienen verdadera responsabilidad en el fenómeno estudiado.

\section{Estudio de Operación de Llamadas a Sistema}
En esta parte hablar del tema de las llamadas a sistema, recordar que en el estudio apsado la libc fue descartada como originadora del problema, y que nos quedamos con la idea de que lo spinlocks son responsables. ¿Lo son?, se revisan las llamadas a sistema usando varias herrramietnas y nos topamos con que las tendencias de tiempos son similares a las de los tiempos de esas llamadas, luego, hay una correlacion de esta estructura inhenrente al socket es la que causa el cuello de botella: indicio, el socket entero está siendo contenido por el spinlock de su estructura.
\subsection{Estudio de Llamadas de sistema}
\subsubsection{Perf}
\subsubsection{FTrace}
\subsection{Resultados}
\subsubsection{Perf}
\subsubsection{Call-Graph}
\subsubsection{FTrace}
\subsubsection{Visualizador mio}
\subsection{Análisis}

\section{Estudio de Canales de Comunicación de Hardware}

Explicar a grosso modo la idea del cuello de botella en el bus de datos, y cómo ese enfoque se hereda entre distintos cambios de arquitectura de computadores.

\subsection{Desarrollo de Arquitecturas Modernas de Hardware}
\subsubsection{FSB}
\subsubsection{alguna otra...}
\subsubsection{Arquitectura Intel QuickPath}

\subsection{Especificación de Eventos}
\subsection{Captura de Eventos}
\subsubsection{Resultados}
\subsubsection{Correlación de Eventos}
acá la idea es ilustrar tendencias más importantes
\subsection{Análisis}

\section{Estudio de Distribución de Carga}
La tercera hipótesis de investigación plantea como responsable del mal rendimiento presentado en el caso de estudio al sistema de gestión y administración de tareas en el sistema operativo. En escenarios multicore como el estudiado, es normal que el sistema operativo realice como procedimiento de rutina la migración de procesos y la re-alocación de recursos y datos del sistema. Un caso práctico de ello es cuando un núcleo de procesamiento está sobreexigido y el sistema operativo redistribuye los hilos en dicho núcleo entre los procesadores disponibles del sistema, como una estratégia de balanceo de carga [AKA UNA REFERENCIA DE ESO].

Ésta tercera hipótesis plantea que dicho proceso de reasignación de recursos sería perjudicial en escenarios de concurrencia basandose en que, mientras un proceso está en plena ejecución al incorporar más y más tareas en el mismo nucleo de procesamiento agregando hilos de ejecución, sería el sistema operativo quien comienzaría la reasignación automática de dichos hilos entre los distintos procesadores, cayendo en problemas como perdida de referencias de memoria en niveles de chacé primario de procesos en cuestión. En su peor escenario, ésta teoría lleva al conocido problema de \emph{caché bouncing} [AKA CITA A ESOS PROBLEMAS] que corresponde al fenómeno de sobre corrección de los datos a nivel de caché, producido por constantes cambios de contexto del \emph{scheduller}. Un problema que ya se mencionó en el estudio de la sección previa.

\subsection{Processor Affinity}
\subsection{Esquemas de Distribución}
\subsection{Resultados Experimentales}