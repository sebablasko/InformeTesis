\documentclass[upright, contnum]{umemoria}
\depto{DEPARTAMENTO DE CIENCIAS DE LA COMPUTACIÓN}
\author{SEBASTIÁN RAMÓN BLASCO VALENCIA}
\title{Estudio de detección y amortización de contención sobre la interfaz de red en sistemas Linux en escenarios de concurrencia sobre máquinas multicore}
\auspicio{NIC Chile Research Labs}
\date{DICIEMBRE 2015}
\guia{Javier Bustos Jiménez}
\carrera{MAGÍSTER EN CIENCIAS, MENCIÓN COMPUTACIÓN}
\memoria{TESIS PARA OPTAR AL GRADO DE}
\comision{Nancy Hitschfeld Kahler}{Luis Mateu Brulé}{Carolina Bonacic Castro}

\usepackage{lipsum}
\usepackage{subfigure}
\usepackage{bigfoot}
\usepackage{tikz}
\usepackage{amsmath}
\newcommand{\timeline}{\hspace{-2.3pt}$\bullet$ \hspace{5pt}}

\usepackage{minted}
\usepackage{multirow}
\colorlet{LightGray}{gray!5!}
\renewcommand{\lstlistingname}{Código}
\definecolor{mygray}{rgb}{0.95,0.95,0.95}
\definecolor{mygray2}{rgb}{0.99,0.99,0.79}

%Defino un par de estilos
\lstdefinestyle{CInputStyle}{
  language=C,
  basicstyle=\small\sffamily,
  numbers=left,
  numberstyle=\tiny,
  numbersep=3pt,
  frame=tb,
  columns=fullflexible,
  backgroundcolor=\color{mygray},
  linewidth=0.9\linewidth,
  xleftmargin=0.05\linewidth
}

\lstdefinestyle{BashInputStyle}{
  language=bash,
  basicstyle=\small\sffamily,
  frame=tb,
  columns=fullflexible,
  backgroundcolor=\color{mygray2}
}

\begin{document}

\frontmatter
\maketitle

%\begin{abstract}
\section*{Resumen}
La proliferación de sistemas con múltiples núcleos de procesamiento ha transformado la aplicación de técnicas de programación paralela en uno de los tópicos más estudiados en los últimos años, ello en pos de mejorar los rendimientos generales de cualquier operación. Una aplicación práctica de dicho enfoque es en el procesamiento de requerimientos DNS, los cuales han evidenciado un gran aumento de la mano del desarrollo de la Internet y la masificación de distintos tipos de dispositivos que demandan conectividad, y que por sus características son un buen candidato a un enfoque de procesamiento paralelo. Sin embargo, distintas investigaciones han revelado que la aplicación de técnicas multithreading per se no son una estrategia que garantice un escalamiento en los resultados.

Distintas empresas de carácter global (como Google, Facebook y Toshiba) así como investigaciones locales (de la mano de NIC Chile) han reconocido la existencia de este problema, reiterando responsabilidades a los Internet Sockets provistos por el kernel de Linux, los cuales al ser expuestos a un consumo concurrente degradan su capacidad de consumo de datos y rendimiento general, aún cuando existe poder de cómputo excedente. Así estas empresas y otros trabajos de investigación han planteado varias hipótesis teóricamente razonables para explicar dicho comportamiento pero sin concretar las mismas en términos experimentales que permitan confirmarlas o desmentirlas. 

La presente investigación plantea un estudio experimental del caso de los Sockets UDP que combina técnicas de profiling y testing de distinto nivel, a fin de verificar las principales sospechas vigentes que den explicación al problema en cuestión, reunidas en 3 líneas de trabajo: Problemas de distribución de carga, degradación del rendimiento por mecanismos de bloqueo y problemas de manejo de caché y defectos de contención de recursos. En la misma línea, se ilustra el impacto de fenómeno de contención de recursos en un escenario concurrente y su repercusión en los canales de comunicación en el procesamiento de datos en arquitecturas modernas multiprocesador como la estudiada. Es fruto de este trabajo un veredicto por cada estudio que concluya explicando las características inherentes a los Internet Sockets que expliquen su mal desempeño, bajo los distintos análisis efectuados. 

Posteriormente, la investigación se traslada a estudiar la técnica denominada \emph{reuseport}, un desarrollo de ingenieros de Google que plantea una solución al problema presentado y que promete mitigar el efecto negativo causado por el uso de multithreading en el contexto estudiado. Así también, se repasan aspectos como el rendimiento de este enfoque, y ciertas debilidades del mismo. 

Finalmente, es producto del presente trabajo el planteamiento e implementación de una solución propia al problema que, inspirado en el diseño de \emph{reuseport} e implementado como un módulo del kernel, provee un mecanismo de distribución de paquetes que permite optar a mejoras de desempeño en el procesamiento de los mismos usando técnicas de procesamiento paralelo clásicas. Una solución que --en su configuración estándar-- provee un rendimiento competitivo a \emph{reuseport}, pero que gracias a ser ampliamente configurable permite obtener mejores resultados que \emph{reuseport} en contextos de distribución de paquetes no uniformes. 

%\end{abstract}

\begin{dedicatoria} % opcional
Una dedicatoria corta. Por ejemplo, \emph{A los creadores de U-Campus}
\end{dedicatoria}

\begin{thanks} % opcional
\lipsum[1-2]
\end{thanks}
\cleardoublepage

\tableofcontents
\listoftables % opcional
\listoffigures % opcional

\mainmatter

%\begin{intro}
almost done
\end{intro}
\begin{intro}
Desde sus orígenes, nunca se sospechó el grado de penetración y protagonismo que llegaría a alcanzar Internet. De acuerdo a importantes y periódicos estudios sobre dicha red \cite{report:akamai} la tónica de últimos años se podría caracterizar como: Grandes y sostenidos incrementos tanto en el número de conexiones a Internet como en la velocidad de las mismas. Éste factor de alto crecimiento es seguramente la característica fundamental de la llamada ``Era de la información'' que nos ha repletado de dispositivos demandando constantemente acceso a Internet. Un escenario que postula todo un desafío para la infraestructura --tanto de hardware como de software-- disponible de la red.

\section*{Crecimiento de la red}
El gran crecimiento en la cantidad de sistemas tecnológicos interconectados vía Internet en el mundo en los últimos 30 años es algo que no ha dejado indiferente a nadie\footnote{\url{http://www.internetworldstats.com/emarketing.htm}}. Para entender este fenómeno distintos organismos internacionales se dedican periódicamente a realizar estimaciones de dicha cifra. Un caso muy popular de esta labor es el contador global de conexiones móviles a Internet del \emph{GSMA Intelligence}\footnote{\url{https://gsmaintelligence.com/}} el cual recientemente ha estimado en más de 7 mil millones el número de dispositivos móviles con conexión a la red, superando por primera vez al total de la población mundial\footnote{\url{http://www.cnet.com/news/there-are-now-more-gadgets-on-earth-than-people/}}. Una premisa que coincide con el último informe \emph{The State Of Internet} de la corporación \emph{Akamai} \cite{report:akamai} donde se resume el estudio de las principales variaciones en capacidad de acceso, velocidades de acceso, tipos de ataques, etc. a Internet de que disponen distintos países del mundo. Éste estudio corroboró lo que ya ha sido tendencia en los últimos años: Tanto las velocidades de navegación, como el total de conexiones a Internet han aumentado generalizadamente en todo el planeta (Ver figura \ref{fig:akamai_stats}). Las proyecciones a futuro preservan ésta tendencia apostando a que tanto la cantidad de dispositivos como el número de accesos a Internet deberían seguir subiendo \cite{nota:2020}, ello producto de factores como la reducción de costos de producción y la minimización de la tecnología, además de distintas tendencias generadas a raíz del fenómeno de \emph{globalización} que --en gran medida-- nos ha forzado a participar de una sociedad más interconectada en todo el mundo. Olvidando un poco las interpretaciones o justificaciones para ésta situación, el hecho concreto es que existe una directa proporcionalidad entre el número de dispositivos y los requerimientos de accesos a la red, y hoy ambos están en su apogeo de crecimiento.

\begin{figure}[!h]
	\centering
	\includegraphics[scale=0.5]{imagenes/conexiones_moviles}
	\caption{Registros de la compañía \emph{Ericsson} ilustrando el crecimiento exponencial en el uso de datos de los dispositivos móviles. Parte del informe de \emph{Akamai} \cite{report:akamai}}
	\label{fig:akamai_stats}
\end{figure}

Más allá del número de dispositivos o la cantidad --o calidad-- del acceso a Internet, las distintas aplicaciones actuales han evolucionado sobre la base de protocolos y sistemas diseñados hace varias décadas, pero exigiendo siempre la mejor performance posible en pos de garantizar buenos tiempos de respuesta. Los protocolos más celebres de la llamada \emph{familia de protocolos de Internet} son TCP e IP. Sin embargo, son decenas los protocolos y mecanismos involucrados en las diferentes fases de comunicación entre computadoras y aplicaciones, que permiten en conjunto la operación de la red de redes como hoy la conocemos.

\section*{El modelo OSI}
La capacidad de conectividad entre 2 distintos dispositivos es resultado del efecto combinado de varias capas de abstracción con responsabilidades divididas. Un diseño de operación que se ilustra en un modelo estándar vigente desde los años 80 es el impulsado por la \emph{Organización Internacional de Normalización} (ISO), mejor conocido como el modelo \textbf{OSI}\footnote{Por sus siglas en inglés \emph{Open System Interconnection}.}. En la práctica, la importancia de éste modelo radica en servir como una referencia técnica que ilustra los límites en las responsabilidades entre componentes que conforman una arquitectura de interconexión de sistemas.

El modelo OSI reconoce 7 capas de abstracción en el proceso de comunicación entre dispositivos (Ver figura \ref{fig:osi7capas}), cada una con obligaciones específicas, pero que combinadas soportan constructivamente un mecanismo de comunicación estándar para sistemas que por él se rijan. El acierto de éste enfoque está en permitir el desarrollo de soluciones modulares y especificas a cada una de las capas, sin interferir entre capas diferentes y manteniendo así la compatibilidad con aplicaciones que ya operen en capas distintas. Las 7 capas en cuestión se describen a continuación:

\begin{figure}[!h]
	\centering
	\includegraphics[scale=.45]{imagenes/OSI7Capas.png}
	\caption{Diagrama esquemático de las 7 capas del modelo OSI, ilustrando el proceso de encapsulamiento y desencapsulamiento de datos al ir agregando y removiendo encabezados de capa, según los procesos de empaquetamiento y desempaquetamiento, respectivamente.}
	\label{fig:osi7capas}
\end{figure}

\begin{description}
\item[Capa Física] La capa de nivel inferior en el modelo OSI es la capa física. Es ésta capa la responsible de la topología de la red y de la especificación de los medios materiales que consiguen la transmisión de la información, así como también es responsable de la generación real de la comunicación por medio del envío de la información. Es, en definitiva, la encargada del paso a canales físicos de la información a transmitir.

\item[Capa de Enlace] Es la segunda capa del modelo. Ella se encarga de proveer un mecanismo de direccionamiento físico en una máquina que permita el reconocimiento individual de la misma, proveiendo de un primer identificador a las máquinas en el modelo OSI dado por las direcciones físicas de los dispositivos (\emph{MAC}). Es responsable también de proveer mecanismos de corrección de errores en el proceso de transmisión de datos que pudiesen manifestarse por problemas de la capa física, haciendo dicha comunicación, desde éste punto, confiable.

\item[Capa de Red] Es la tercera capa del modelo OSI. Hace su debut en contextos de multiples equipos interconectados brindando mecanismos de identificación para proveer una capacidad de direccionamiento más amplia con respecto al disponible en las dos primeras capas (que sólo permitian la comunicación entre un par de máquinas, punto a punto). En éste nivel aparece uno de los protocolos más populares en Internet, el denominado protocolo IP que supone un mecanismo de identificación único para cada dispositivo en una red, provisto en base a una dirección homónima. Ésta capa establece la capacidad de ruteo en la comunicación como una responsabilidad de los nodos en una infraestructura en red, haciendo cada componente de la red alcanzable para cualquier otro integrante de la misma.

\item[Capa de Transporte] La cuarta capa del modelo, es la que provee de lleno la capacidad de transporte de datos. En éste nivel se incorporan los también celebres protocolos homónimos: TCP (orientado a la conexión) y UDP (orientado a la mensajería). Ésta capa es también responsible de brindar la capacidad de multiplexación a nivel de una máquina, permitiendo la generación de multiples conexiones desde el mismo dispositivo. Dicho mecanismo lo consigue al incorporar puertos numerados, que sirven como puntos lógicos de comunicación. De ésta manera, a partir de la capa de transporte se establece un paradigma base en el área de redes, en lo que a programación y esquematización de la misma corresponde: La construcción de conexiones \emph{IP:PUERTO}, que son la base del concepto de tuplas de direccionamiento en el proceso de transporte de datos.

\item[Capa de Sesión] Es la quinta capa del modelo OSI. Tal y como su nombre lo indica su función radica en ser la responsible de mantener un control de sesión en una conexión entre extremos, proveiendo mecanismos de corrección y reconexion en caso de interferencia de una operación entre máquinas. Es responsable de mantener el enlace de comunicación construido en base a las capas inferiores en un proceso de comunicación.

\item[Capa de Presentación] El sexto nivel en el modelo OSI es la capa de presentación, cuya responsabilidad comprende proveer el soporte para dar una correcta interpretación de los datos transmitidos, de manera de conseguir que los datos lleguen de manera reconocible al host de destino. A diferencia de las capas inferiores que se enfocan en los mecanismos de envío de la infromación, ésta capa guarda directa relación con la información transmitida y con su correcta interpretación final.

\item[Capa de Aplicación] Es la última -y de más alto nivel- capa de abstracción del modelo OSI. Es la responsible de proveer una interfaz simple a aplicaciones al acceso a mecanismos de comunicación en red. En otras palabras, es la responsible de proveer el servicio de comunicaciones a las distintas aplicaciones que tengan distintos requerimientos de comunicación.

\end{description}

A pesar de que el modelo OSI plantea responsabilidades delimitadas a cada capa de abstracción, la correspondencia de dicho estándar teórico en la práctica es una labor que queda supeditada a los programadores de sistemas operativos. Finalmente son ellos los que, en mayor o menor medida, hacen corresponder para con el modelo las implementaciones finales de los módulos de red de un sistema.


\section*{Familia de Protocolos de Internet}
El conjunto de múltiples protocolos que facultan a los sistemas con mecanismos de interconexión comprende a varios cientos que se distribuyen entre las distintas capas del modelo OSI. A este conjunto se le denomina \emph{Familia de Protocolos de Internet}. Sin embargo, históricamente se le ha prestado especial atención a un subconjunto de ellos que son estructurales en la infraestructura predominante de Internet y que rigen la misma, hablamos del conjunto de protocolos \textbf{TCP/IP}.

La arquitectura de los protocolos TCP/IP sigue la inspiración del modelo OSI en las responsabilidades a soportar --a pesar de romper la estructura de capas del mismo-- combinando algunas responsabilidades de dicho modelo en funciones únicas y obviando otras. Un ejemplo de esto se puede apreciar en la tabla \ref{tabla:tcpiposi} que ilustra la correspondencia de los protocolos del conjunto TCP/IP con su atribución según el modelo OSI.

\begin{table}[h!]
\centering
\begin{tabular}{|c|p{4cm}|l|p{5cm}|}
\hline
\multicolumn{1}{|c|}{\textbf{\begin{tabular}[c]{@{}c@{}}Ref. OSI\\ Nº de capa\end{tabular}}} & \multicolumn{1}{c|}{\textbf{\begin{tabular}[c]{@{}c@{}}Equivalente \\ de capa OSI\end{tabular}}} & \multicolumn{1}{c|}{\textbf{Capa TCP/IP}} & \multicolumn{1}{c|}{\textbf{\begin{tabular}[c]{@{}c@{}}Ejemplos de\\ protocolos TCP/IP\end{tabular}}} \\ \hline
5,6,7                                                                                        & Aplicación, Sesión, Presentación                                                                 & Aplicación                                & NFS, NIS, DNS, LDAP, telnet, ftp, rlogin, rsh, rcp, RIP, RDISC, SNMP y otros.                         \\ \hline
4                                                                                            & Transporte                                                                                       & Transporte                                & TCP, UDP, SCTP                                                                                        \\ \hline
3                                                                                            & Red                                                                                              & Internet                                  & IPv4, IPv6, ARP, ICMP                                                                                 \\ \hline
2                                                                                            & Vínculo de datos                                                                                 & Vínculo de datos                          & PPP, IEEE 802.2                                                                                       \\ \hline
1                                                                                            & Física                                                                                           & Red física                                & Ethernet (IEEE 802.3), Token Ring, RS-232, FDDI y otros.                                              \\ \hline
\end{tabular}
\caption{Comparativa de correspondencia de algunos protocolos de la familia de Internet de acuerdo al modelo OSI.}
\label{tabla:tcpiposi}
\end{table}
%\url{https://docs.oracle.com/cd/E19957-01/820-2981/6nei0r0r9/index.html}

\subsection*{UDP}
UDP \cite{rfc:768} es un protocolo de la familia de protocolos de Internet con funciones a nivel de la capa de transporte según el modelo OSI, que se caracteriza por ser un protocolo orientado a mensajes, vale decir, por permitir enviar mensajes a través de la red sin necesidad de establecer previamente una conexión con el equipo receptor (situación que si ocurre y es característica de los protocolos orientados a la conexión como es el caso de TCP).
%[OSI reference model—The ISO model of architecture for open systems interconnection]%

Ésta naturaleza de UDP tiene varias implicancias:
\begin{itemize}
\item En primer lugar, ser un protocolo orientado a mensajes supone una premura en el envío de información, ello significa que éste protocolo no verifica la correctitud en la recepción de los datos enviados. En ese sentido, UDP es lo que se denomina un protocolo \textbf{no fiable}.
\item Por otro lado, UDP trabaja en estados denominados \textbf{sin conexión}, lo que significa que no hay una verdadera sincronización entre origen y destino. Esto supone el uso de operaciones del tipo asíncronas las que hacen más flexible la comunicación entre los extremos.
\end{itemize}

Por su funcionamiento, la anatomía o estructura de un paquete UDP es bastante simple (Ver fig. \ref{fig:datagramaudp}). Al ser un protocolo orientado al envío de mensajes, los paquetes disponen de pocos campos de información, de manera de evitar sobrecargar los mismos en su transferencia. Así mismo, en éste nivel los paquetes se denominan \emph{datagramas}. Los encabezados de un paquete UDP comprenden: \textbf{Puerto de Origen/Destino}, que hace referencia a los puntos de conexión de la capa Transporte del modelo OSI y \textbf{Longitud del Mensaje} y \textbf{\emph{Checksum}}\footnote{Corresponde a un valor generado a partir de una función matemática que se aplica sobre datos para corroborar la integridad de los mismos y su correcta recepción tras un envío.}, que sirven como componentes de verificación del paquete mismo.

\begin{figure}[!h]
	\centering
	\includegraphics[scale=.55]{imagenes/estructuraUDP.png}
	\caption{Estructura de un paquete UDP con detalle del tamaño de sus encabezados de capa de transporte.}
	\label{fig:datagramaudp}
\end{figure}

El protocolo UDP se estandarizó el año 1980, sin embargo su aplicación ha sido muy variada en distintos sistemas modernos. Hoy por hoy UDP es uno de los componentes estructurales de la Internet estando presente en diversas aplicaciones que van desde transmisiones en uso intensivo de datos para redes de alta velocidad \cite{udp:highbandwidth}, hasta mecanismos de transmisión de video \cite{udp:video}, entre otros. Sin embargo, una de las aplicaciones más importantes (sino la más trascendental) de éste protocolo en la infraestructura de Internet, es su labor en el servicio DNS.

\subsection*{DNS}
\label{section:dns}
Todo dispositivo conectado a una red de computadoras se identifica a si mismo por la denominada \emph{dirección IP}, un identificador que permite referenciarlo y diferenciarlo de otros dispositivos conectados a la misma red, definido por el nivel 3 del modelo OSI. No obstante para poder conectarse con un recurso disponible en Internet, generalmente los dispositivos consultan por lo que llamamos \emph{nombres de dominio}, que son identificadores con carácter semántico para los usuarios que definen una red de identificación asociada a un grupo de dispositivos o equipos conectados a la red \cite{wiki:nombre_dominio}. Éste mecanismo de traducciones se conoce como \textbf{servicio DNS} y es vital para el funcionamiento de Internet como lo conocemos.

El servicio DNS \cite{rfc:1034, rfc:1035} opera como una base de datos distribuida que permite resolver consultas de manera jerarquizada basado en un principio de consultas recursivas. Frente a una consulta por un nombre de dominio, el primer paso es la resolución de la misma empleando datos de la caché local del servicio DNS del sistema (Esto es, emplear información previa en caso de que el nombre consultado ya hubiese sido preguntado anteriormente). En caso de encontrarse la respuesta en la caché local, se responde la consulta con ésta información y el proceso finaliza. En caso de que la consulta no coincida con algún registro local, se promueve el proceso de resolución de nombre al servidor DNS preferido (Configurado en el sistema).  

Frente a una consulta, un servidor DNS hace una verificación en su información local para determinar si tiene autoridad para responder al nombre solicitado en función de la zona que tenga configurada. En caso de que el nombre consultado coincida con algún registro de dicho servidor, el servidor usa ésta información para responder la consulta con autoridad y finaliza el proceso. En caso de que éste servidor no disponga de una entrada con el nombre pedido, se verifica en su caché local en caso de que previamente el servidor hubiese formado parte de una cadena de resolución para el nombre pedido, en cuyo caso puede responder con la resolución almacenada en caché y finalizar la consulta. Si tras todas las verificaciones anteriores (Sobre los registros de la zona DNS y de caché de resoluciones) el nombre de dominio consultado no registra apariciones en el servidor DNS preferido, la consulta se promueve recursivamente a otros servidores DNS con sugerencias de raíz para el nombre solicitado, de manera de poder conseguir una respuesta autoritativa en cada caso. Para esto la consulta se resuelve desde el dominio de nivel superior, resolviendo desde lo general a lo particular del nombre de dominio. Un ejemplo de resolución del dominio dcc.uchile.cl se ilustra en la imagen \ref{fig:dns}

\begin{figure}[!h]
	\centering
	\includegraphics[scale=0.75]{imagenes/dns-system.png}
	\caption{Diagrama de operación del servicio DNS ilustrando comunicación entre servidores de dicho tipo.}
	\label{fig:dns}
\end{figure}

En la práctica, el servicio DNS emplea el protocolo UDP para la comunicación desde y hacia los servidores de dicho tipo, esto pues las características de una consulta DNS contemplan:

\begin{itemize}
\item \textbf{Consultas Auto-Contenidas} La información de una consulta DNS calza fácilmente en un paquete de UDP.
\item \textbf{Orden Irrelevante} Los requerimientos DNS no necesitan ser procesados en un orden establecido. Sin embargo, si requieren ser eventualmente procesados y ello en el menor tiempo posible. Un factor que justifica el uso de UDP al ser precisamente un protocolo orientado a mensajes.
\end{itemize}

De esta manera, las características anteriores justifican el uso de UDP como protocolo para la comunicación entre los servidores DNS.

El escenario antes descrito explica el por qué UDP tiene una participación de gran importancia en las redes actuales (en especial sobre Internet) y además, dado el sostenido aumento en el número de conexiones explicado en las secciones previas, da cuenta de cómo el tráfico de éste tipo de comunicación representa una porción muy significativa en la operación de las redes modernas. Una situación que con el paso de los años ha exigido cada vez mejores técnicas de procesamiento para dar mejores tiempos de respuesta.

\section*{Planteamineto de Investigación}

La pregunta natural entonces es: ¿Están capacitados los servidores DNS para responder tal cantidad de peticiones correctamente? Más allá de la respuesta a ésta interrogante el hecho es que, para su correcta operación, los servidores DNS deben garantizar bajos tiempos de respuesta y eficiencia en su operación en todo escenario.

Un excelente enfoque para abordar la problemática anterior mora en usar paralelismo, ósea, aprovechar la independencia entre peticiones DNS para procesar varios requerimientos a la vez de manera simultánea. En la práctica, dicha solución propone compartir una misma interfaz de conexión (denominada \emph{Socket}) entre varios hilos de ejecución. Ésta práctica supone --en teoría-- un incremento en el rendimiento de operación del procesamiento de las peticiones DNS, siempre que existan procesadores disponibles para atender cada hilo de forma independiente.

Sin embargo, los resultados al evaluar el enfoque antes propuesto revelan un preocupante e insospechado panorama donde el rendimiento de los servidores DNS no escala al incorporar directamente paralelismo en el procesamiento de peticiones como se esperaría \cite{tesis:diegoDCC}. Éste fenómeno ha sido repasado por distintos grupos de investigación en el área como Facebook, Toshiba \cite{post:facebook, paper:toshiba} e incluso industrias nacionales como NIC Chile, pero sin mayor solución más que delegar dicha responsabilidad a unidades estructurales del sistema operativo (como el kernel del sistema operativo) de donde tampoco se han conseguido mayores mejoras.

La pregunta entonces es ¿Por qué el sistema de procesamiento de peticiones DNS no escala al incorporar paralelismo directamente? Una sospecha interesante propone que probablemente el sistema de procesamiento de paquetes para UDP (que es el protocolo involucrado en las peticiones DNS) podría tener imperfecciones a nivel de diseño en el sistema operativo que no garantizarían su óptimo funcionamiento en condiciones de concurrencia, sugiriendo como responsable del problema a un defecto de contención a nivel de estructuras compartidas del sistema operativo. Sin embargo, dicha hipótesis no ha sido comprobada ni desmentida. Por otra parte, determinar la causa del problema anterior significa un tremendo desafío pues implica trabajar a nivel del núcleo del sistema operativo, que combina diversos paradigmas y enfoques de programación incorporados a lo largo de su desarrollo, unido a la dificultad inherente de trabajar analizando un sistema complejo como lo es el kernel mismo.

El presente trabajo plantea precisamente un estudio amplio que reune las principales sospechas de investigación vigentes del problema al momento de su desarrollo, de manera de identificar y analizar las distintas componentes responsables del problema detectado, así como también el estudio de alternativas para palear dicha situación.


\end{intro}
%\input{cap1.tex}
%\input{cap2.tex}
\chapter{Estudio de Distribución de Carga}
La tercera hipótesis de investigación plantea como responsable del mal rendimiento presentado en el caso de estudio al sistema de gestión y administración de tareas en el sistema operativo. En escenarios multicore como el estudiado, es normal que el sistema operativo realice como procedimiento de rutina la migración de procesos y la re-locación de recursos y datos para evitar la saturación de las componentes del mismo. Un caso práctico de ello es cuando un núcleo de procesamiento está sobre exigido y el sistema operativo redistribuye los procesos que están ejecutándose en dicho núcleo entre los otros procesadores disponibles del sistema con el costo que ello significa. Ésta práctica es conocida como \textbf{distribución de carga}, y a pesar de que existen varios mecanismos de aprovechamiento de dicho esquema como una estrategia de balanceo de carga en arquitecturas como la estudiada \cite{paper:NUMA}, en ciertos escenarios puede degradar el desempeño general del sistema.

Esta hipótesis plantea que dicho proceso de reasignación de recursos sería perjudicial en escenarios de concurrencia sobre arquitecturas como la estudiada, basándose en que mientras un proceso está en plena ejecución, al incorporar más y más tareas en el mismo núcleo de procesamiento agregando nuevos hilos de ejecución, sería el sistema operativo quien comenzaría la reasignación automática de dichos hilos entre los distintos procesadores cayendo en problemas como perdida de referencias de memoria en niveles de caché primario, yendo así en contra del principio de localidad de acceso a la memoria. En su peor escenario, ésta teoría lleva al ya mencionado problema de \emph{caché bouncing} que corresponde al fenómeno de sobre corrección de los datos a nivel de líneas de cache de un procesador, producido por constantes cambios de contexto del \emph{scheduller} que genera migración de procesos. Un problema que ya se mencionó en el estudio de la sección previa.

Una alternativa que se ha estudiado para solventar éste problema es la técnica de \emph{Processor Affinity} que consiste en la asociación de tareas o procesos en CPUs específicas, de manera de controlar la ubicación de memoria y zona real de ejecución del código en la máquina.
\begin{defn}[ver \cite{article:processoraffinity}] \textbf{Processor Affinity} es una estrategia de trabajo que consiste en la asociación de ciertos procesos o hilos de trabajo con determinados núcleos de procesamiento lógicos de un sistema. Dicha asociación restringe la capacidad de ejecución del hilo o proceso exclusivamente a su núcleo de procesamiento asignado.
\end{defn}
En otras palabras, con la aplicación de \emph{processor affinity} se remueve la utilidad del mismo scheduler del sistema operativo para coordinar la mejor operación en la asignación de los hilos de ejecución a las distintas CPU disponible, reemplazándolo por un criterio de diseño humano construido consientes de la tarea que se desea realizar. Es una técnica muy ambiciosa en el sentido de que bien empleada puede proveer muy buenos resultados en el sistema \cite{paper:cacheaffinity}, sin embargo, es muy fácil errar al interpretar el diseño de operación que se desea coordinar, llevando a una mala implementación en la asignación de recursos que termina degradando fuertemente el desempeño del sistema completo.

En ésta tercer estudio se plantea la utilización de la técnica de \emph{processor affinity} en pos de conseguir un mejor rendimiento del caso de estudio presentado, ello por medio de la reasignación de los hilos de ejecución entre los distintos cores lógicos del sistema en pos de explotar mejor la localidad de recursos en una arquitectura como la que se dispone. Para evaluar lo anterior, se plantean distintos esquemas de asignación de recursos basados en argumentos arquitecturales del escenario de prueba y se evalúan comparativamente los resultados.

\section{Esquemas de Distribución}
Se diseñaron variados esquemas de asignación a fin de evaluar distintos enfoques de aprovechamiento del principio de localidad de memoria. En total se diseñaron 6 esquemas para evaluar combinaciones dinámicas de cores lógicos reconocidos por el sistema operativo en búsqueda de mejores rendimientos. Los esquemas se detallan a continuación:
\begin{description}
\item[Sin Processor Affinity] Asignación dinámica por el scheduller del sistema operativo. En éste caso, la elección la hace el sistema de acuerdo a complejos algoritmos que consideran prioridad de proceso, carga de CPU, entre otros. Sirve como punto base de comparación con respecto a los demás esquemas.
\item[DummyAffinity] Asignación directa de hilos a ejecución en el core 0 del sistema, así, todos los hilos se delegan al mismo core. Bajo éste esquema se presume que se puede aprovechar mejor la localidad de memoria al disponer en el banco de memoria más próximo al núcleo de ejecución todas las referencias necesarias, disminuyendo el efecto de sobrecarga de los protocolos de coherencia y coordinación de cache.
\item[EquitativeAffinity] Asignación secuencial equitativa entre los cores lógicos del sistema siguiendo la numeración que el sistema operativo dispone de los núcleos mismos. Supone que la distribución completamente justa y equitativa entre cores entrega un mejor rendimiento general al reducir la carga de procesamiento de cada core.
\item[PairAffinity] Asignación secuencial de hilos a cores de numeración par. Busca descartar un escenario donde los cores duplicados por efecto de la tecnología \emph{hyperthreading} de Intel pudieran no aprovechar cores reales, generando así un mejor desempeño final.
\item[ImpairAffinity] Asignación secuencial de hilos a cores de numeración impar. Similar al anterior variando los cores elegidos.
\item[NumaPairAffinity] Asignación de los hilos a cores del segundo conjunto NUMA disponible en el sistema. Sigue la idea de aprovechar toda una unidad lógica de procesamiento según la arquitectura \emph{Quickpath}, persiguiendo mejores tiempos al tener mejor acceso a memoria desde éstos cores.
\item[SimpleCoreAffinity] Asignación de los hilos a los primetos dos cores lógicos del sistema, correspondientes al primer core real duplicado por efecto \emph{HyperThreading}.
\end{description}

\begin{table}[]
\centering
\begin{tabular}{|l|l|}
\hline
\multicolumn{1}{|c|}{\textbf{Esquema}} & \multicolumn{1}{c|}{\textbf{Fórmula de numeración de afinidad por hilos}}                                  \\ \hline
DummyAffinity                          & 

$\displaystyle 0$
                                                                                            \\ \hline
EquitativeAffinity                     & $\displaystyle i+1 \pmod \Phi $                                                                                 \\ \hline
PairAffinity                           & $\displaystyle 2i \pmod \Phi $                                                                                 \\ \hline
ImpairAffinity                         & $\displaystyle 2i + 1 \pmod \Phi $                                                                               \\ \hline
SimpleCoreAffinity                     & $\displaystyle \frac{i \pmod 2}{2} \Phi$                                                                              
\\ \hline
NumaPairAffinity                       & 
$
\begin{cases}
    \displaystyle i \pmod \Phi & \displaystyle \text{si } i \pmod 2 = 0 \\
    \displaystyle (i-1) + \frac{\Phi}{2} \pmod \Phi & \displaystyle \text{si } i \pmod 2 \neq 0
\end{cases}
$ \\ \hline
\end{tabular}
\caption{Numeración de afinidad a cada hilo en ejecución, donde $i$ representa el i-ésimo hilo. y $\Phi$ corresponde al número de procesadores lógicos reconocidos en el sistema (24 para nuestro caso).}
\label{my-label}
\end{table}

Con los esquemas antes mostrados, se pretende explotar las bondades de la arquitectura NUMA, de los principios de localidad de memoria, y de proximidad de acceso en los datos, a través de una ganancia efectiva en los tiempos de la prueba del caso de estudio.

\section{Metodología de Experimentación}
Para evaluar los distintos esquemas de afinidad ya postulados, se modificó el caso de estudio para incorporar la asociación de hilos de ejecución para con los distintos cores del sistema. Para ello se aprovechó la flexibilidad de la librería \emph{PThreads}, responsable de brindar los hilos de ejecución en nuestro caso, para usar ciertas llamadas a sistema que nos otorgasen el control de la localidad de ejecución buscada. En concreto, se empleó la función \verb=pthread_attr_setaffinity_np()= que permite especificar precisamente la asociación de un hilo para con el core de ejecución deseado. 

La prueba anterior se recopiló en un nuevo experimento\footnote{\url{https://github.com/sebablasko/Test_DifferentAffinityThreadsBySocket}} que, para efectos de validez estadística, se ejecutó un total de 60 veces, y se rescataron los valores promedios de dichas ejecuciones, ello siguiendo las mismas lógicas de colección de datos del promedio de las ecuaciones \ref{eq:tupla1} y \ref{eq:promedio}.

\newpage

\section{Resultados}

\begin{figure}[h!]
	\centering
	\includegraphics[scale=.6]{resultados/processoraffinity-crop.pdf}
	\caption{Resultados experimentales de los distintos esquemas de afinidad.}
	\label{fig:resAffinity}
\end{figure}

\section{Análisis y Discusión de Resultados}
Los resultados experimentales ilustrados en la imagen \ref{fig:resAffinity} son diversos. En efecto, ningún esquema de distribución evaluado consiguió reducir verdaderamente el tiempo de ejecución con respecto al uso de una configuración trivial de un único thread. Sin embargo los resultados obtenidos reflejan distintas tendencias interesantes en los tiempos resultantes del experimento.

Una primera tendencia evidenciada en los resultados viene dado por los esquemas \textbf{DummyAffinity} y \textbf{SimpleCoreAffinity}, donde los tiempos se mantuvieron estables a lo largo de la evaluación de las distintas configuraciones de threads. Si recordamos, ambos esquemas son rígidos en la posibilidad de locación de los hilos en la CPU, el primero los asigna todos al core \#0 mientras que el segundo los asigna todos entre el core \#0 y el core \#12, numeración que da el sistema operativo a los núcleos duplicados por efecto de la tecnología \emph{HyperThreading}. De ésta manera, ambos esquemas tienen en común que ejecutan la totalidad de los hilos en un mismo núcleo real de procesamiento. Ahora bien, ninguno de los dos consigue ganancias efectivas de tiempo con respecto al uso de un único thread, ello se explica pues, dado que los distintos hilos son asignados todos a un mismo core, la ejecución de los mismos termina degenrando a una ejecución secuencial, es decir, por medio de la afinidad de proceso hemos eliminado la capacidad de paralelismo al llevar el experimento a un escenario monocore. Esto explica también el por qué los tiempos son tan uniformes sin importar la cantidad de hilos que se usen.

Una segunda tendencia reconocible es la ilustrada por el esquema \textbf{EquitativeAffinity}. Resultante como el de peor desempeño, éste esquema que se caracterizaba por la distribución justa de los hilos entre los cores registra siempre los peores tiempos de la prueba. Ésta tendencia se explica por la naturaleza de la arquitectura que se está usando. Como se mencionó antes, la distribución del sistema corresponde a dos nodos NUMA de procesamiento. La aplicación del esquema equitativo vulnera derechamente tanto los beneficios que provee el esquema NUMA como el principio mismo de localidad en acceso a la memoria, ello pues en la práctica, termina distribuyendo los distintos hilos en las componentes más alejadas (topológicamente) de la arquitectura disponible.

Finalmente, una tercera tendencia es la de crecimiento producido por el conjunto de los demás esquemas de afinidad. La similitud entre \textbf{PairAffinity} e \textbf{ImpairAffinity} era previsible pues el modelo de distribución era equivalente para ambos. La misma naturaleza sigue el esquema \textbf{NumaPairAffinity} que, a pesar de aprovechar la distribución de componentes para la localización de hilos, no consiguió mejores tiempos que los ya representados. Ahora bien, hay que destacar que en términos prácticos, éste último grupo de esquemas de distribución tuvo un rendimiento muy similar al del mismísimo scheduller del sistema operativo, más aún, con la técnica de \emph{processor affinity}, la ejecución del programa es rígida en la localidad del proceso con respecto al procesador seleccionado, lo cual produce que en ciertas situaciones --como al tener sobrecarga de memoria-- los niveles de caché no cooperen con la tarea general, y no brinden mayor beneficio a la prueba final, una garantía que si dispone el scheduller del sistema operativo, por lo tanto su desempeño no debe ser menospreciado y sólo da cuenta de que lamentablemente, el problema de rendimiento parece ser producto de un defecto de diseño inherente a la estructura que se está compartiendo, defecto que no permite optar a mejores tiempos sin importar la estrategia de paralelismo usada.


\section{Conclusiones}
A raíz del estudio de distribución de carga se pueden rescatar varios aspectos interesantes:
\begin{itemize}
\item Las técnicas de \emph{processor affinity} empleadas no consiguieron beneficio alguno en el rendimiento de consumo del caso de estudio, ello a pesar de evaluar distintos esquemas en pos de sacar provecho de la arquitectura evaluada.
\item Los distintos esquemas, a pesar de ser rígidos en la ejecución de hilos sobre ciertos cores, fueron competitivos con respecto a la capacidad dinámica del scheduller del sistema operativo en los tiempos producidos en el experimento.
\item Se determina que es el mismo socket –o alguna componente inherente de sincronización como su spinlock-- una estructura que se vuelve un punto de contención al emplear hilos paralelos, calificando al mismo como no apto para soportar accesos concurrentes, ello dado su diseño y mecanismos de protección implementados.
\end{itemize}
%\chapter{Estudio del Problema}

Como se mencionó al final del capítulo anterior, la investigación vigente del fenómeno estudiado en el presente trabajo ha concluido en generar distintas hipótesis para explicar el mal rendimiento presentado por los sockets de internet en escenarios de concurrencia, más sin corroborar la validez de ninguna experimentalmente. Para ser exactos, y reiterando lo anterior, las sospechas se pueden agrupar en tres aristas distintas: Problemas de operación en primitivas del sistema, problemas de operación a nivel de canales de comunicación de hardware y problemas de operación en los mecanísmos de administración de recursos y balanceo de carga.

En el presente capítulo se estudian las tres líneas hipotéticas responsables del problema, inspeccionando en distintos niveles el funcionamiento del sistema, indagando en las operaciones teóricas de cada funcionamiento y contrastando dicho planteamiento a resultados experimentales obtenidos en cada caso. Es fruto de cada una de las siguientes subsecciones un análisis profundo del aspecto problemático estudiado, junto con una conclusión sobre el mismo. Así también, es fruto del presente capítulo un veredicto sobre cuales -Y en qué medida- de los problemas estudiados tienen verdadera responsabilidad en el fenómeno estudiado, de modo de confeccionar un marco de trabajo que nos permita empezar a postular enfoques de trabajo que paleen la problemática en cuestión.

\section{Estudio de Operación Primitivas de sincronización del Sistema}

La primera hipótesis a estudiar plantea que el bajo rendimiento de la operación de la interfaz de red -Ilustrada en nuestro caso de estudio por medio de sockets UDP- en escenarios de concurrencia, es causado por un mal desempeño de las estructuras que proveen los mecanismos de sincronización para dichos escenarios. Cómo se mencionó en el capítulo anterior, la capacidad multiprocesador de las computadoras modernas provee de un mayor poder de cómputo que se postula a ser aprovechado por medio del uso de técnicas de programación paralela, con el cuidado de que, en esos contextos de trabajo, los sistemas operativos deben estar preparados para atender situaciones de conflicto en el acceso a los recursos compartidos. Para éste propósito, se disponen de los mecanísmos de sincronización ya repasados en secciones anteriores que para estructuras de bajo nivel, cómo son los sockets de internet provistos por el propio sistema operativo- emplean el uso de mecanísmos de sincronización de bajo nivel como lo son los spinlocks, que protegen ciertas secciones de la estructura, tal y como se repasó en la sección XXXXX.

En éste caso, la priemra hipótesis describe que el causante del mal rendimiento al incorporar concurrencia en las lecturas a un socket es generado derechamente por dichas estructuras de protección en el acceso, situación causaría un fenómeno denominado \emph{Contención de Recurso}, que corresponde a un estado de conflicto en el acceso a un recurso compartido, producido por una situación de competencia en el acceso al mismo.

Para ratificar el planteamiento anterior, se hizo un estudio de llamadas a sistema siguiendo otros modelos de recopilación de datos ya evaluados \cite{slides:hpPerf} que permitiese vislumbrar la operación de las primitivas de sincronización operativas en el caso de estudio, a medida que se van agregando hilos de procesamiento, generando así un escenario de concurrencia, cómo se propuso desde un comienzo.

\subsection{Estudio de Llamadas de sistema}

La operación de las primitivas de sincronización que actúan en los procesos de bajo nivel del sistema operativo tienen la característica de estar determinadas por el uso de llamadas a funciones del sistema, ello pues es el mismo sistema operativo (o mejor dicho su núcleo) el que provee una interfaz simple para invocar dicha operación. Cómo son llamadas a sistema, es posible cuantificar cuando y cómo se realiazan las mismas, pudiendo modelar el proceso completo por medio de éste mecanismo.

Cómo en nuestro caso interesa estudiar el comportamiento de primitivas de sincronización de bajo nivel como son los spinlocks, se debe contemplar la API\footnote{Abreviatura de \emph{Application Programming Interface}, ó Interfaz de Programación de Aplicaciones en español.} con que trabaja el sistema para controlar éstas estructuras, ello pues, a pesar de que la estructura spinlock está bien definida, eisten distintas funciones que proveen variantes en el funcionamiento de los spinlocks, y dichos escenarios son presentables a lo largo de la ejecución del caso de prueba del estudio. El objetivo de éste estudio concierne un análisis cuantitativo de la cantidad de llamadas a sistema que sean bloqueantes sobre estructuras bloqueantes de tipo spinlock, y del tiempo que el sistema gasta en dichas condiciones.

En Linux los spinlocks se representan con estructuras \verb=spinlock_t= (incluidas en el archivo \verb= <linux/spinlock.h>=) que básicamente consisten en un campo de lock con un valor 1 (si está libre) o 0 (si está ocupado). Existen diversas funciones de atención que aplican distintos tipos de bloqueo \cite{book:spinlocks}:

\begin{description}
\item[void spin\_lock\_init(spinlock\_t *lock);] Inicializa una estructura spinlock y setea su valor inicial en 1.
\item[void spin\_lock(spinlock\_t *lock);] Es el bloqueo básico del sistema para tomar el lock. Consistente en la espera del lock hasta su valor 1 para luego setearlo en 0. Dicha espera se realiza con ciclos de \emph{busy-waiting} hasta que se brinde acceso. Es un bloqueo interrumpible por el sistema operativo, tanto por interrupciones de software como de hardware, dando paso a situaciones como que la CPU determine enviar el proceso responsable de la llamada a dormir por falta de recursos, memoria, etc.
\item[void spin\_lock\_irq(spinlock\_t *lock);] Bloqueo que deshabilita interrupciones del procesador local antes de adquirir el spinlock. Se debe cuidar de reactivar las interrupciones luego de liberado el lock.
\item[void spin\_lock\_irqsave(spinlock\_t *lock, unsigned long flags);] Similar a la operación de \verb=spin_lock_irq=, pero con la diferencia de que almacena el estado de interrupción previo en el valor \verb=flags=, de manera de que puede restablecerlo facilmente luego de liberar el lock.
\item[void spin\_lock\_bh(spinlock\_t *lock)] Similar a \verb=spin_lock_irq= con la diferencia de que sólo deshabilita las interrupciones de software, manteniendo habilitadas las interrupciones por hardware del sistema.
\item[int spin\_trylock(spinlock\_t *lock);] Para operaciones no bloqueantes para el uso de spinlocks. Retorna cero en caso de fallo al obtener el lock. No deshabilita interrupciones.
\item[bool mutex\_spin\_on\_owner(struct mutex *lock, struct task\_struct *owner)] Bloqueo que opera sobre una estructura de exclusión mutua (mutex) que utiliza el enfoque de \emph{Read-Copy-Update} (RCU), en donde los lectores son no bloqueantes. Ésta estructura tiene una sobrecarga menor que las anteriores; Sin embargo, las actualizaciones son más costosas ya que las versiones anteriores de la estructura de datos se deben guardar con el fin de dar cabida a los lectores ya existentes que se sincronizan a través de las barreras del mutex. Utilizando el enfoque de la RCU el bloqueo con esta estructura mutex asegura que la operación \emph{Test-and-Set} se ejecute en la misma CPU del propietario del lock, lo que reduce la cantidad de comunicación de memoria caché (y por consiguiente, el efecto de contención).
\end{description}

Asociadas a las anteriores llamadas de sistema están las variantes \verb=*_unlock*= que permiten liberar el elemento de bloqueo (seteando el valor del lock a 1) para recuperar así su disponibilidad para otros procesos.

Para poder rescatar las llamadas a sistema existen herramientas de software de bajo nivel, creados por los mismos desarrolladores del núcleo de Linux, que permiten realizar la tarea que nos proponemos en éste caso.

\subsubsection{Perf}
Perf \cite{slides:perfTools} o también llamado \emph{Perf\_events\footnote{Mayor documentación disponible en \url{https://perf.wiki.kernel.org/}}} es una herramienta de análisis de performance para entornos Linux. Corresponde a un subsistema del mismo kernel de Linux que provee todo un framework para el estudio de performance del sistema y de programas por medio de la captura de una amplia variedad de fuentes de datos. Perf es capaz de colectar datos por operatividad de software (contadores de software, \emph{tracepoints}, ejecución de funciones, paso a assembler, etc.) y también colectar información a nivel de hardware (manejo de PMU, lectura de \emph{Performance Counters}, etc.), características que lo postulan como uno de los sistemas más completos y flexibles para las tareas de profilling de aplicaciones y sistemas, y que lo hacen una buena herramienta para el actual estudio.

\begin{figure}[!h]
	\centering
	\includegraphics[scale=.3]{imagenes/fcfm}
	\caption{Arquitectura de operación del framework provisto por \emph{Perf}.}
	\label{fig:perfFramework}
\end{figure}

Además de su gran capacidad para colectar datos, Perf es una herramienta de sencillo uso, pues su funcionamiento se basa en la supervisión de un determinado proceso o tarea de la cual construye un archivo con la información que se haya seleccionado a colectar \cite{article:perf}. Posteriormente, se pueden realizar operaciones de análisis más exhaustivo sobre dichos archivos de resultados.

El potencial de ésta herramienta la perfila como una utilidad indispensable para el estudio en cuestión. En primer lugar por su capacidad de análisis de ejecución de código que permite obtener información cuantificada de las llamadas a sistema y de la dinámica del árbol de llamados\footnote{Acá explicar brevemente que es un árbol de llamados} que permite reconocer la naturaleza de las funciones involucradas en el caso de estudio. En segunda instancia Perf es una estupenda herramienta para la recolección de datos de hardware al aprovechar el uso de la \emph{Performance Monitoring Unit} del hardware del sistema, una característica que será revisada en detalle en secciones posteriores.

\begin{figure}[!h]
	\centering
	\includegraphics[scale=.3]{imagenes/fcfm}
	\caption{Esquema de captura de datos de un programa usando el comando \emph{perf-record}.}
	\label{fig:perfRecord}
\end{figure}

\subsubsection{FTrace}
Ftrace\footnote{Mayor documentación disponible en \url{http://elinux.org/Ftrace}} es otra poderosa herramienta para estudios de profiling disponible para sistemas Linux \cite{paper:FTraceSony}. Su funcionamiento opera de naturaleza muy intima con respceto al kernel mismo del sistema pues su recolección de datos se basa en el rastreo de la ejecución de funciones de forma dinámica en el espacio de kernel, lo que lo hace una estupénda utilidad para el estudio de llamadas al sistema, pudiendo recuperar datos como el tiempo de ejecución y cantidad de ocurrencia de las mismas.

Para su uso, FTrace opera como un verdadero framework del sistema sobre el kernel, del cual se pueden usar distintos métodos de rastreo de llamadas basados en distintos algoritmos. Una de las funciones más poderosas de FTrace es el resultado que se puede obtener por medio de la instrumentación de código, que se refiere a la práctica de incorporar a los programas a analizar \emph{tracepoints}, que son declaraciones explicitas de secciones de código a analizar y registrar. A pesar de que ésta característica es muy cómoda para programas propios, en el caso del análisis de funciones y llamadas de sistema la instrumentación de código es una característica obviable, siendo sólo necesaria la precisión de qué llamadas considerar en el análisis. El uso de ésta herramienta es muy flexible, siendo activable a disposición del usuario y conservando un registro de resultados. Además, FTrace es altamente configurable pudiendo explicitar filtros que usar como registros para las llamadas de sistema a analizar.

El provecho que se puede sacar de ésta herramienta es usar su capacidad para cuantificar tiempo de funciones del kernel para estudiar la atomicidad de las llamadas bloqueantes del sistema. Así por ejemplo, se pretende determinar el tiempo que se pasa en estados bloqueantes de spinlocks (de \emph{busy-waiting}) en los cuales sólo se pierde tiempo por caso de contención.


\subsection{Resultados}
Nos topamos con que las tendencias de tiempos son similares a las de los tiempos de esas llamadas, luego, hay una correlacion de esta estructura inhenrente al socket es la que causa el cuello de botella: indicio, el socket entero está siendo contenido por el spinlock de su estructura.

\subsubsection{Perf}
\subsubsection{FTrace}

\subsection{Análisis y Discusión de Resultados}
Aka un análisis general de los resultados en términis de gráficos obtenidos

\subsubsection{Call-Graph}
Una visualización de call-graph, quizá no como subseccion

\subsubsection{TraceDisplay}
Para poder obtener una interpretación adicional del fenómeno reconocido, se implementó una herramienta de visualziación de las llamadas a sistema para funciones de sincronización que permitiese reconocer las porciones de tiempo que tomaban en cada procesador dichas funciones. Para ello, la herramienta recibe un log de \emph{FTrace} con las llamadas de sistema yá filtradas y construye un mapa de tiempo coloreado, donde se pueden apreciar las porciones de tiempo que consumen las llamadas y desde que CPU se originan. El resultado se puede apreciar en la imagen \ref{fig:traceDisplay}. Éste subproducto de la investigación principal junto con su documentación de uso está publicado\footnote{Disponible en \url{https://github.com/sebablasko/TraceDisplay}} y disponible para su uso.

\begin{figure}[!h]
	\centering
	\includegraphics[scale=0.35]{imagenes/traceVisualization.png}
	\caption{visualización de aplicación de llamadas de sistema de sincronización realizadas entre procesadores, generada con la herramienta TraceDisplay desarrollada en el marco de éste trabajo. AKA FALTA EXPLICAR EL CASO EVALUADO}
	\label{fig:traceDisplay}
\end{figure}

AKA ME FALTA DECIR PARA QUE ME SIRVIO ESTA HERRAMIENTA



\section{Estudio de Canales de Comunicación de Hardware}
La segunda hipótesis para explicar la mala performance del caso de estudio presentado se centra en una componente de hardware más que de software. Cómo ya se mencionó, la capacidad multiprocesador de que se dispone en equipos modernos no es un recurso facil de aprovechar, de hecho, requiere una sofisticada operación y diseño tanto de las aplicaciones que solicitarán recursos, como del sistema operativo que ha de administrarlos. Como se repaso en secciones anteriores, en la práctica, la capacidad de paralelísmo viene dada gracias a un conjunto de protocolos y algoritmos de muy bajo nivel que coordinan y mantienen coherentes las componentes de datos para los diferentes procesadores \cite{paper:MESI, paper:snoop}, sin embargo, por muy sofisticados que dichos mecanismos sean, las nuevas tecnologías de hardware que prometen velocidades de trasnferencia y acceso nunca antes imaginiadas podrían significar un problema para éstas componentes.

Es precisamente en ésta línea que se establece la segunda hipótesis. En éste caso, se responsabiliza por el mal rendimiento a un problema de contención de recursos  nuevamente al spinlock de los sockets, pero ésta vez relacionado a la persistencia en el acceso al mismo y a la disponibilidad que se da del mismo a través de los mecanismos antes mencionados. En las arquitecturas modernas, los protocolos \emph{MESI} y de  \emph{SNOOP} son cruciales en la operación de ejecuciones paralelas para garantizar integridad en los datos, pero las arquitecturas mdoernas proponen nuevas distribuciones de los componentes internos de hardware, dotando de canales de mayor velocidad y reasignando los recursos. Ésta hipótesis plantea la posibilidad de que el degradamiento del caso de estudio sea causado por un fenómeno de \emph{Caché Bouncing}, que consiste en un 

En ésta línea, el fenómeno de \emph{Cache Bouncing} se podría acrecentar dada la arquitectura del sistema, la que al contemplar bancos de memoria distribuidos, algunos compartidos y otros exclusivos para los núcleos de procesamiento podría estar manifestandose como resultado de las modificaciones concurrentes de los distintos procesadores sobre el socket compartido, combinado a la operación de los protocolos de consistencia y correctitud para las lineas de caché del sistema que perfectamente pueden generar un overhead que termine sobrecargando los timepos totales.

Para validar la hipótesis anterior, es preciso un cabal entendimiento de la arquitectura de hardware objeto de estudio, a fin de poder localizar puntos de contención, junto con una comprensión importante de la \emph{Performance Monitoring Unit} que provee el fabricante, lograr configurarla y aprovecharla para la recolección de datos finales. En las siguientes secciones se realiza un estudio de la arquitectura descrita en la figura \ref{fig:pc3} del equipo sobre el que se realizan las pruebas experimentales reales. Posteriormente se realiza un análisis experimental de las tendencias presentes en una tarea de acceso concurrente como aldescrita en el caso de estudio de ésta investigación con el fin de corroborar o descartar las sospechas ya mencionadas del efecto de contención por eventos de perfomance de hardware.

\subsection{Características de Arquitecturas de Hardware Modernas}
Cómo ya se mencionó en secciones anteriores, los fabricantes de partes y piezas de computadoras están constantemente desarrollando importantes avances de la mano con el desarrollo técnico que brinda mejores componentes de hardware cada día, y la linea de la infraestructura de hardware principal de los computadores no está excenta de dicha evolución. En secciones anteriores se presentó como las arquitecturas han evolucionado desde el primer esquema \emph{SMP} propuesto con la distribución \emph{FSB}, pasando luego por nuevas configuraciones como \emph{DIB} y \emph{DHSI}, entre otras.

\subsubsection{Arquitectura Intel QuickPath}
Al año 2008, el fabricante de procesadores Intel® lanzó al mercado una nueva tecnología denominada \emph{Intel® QuickPath Architecture} la que planteaba un nuevo esquema organizacional de los componentes internos de la placa principal de las computadoras, así como también un nuevo esquema de conectividad entre los componentes de la misma, prometiendo entre otras cosas: un sistemas más confiable, eficiente, rápido y escalable, que podría aprovechar mejor la capacidad de los procesadores de su misma linea [AKA CITA AL QUICKPATH]. La apuesta de Intel resultó todo un éxito. Rápidamente \emph{QuickPath} se impuso por sobre la tecnología \emph{HyperTransport} desarrollada por \emph{AMD}, consiguiendo una improtante cuota del mercado en procesadores multicore [AKA REF A ESO].

\begin{figure}[!h]
	\centering
	\includegraphics[scale=.3]{imagenes/fcfm}
	\caption{Diseño organizacional de los componentes de sistema en una arquitectura \emph{QuickPath} de Intel.}
	\label{fig:quickpath}
\end{figure}

El esquema \emph{QuickPath} postula una reformación arquitectural de los componentes principales de un sistema \ref{fig:quickpath}. En éste esquema, las distintas unidades de procesamiento (CPU) están interconectadas por canales de comunicación especiales denominados \emph{Intel® QuickPath Interconnect (QPI)} que son conexiones punto a punto entre CPU de enorme velocidad de transferencia que flexibiliza los protocolos de coherencia de caché (como MESI y modificaciones de cache-a-cache entre procesadores) dotándolos de mayor eficiencia y velocidad de acción, al ser una comunicación directa.
Por otro lado, en \emph{QuickPath} CPU dispone de su propio controlador de memoria y de un banco de memoria de acceso próximo. Dicho diseño se denomina \textbf{NUMA} de sus siglas en inglés \emph{\textbf{N}on \textbf{U}niform \textbf{M}emory \textbf{A}ccess} [CITA A NUMA] la que permite a las CPU de cada nodo NUMA disponer de un banco de memoria con un acceso garantizado más rápido que al que se tendría acceso en una arquitectura tradicional. El enfoque \emph{NUMA} se aprovecha del principio de localidad de memoria [CITA URGENTE], por la cual postula que los datos son separables en su acceso por las distintas CPU, logrando así mayor velocidad en el acceso a la memoria, y menor problemas de coherencia de la misma por modificaciones entre CPUs.

El diseño de Intel va más allá. Concientes de la necesidad de herramientas y utilidades para analizar la verdadera perfomance que provee ésta arquitectura, Intel provee unidades de monitoreo de perfomance (o \textbf{PMU}, por sus siglas en ingles \emph{\textbf{P}erformance} \textbf{M}onitoring \textbf{U}nit) que son componentes de hardware incorporado a los sistemas que permiten operaciones de inspección del nivel de comunicación entre componentes del sistema. A éste tipo de análisis de se denomína \emph{Estudio de Perfomance Counters}, dado que para poder realizar una medición, el fabricante de la PMU provee una colección de posibles \emph{Eventos} a colectar, con significaciones puntuales.

\begin{defn}[ver \cite{KAR00}] \textbf{Performance Counters} son identificadores de maquina que permiten ''contar'' determinados eventos a nivel de hardware, como lecturas de caché, corrección de lineas de caché, comunicación de protocolos de coherencia, etc. Usados para analizar el comportamiento de ciertas unidades de hardware y que conforman la base de las herramientas de profiling para el rastreo en el comportamiento de funciones de un sistema.
\end{defn}
 
QPI imple,enta la wea de 5 capas!!!!! importante ponerlo!

El estudio de performance counters corresponde a uno de los estudios de más bajo nivel realizables en pos de obtener datos que representen la forma de comunicación entre componentes del sistema. Ello lo hace también dificultoso pues amerita gran conocimiento de la arquitectura puntual sobre el sistema que se desea estudiar.

\subsection{Especificación y Captura de Eventos}
Para definir el marco conceptual de la prueba, se debe mantener presente el contexto de la hipótesis que fundamenta la misma. En éste caso, la motivación de éste estudio está en linea con entender el comportamiento de un comsumo concurrente en una estructura socket, o más precisamente ver cómo una instancia de una primitiva de sincronización --un spinlock-- se conporta en un escenario multithread, dado la misma podría alocarse en caches de distintos procesadores, y cómo se manifestaría la expresión de los protocolos de coherencia de cache en ese caso. Dichas circunstancias en un escenario mutiprocesador podrian dar cuenta de que el mal desempeño general del  caso de estudio tiene sus origines en los mismos sistemas de coordinación de bajo nivel del sistema.

\subsubsection{Arquitectura de la máquina para las pruebas}
Siguiendo el caso de prueba evaluado a lo largo de la investigación, se inspeccionará el caso de estudio de saturación de un socket UDP en el equipo XXXXXXXXX \ref{fig:hwspecs}. En éste caso, se cuenta con un equipo placa Dell Inc. 00NH4P A07, provisto de dos CPU Intel Xeon 5600 2.8Ghz, dotado de 6 cores cada uno. Cada CPU dispone de hasta tres niveles de cache de 192 kb, 1536kb y 12288kb respectivamente, que combinados con tres memorias de 4096MB (DDR3 1333MHz) cada uno, conforman 2 nodos NUMA, con un monto total de 24GB de memoria.

\begin{figure}[!h]
	\centering
	\includegraphics[scale=.3]{imagenes/fcfm}
	\caption{Esquema de arquitectura interna del esquipo XXXX sobre el que se realizan las pruebas de performance counters.}
	\label{fig:hwspecs}
\end{figure}

Por otro lado, al inspeccionar el sistema en detalle, se da cuenta de que para el modelo de procesador presente, se dispone de 5 unidades de PMU disponibles en el sistema, las cuales las separamos en 3 grupos:
\begin{description}
\item[PMU Genéricas] Incluyen a \verb=perf= y  \verb=perf_raw=. Disponen de las especificaciones estándar de perfomrance counters de la línea del software \emph{Perf}, lo cual las hace poco exactas en los valores descritos por cada evento y no necesariamente fieles a su descripción pues dependen en gran parte de que el fabricante sea riguroso en su implementación.
\item[PMU x86] PMU generacional de Intel para la línea x86 de Intel que incluye a \verb=ix86arch=. Dispone de eventos comunes a dicha linea de procesadores por lo que no da soporte especifico para la arquitectura \emph{quickPath} y del comportamiento multiprocesador.
\item[PMU westmere] PMU especificas de la línea Westmere [AKA UNA CITA] que soporta la base del desarrollo de la \emph{Intel QuickPath Architecture}, incluyendo a \verb=wsm_dp= y \verb=wsm_unc=. Es el nivel más exacto de PMU que provee el fabricante con los eventos más especificos y documentados del sistema.
\end{description}

\lstset{language=Bash,
		breaklines=true,
		frame=single
		}
\begin{lstlisting}[caption=Sacado con la super herramienta]
Detected PMU models:
[18, ix86arch, "Intel X86 architectural PMU", 6 events, 1 max encoding, 7 counters, core PMU]
[51, perf, "perf_events generic PMU", 104 events, 1 max encoding, 0 counters, OS generic PMU]
[53, wsm_dp, "Intel Westmere DP", 91 events, 2 max encoding, 7 counters, core PMU]
[54, wsm_unc, "Intel Westmere uncore", 52 events, 1 max encoding, 9 counters, uncore PMU]
[114, perf_raw, "perf_events raw PMU", 1 events, 1 max encoding, 0 counters, OS generic PMU]
\end{lstlisting}

\subsubsection{Metodología de captura de eventos}
Para la especificación de la captura de eventos, se debe prestar especial atención a la arquitectura interna en la comunicación interprocesador del sistema. En la figura \ref{fig:hwcomm} se da cuenta de los mecanísmos de comunicación y canales de que dispone el sistema estudiado. 

\begin{figure}[!h]
	\centering
	\includegraphics[scale=.3]{imagenes/fcfm}
	\caption{Esquema de arquitectura interna del equipo XXXX en las vías de comunicación del procesador que da cuenta de los principales puntos de alta comunicación en el escenario de cache bouncing por contención de valores.}
	\label{fig:hwcomm}
\end{figure}

La figura \ref{fig:hwcomm} da cuenta de los puntos en que se pueden suceder escenarios de congestión dados ya sea por un alto nivel de uso de los protocolos de coordinación de memoria. Dichos canales de comunicación podemos resumirlos en las siguientes funciones dedicadas, que están asociadas a ciertos \emph{Performance Counters} del sistema:
\begin{itemize}
\item Uso de QPI
\item Acciones de SNOOP
\item Pasos de data entre caches
\item Transiciones en MESI(F) Protocol
\end{itemize}

Con ello en mente, se consultó el manual del fabricante en búsqueda de documentación de los cerca de XXXXXX eventos disponible en el sistema.


[[[Hablar un poco de dicha busqueda y del manual]]]]

Finalmente, se confeccionó una colección de eventos (ver Tabla \ref{table:eventos}) según los cuales se debería tener una correcta representación del nivel de saturación y uso de los componentes.

\begin{table}[]
\centering
\begin{tabular}{|l|l|l|}
\hline
Nivel de Inspección                 & Registro & Descripción                                                \\ \hline
\multirow{8}{*}{Uso de QPI}         & r500104  & Cycles GQ data is imported from Quickpath interface        \\ \cline{2-3} 
                                    & r500204  & Cycles GQ data is imported from Quickpath memory interface \\ \cline{2-3} 
                                    & r500404  & Cycles GQ data is imported from LLC                        \\ \cline{2-3} 
                                    & r500105  & Cycles GQ data sent to the QPI or QMC                      \\ \cline{2-3} 
                                    & r500205  & Cycles GQ data sent to LLC                                 \\ \cline{2-3} 
                                    & r500405  & Cycles GQ data sent to cores                               \\ \cline{2-3} 
                                    & r500420  & Quickpath Home Logic remote read requests                  \\ \cline{2-3} 
                                    & r500820  & Quickpath Home Logic remote write requests                 \\ \hline
\multirow{3}{*}{Snoop}              & r530451  & L1D cache lines replaced in M state                        \\ \cline{2-3} 
                                    & r530251  & L1D cache lines allocated in the M state                   \\ \cline{2-3} 
                                    & r530851  & L1D snoop eviction of cache lines in M state               \\ \hline
\multirow{15}{*}{Pasos entre Cache} & r500108  & Number of LLC read hits                                    \\ \cline{2-3} 
                                    & r500208  & Number of LLC write hits                                   \\ \cline{2-3} 
                                    & r500109  & Number of LLC read misses                                  \\ \cline{2-3} 
                                    & r500209  & Number of LLC write misses                                 \\ \cline{2-3} 
                                    & r50010a  & LLC lines allocated in M state                             \\ \cline{2-3} 
                                    & r50020a  & LLC lines allocated in E state                             \\ \cline{2-3} 
                                    & r50040a  & LLC lines allocated in S state                             \\ \cline{2-3} 
                                    & r50080a  & LLC lines allocated in F state                             \\ \cline{2-3} 
                                    & r500f0a  & LLC lines allocated                                        \\ \cline{2-3} 
                                    & r50010b  & LLC lines victimized in M state                            \\ \cline{2-3} 
                                    & r50020b  & LLC lines victimized in E state                            \\ \cline{2-3} 
                                    & r50040b  & LLC lines victimized in S state                            \\ \cline{2-3} 
                                    & r50080b  & LLC lines victimized in I state                            \\ \cline{2-3} 
                                    & r50100b  & LLC lines victimized in F state                            \\ \cline{2-3} 
                                    & r501f0b  & LLC lines victimized                                       \\ \hline
\multirow{4}{*}{MESI}               & r501f0b  & L2 data demand loads in E state                            \\ \cline{2-3} 
                                    & r530126  & L2 data demand loads in I state (misses)                   \\ \cline{2-3} 
                                    & r530326  & L2 data demand loads in M state                            \\ \cline{2-3} 
                                    & r530526  & L2 data demand loads in S state                            \\ \hline
\end{tabular}
\caption{Colección de eventos resumidos para la inspección de los canales de comunicación del sistema en escenarios multithread.}
\label{table:eventos}
\end{table}


\subsubsection{Resultados}

\subsubsection{Correlación de Eventos}
Para poder comprender mejor la tendencia de comportamiento en el experimento entre los distintos eventos capturados se repasaron posibles mecanismos de visualziación que permitieran una simple comparación entre dichas mediciones. Finalmente se optó por emplear una visualziación mediante el uso de una matriz de correlación, de manera de poder detectar facilmente conjuntos de eventos relacionados, ello combinando algúna estratégia de clusterización en el proceso de visualizar los datos. Ésta técnica es muy práctica y se ha empleado en otros escenarios sobre el mismo kernel en otras investigaciones con buenos resultados \cite{paper:clusteringKernel}.

\subsection{Análisis y Discusión de Resultados}

\section{Estudio de Distribución de Carga}
La tercera hipótesis de investigación plantea como responsable del mal rendimiento presentado en el caso de estudio al sistema de gestión y administración de tareas en el sistema operativo. En escenarios multicore como el estudiado, es normal que el sistema operativo realice como procedimiento de rutina la migración de procesos y la re-alocación de recursos y datos del sistema. Un caso práctico de ello es cuando un núcleo de procesamiento está sobreexigido y el sistema operativo redistribuye los hilos en dicho núcleo entre los procesadores disponibles del sistema, como una estratégia de balanceo de carga \cite{paper:NUMA}.

Ésta tercera hipótesis plantea que dicho proceso de reasignación de recursos sería perjudicial en escenarios de concurrencia basandose en que, mientras un proceso está en plena ejecución al incorporar más y más tareas en el mismo nucleo de procesamiento agregando hilos de ejecución, sería el sistema operativo quien comienzaría la reasignación automática de dichos hilos entre los distintos procesadores, cayendo en problemas como perdida de referencias de memoria en niveles de chacé primario de procesos en cuestión. En su peor escenario, ésta teoría lleva al conocido problema de \emph{caché bouncing} \cite{paper:cachebouncing} que corresponde al fenómeno de sobre corrección de los datos a nivel de lineas de cache de un procesador, producido por constantes cambios de contexto del \emph{scheduller} que genera migración de procesos. Un problema que ya se mencionó en el estudio de la sección previa.

Una alternativa que se ha estudiado para solventar éste problema es la técnica de \emph{Processor Affinity} que consiste en la asociación de tareas o procesos en procesadores especificos, de manera de controlar la ubicación de memoria y zona de ejecución del código. En otras palabras, se remueve la utilidad del mismo scheduler del sistema oeprativo para coordinar la mejor operación en la asignación de las mismas a las distintas CPU disponible, reemplazándolo por un criterio propio y estricto. Es una técnica muy interesante en el sentido de que, bien empleada puede proveer muy buenos resultados \cite{paper:cacheaffinity}, sin embargo también es facil degradar fuertemente el sistema al hacer mal uso de la misma.

\subsection{Processor Affinity}
\subsection{Esquemas de Distribución}
\subsection{Resultados Experimentales}
\subsection{Análisis y Discusión de Resultados}
\chapter{Estudio de Operación Primitivas de Sincronización del Sistema}

La primera hipótesis a estudiar plantea que el bajo rendimiento de la operación de la interfaz de red --Ilustrada en nuestro caso de estudio por medio de los Internet sockets UDP de Linux-- en escenarios de concurrencia, es causado por un mal desempeño de las estructuras que proveen los mecanismos de sincronización para dichos escenarios. Cómo se mencionó en el capítulo anterior, la capacidad multiprocesador de las computadoras modernas provee un mayor poder de cómputo que se postula a ser aprovechado por medio del uso de técnicas de programación paralela, con el cuidado de que en esos contextos de trabajo, los sistemas operativos deben estar preparados para atender situaciones de conflicto en el acceso a los recursos compartidos. Para este propósito, los sistemas operativos proveen de mecanismos de sincronización ya repasados en secciones anteriores que para estructuras de bajo nivel --como los sockets de Internet-- emplean el uso de mecanismos de sincronización de bajo nivel como lo son los spinlocks, que protegen ciertas porciones críticas de la estructura, tal y como se repasó previamente.

En este caso, la primera hipótesis del problema describe que el responsable del mal rendimiento al incorporar concurrencia en las lecturas a un socket es generado por dichas estructuras de protección en el acceso simultáneo, situación que causaría el fenómeno denominado \emph{Contención de Recurso} sobre la estructura compartida, o sobre alguna de las componentes constitutivas de la misma.

\begin{defn}[ver \cite{paper:resourceContention}] \textbf{Contención de Recurso} corresponde a un estado de conflicto en el acceso a un recurso compartido entre distintas componentes de un sistema, producido por una situación de competencia en el acceso al mismo que puede degenerar en escenarios problemáticos como situaciones de bloqueo, conflictos por situaciones de carrera y degradación generalizada de performance, entre otros.
\end{defn}

Para ratificar el planteamiento anterior, se estudió la dinámica de llamadas de sistema presentes al momento de la ejecución experimental del caso de estudio postulado en el capítulo anterior, ello siguiendo otros modelos de recopilación de datos ya evaluados en otros trabajos exitosos en la misma línea \cite{slides:hpPerf} a modo de poder modelar la operación de las primitivas de sincronización a medida que se van agregando hilos de procesamiento en el consumo de una misma estructura socket compartida. De la misma forma, es interesante analizar cómo el socket compartido actúa como un potencial punto de contención, o como alguna de las estructuras internas de limitación en su acceso (como sería el spinlock del socket mismo) tienen responsabilidad en el rendimiento presentado.

\section{Estudio de Llamadas de sistema}

La operación de las primitivas de sincronización que actúan en los procesos de bajo nivel del sistema operativo tienen la característica de estar determinadas por el uso de llamadas a funciones del sistema, ello pues es el mismo sistema operativo (o mejor dicho el núcleo del sistema) aquel que provee una interfaz simple para la invocación de dichas operaciones. Como son llamadas a funciones, es posible cuantificar cuándo y cómo se realizan las mismas pudiendo modelar el proceso completo por medio de este mecanismo.

Como en nuestro caso de estudio interesa inspeccionar el comportamiento de los spinlocks, se debe contemplar la API con que trabaja el sistema para controlar estas estructuras. Existen distintas funciones que proveen variantes en el funcionamiento de los spinlocks para operar en condiciones especiales, y dichos escenarios son presentables a lo largo de la ejecución del caso de prueba del estudio pudiendo impactar el rendimiento final. El objetivo de este estudio concierne un análisis cuantitativo de la cantidad de llamadas a sistema que sean bloqueantes sobre estructuras spinlock y del tiempo que el sistema gasta en dichas condiciones.

En Linux los spinlocks se representan con estructuras \verb=spinlock_t= (incluidas en el archivo \verb= <linux/spinlock.h>=) que básicamente consisten en un campo de \emph{lock} con un valor 1 (si está libre) o 0 (si está ocupado). Existen diversas funciones de atención que aplican distintos tipos de bloqueo \cite{book:spinlocks}:

\begin{description}
\item[void spin\_lock\_init(spinlock\_t *lock);] Inicializa una estructura spinlock y setea su valor inicial de \emph{lock} en 1.
\item[void spin\_lock(spinlock\_t *lock);] Es el bloqueo básico del sistema para tomar el \emph{lock}. Consistente en la espera del \emph{lock} hasta que presente un valor igual 1 para luego setearlo en 0. Dicha espera se realiza con ciclos de \emph{busy-waiting} hasta que se brinde acceso. Es un bloqueo interrumpible por el sistema operativo, tanto por interrupciones de software como de hardware, dando paso a situaciones como que la CPU determine enviar el proceso responsable de la llamada a dormir por falta de recursos, memoria, etc.
\item[void spin\_lock\_irq(spinlock\_t *lock);] Bloqueo que deshabilita interrupciones del procesador local antes de adquirir el spinlock. Se debe cuidar de reactivar las interrupciones luego de liberado el \emph{lock}.
\item[void spin\_lock\_irqsave(spinlock\_t *lock, unsigned long flags);] Similar a la operación de \verb=spin_lock_irq=, pero con la diferencia de que almacena el estado de interrupción previo en el valor \verb=flags=, de manera de que puede restablecerlo fácilmente luego de liberar el \emph{lock}.
\item[void spin\_lock\_bh(spinlock\_t *lock)] Similar a \verb=spin_lock_irq= con la diferencia de que sólo deshabilita las interrupciones de software, manteniendo habilitadas las interrupciones por hardware del sistema.
\item[int spin\_trylock(spinlock\_t *lock);] Intenta obtener el lock en un único intento, evitando entrar en ciclos muertos de espera, y retornar un valor inmediatamente sobre éxito o fracaso en dicha labor. La función retorna un valor distinto de cero si en su primer intento adquiere el spinlock, o 0 sino. Además, se puede usar en todos los contextos de la función \verb=spin_lock= con el cuidado de administrar el contexto de interrupciones producidas durante la ejecución e intento de adquisición del lock.
%\item[bool mutex\_spin\_on\_owner(struct mutex *lock, struct task\_struct *owner)] Bloqueo que opera sobre una estructura de exclusión mutua (\emph{mutex}) que utiliza el enfoque de \emph{Read-Copy-Update} (RCU), en donde los lectores son no bloqueantes. Ésta estructura tiene una sobrecarga menor que las anteriores; Sin embargo, las actualizaciones son más costosas ya que las versiones anteriores de la estructura de datos se deben guardar con el fin de dar cabida a los lectores ya existentes que se sincronizan a través de las barreras del \emph{mutex}. Utilizando el enfoque de la RCU el bloqueo con esta estructura \emph{mutex} asegura que la operación \emph{Test-and-Set} se ejecute en la misma CPU del propietario del \emph{lock}, lo que reduce la cantidad de comunicación de memoria caché (y por consiguiente, el efecto de contención).
\end{description}

Asociadas a las anteriores llamadas de sistema están las variantes \verb=*unlock*= que permiten liberar el elemento de bloqueo (seteando el valor del \emph{lock} de regreso a 1) para recuperar así su disponibilidad para otros procesos.

Para poder rescatar información relacionada a dichas llamadas a sistema existen distintas herramientas de software, las cuales, implementadas por los mismos desarrolladores del kernel de Linux, son capaces de inspeccionar la operación de bajo nivel del sistema para registrar la dinámica de todas las llamadas que sean de interés. En particular, se pueden usar para analizar la dinámica de llamadas bloqueantes antes descritas.

\subsection{Perf}
Perf \cite{slides:perfTools} o también llamado \emph{Perf\_events\footnote{\url{https://perf.wiki.kernel.org/}}} es una herramienta de análisis de performance para entornos Linux. Corresponde a un subsistema del mismo kernel de Linux que provee un completo framework para el estudio de performance del sistema y de programas por medio de la captura de una amplia variedad de fuentes de datos. Perf es capaz de colectar datos por operatividad de software (contadores de software, \emph{tracepoints}, estadísticas de ejecución de funciones, paso a assembler, etc.) y también de colectar información de control de hardware (manejo de PMU, lectura de \emph{performance counters}, etc.) abarcando los principales registros de control que mantiene el sistema en su operación.

\begin{figure}[!h]
	\centering
	\includegraphics[scale=.45]{imagenes/perfArchitecture.png}
	\caption{Arquitectura de operación del framework provisto por \emph{Perf}.}
	\label{fig:perfFramework}
\end{figure}

Además de su gran capacidad para colectar datos, Perf es una herramienta de sencillo uso. Su funcionamiento se basa en la supervisión de un determinado proceso o tarea de la cual se construye un archivo con la información que se haya seleccionado a colectar \cite{article:perf}. Para ello, se pueden emplear las utilidades \verb=perf-record= y \verb=perf-stat= del framework las que trabajan supervisando un determinado proceso y proveyendo páginas de datos al espacio del kernel de Linux las cuales son escritas con información del sistema de dicho proceso, y son retornadas al espacio de usuario construyendo un informe a modo de output\footnote{La asignación del espacio de páginas a rellenar se hace por medio de la utilidad \verb=mmap= de Linux, que provee direcciones virtuales en un proceso para almacenar información.} (Ver figura \ref{fig:perfRecord}). Esta característica es muy práctica pues se pueden realizar operaciones de análisis más exhaustivos sobre los reportes generados en etapas posteriores a su obtención misma.

El potencial de esta herramienta la perfila como una utilidad indispensable para el estudio en cuestión. En primer lugar por su capacidad de análisis de ejecución de código que permite obtener información cuantificada de las llamadas a sistema y de la dinámica del árbol de llamados\footnote{\emph{Call graph}, correspondiente a un grafo dirigido que representa la relación de llamados entre las subrutinas constitutivas de un proceso principal.} que permite reconocer la naturaleza de las funciones involucradas en el caso de estudio. En segunda instancia Perf es una estupenda herramienta para la recolección de datos de hardware al aprovechar el uso de la \emph{Performance Monitoring Unit (PMU)} del hardware del sistema, una característica que será revisada en detalle en secciones posteriores.

\begin{figure}[!h]
	\centering
	\includegraphics[scale=.52]{imagenes/perfRecord.png}
	\caption{Esquema de captura de datos de un programa usando el comando \emph{perf-record}.}
	\label{fig:perfRecord}
\end{figure}

\subsubsection{FTrace}
FTrace\footnote{\url{http://elinux.org/Ftrace}} es otra poderosa herramienta para análisis de rendimiento de software disponible para sistemas Linux \cite{paper:FTraceSony}. Su funcionamiento opera de naturaleza muy íntima con respecto al kernel mismo pues su recolección de datos se basa en el rastreo de la ejecución de funciones de forma dinámica en el espacio de kernel, caracterizando los tiempos reales de permanencia de cada llamada. Característica que lo hace una estupenda utilidad para el estudio de llamadas al sistema pudiendo recuperar datos como el tiempo de ejecución y cantidad de ocurrencia de las mismas.

Para su uso, FTrace opera como un verdadero framework del sistema sobre el kernel, del cual se pueden usar distintos métodos de rastreo de llamadas. Una de las funciones más poderosas de FTrace es el resultado que se puede obtener por medio de la instrumentación de código, que se refiere a la práctica de incorporar a los programas a analizar \emph{tracepoints} --o puntos de rastreo-- que son declaraciones explicitas de secciones de código a analizar y registrar. A pesar de que esta característica es muy cómoda para programas propios, en el caso del análisis de funciones y llamadas de sistema propias del kernel de Linux, la instrumentación de código es una característica dispensable, siendo sólo necesaria la precisión de qué llamadas a sistema considerar en el análisis pues FTrace es flexible para hacer análisis directamente de funciones del sistema. El uso de esta herramienta es muy flexible y configurable siendo activable a disposición del usuario, una característica muy importante pues al hacer una operación de rastreo de bajo nivel en el sistema, la actividad de FTrace significa una leve degradación en los tiempos netos de actividad de ciertas características del sistema.

En el contexto de la presente investigación, el provecho que se puede sacar de esta herramienta es usar su capacidad para cuantificar tiempo de funciones del kernel de Linux para estudiar la atomicidad de las llamadas bloqueantes del sistema. Así por ejemplo, se pretende determinar el tiempo que se pasa en estados bloqueantes de spinlocks (que terminan siendo pasos de \emph{busy-waiting}) en los cuales sólo se pierde tiempo por efectos de contención de recursos.

\section{Metodología de Experimentación}

Dado que la naturaleza de este estudio se relaciona con el comportamiento de funciones del sistema que administran las primitivas de acceso y sincronización, se realizarán las configuraciones pertinentes para cada herramienta a fin de contemplar dichos puntos de análisis. En el caso de Perf, la recolección de datos se realiza con la herramienta \verb=perf-record=, contemplando un post-procesamiento sobre los archivos de reporte generados a fin de colectar estadísticos asociados a las distintas funciones de manejo de spinlock antes mencionadas\footnote{Experimento \url{https://github.com/sebablasko/Test_MultiThreadStressTransmision} con privilegios de administrador.}.

Por otro lado, para el estudio con FTrace la configuración resulta un poco más compleja. Dado que es una herramienta de traceo dinámico que opera inspeccionando las llamadas de funciones de sistema, la activación de FTrace sobrecarga el funcionamiento del sistema general. Por ello, FTrace se debe activar y desactivar manualmente para analizar sólo los instantes de operación de la prueba de interés. Además, dado el amplio espectro de funciones disponibles para inspeccionar con el framework, se deben emplear las utilidades de filtrado de funciones a inspeccionar que provee el mismo framework, ello en pos de capturar funciones en línea con la dinámica del spinlock del Internet socket.

\vspace{1pc}
\begin{minipage}{\linewidth}
\begin{lstlisting}[style=BashInputStyle, label={code:ftrace}, caption={Configuración de filtros de FTrace sobre funciones a estudiar.}, captionpos=b]
[sebastian@labs-vhost ~]$ echo *spin* > /sys/kernel/debug/tracing/set_ftrace_filter 
[sebastian@labs-vhost ~]$ cat /sys/kernel/debug/tracing/set_ftrace_filter 
mutex_spin_on_owner
spin_msec
_spin_trylock
_spin_lock_irqsave
_spin_lock_irq
_spin_lock
_spin_unlock_irqrestore
_spin_lock_bh
_spin_trylock_bh
_spin_unlock_bh
bit_spin_lock
kvm_vcpu_on_spin
\end{lstlisting}
\end{minipage}

Por otra parte, el encendido y apagado del framework se configuró como parte del script de experimentación en el programa de la prueba\footnote{\url{https://github.com/sebablasko/Test_UDPTrace/}}. En el mismo, se configuró la opción \verb=set_ftrace_pid= para explicitar la inspección de FTrace sólo sobre el programa de la prueba, además del uso de la utilidad \verb=trace_marker= para instrumentar porciones de código de la prueba (como creación de threads, y término de consumo de datos) que permitiese una mayor facilidad al momento de estudiar los logs de ejecución recuperados.


\section{Resultados}
A continuación se ilustran los resultados experimentales obtenidos correspondientes a los datos colectados por Perf y FTrace. A modo de validación estadística, los resultados de las pruebas contemplan el promedio de un rango de 60 repeticiones en cada configuración dada por el caso de estudio.

\begin{figure}[!h]
	\centering
	\includegraphics[scale=.6]{resultados/perfdetalle-crop.pdf}
	\caption{Resultados experimentales de los porcentajes de ejecución de las llamadas a sistema recolectadas por \emph{Perf}.}
	\label{fig:resPerf}
\end{figure}

\begin{figure}[!h]
	\centering
	\includegraphics[scale=.6]{resultados/detalleFtrace-crop.pdf}
	\caption{Resultados experimentales de los tiempos de ejecución de las llamadas a sistema recolectadas por \emph{FTrace} para la adquisición y liberación del \emph{lock}.}
	\label{fig:detalleftrace}
\end{figure}


\section{Análisis y Discusión de Resultados}
A primera vista, los resultados corroboran una tendencia creciente de las operaciones (tanto en porcentaje como en tiempo neto de ejecución) de las operaciones de bloqueo sobre spinlock en las pruebas desarrolladas.

En el caso de los resultados de la prueba de Perf disponibles en la figura \ref{fig:resPerf}, se puede apreciar un comportamiento dominante en el porcentaje de llamado de funciones de una de las funciones por sobre las demás: \textbf{\_spin\_lock\_bh}, llamada que en condiciones de concurrencia puede aumentar su porcentaje de presencia en la ejecución desde un 22\% a casi un 35\%. Otro punto interesante es que, a medida que el escenario se vuelve más competitivo para los threads, los porcentajes rescatados por esta prueba para llamadas de bloqueo aumentan hasta aproximadamente los 4 threads, desde donde se estabiliza, manteniéndose sobre el 40\% sólo en éste apartado.

\begin{figure}[!h]
	\centering
	\includegraphics[scale=.8]{resultados/sumaperf-crop.pdf}
	\caption{Gráfico de suma de porcentajes de llamadas de bloqueo del \emph{lock} de spinlock.}
	\label{fig:sumaperf}
\end{figure}

Para reconocer mejor la dinámica de consumo de tiempos en las funciones inspeccionadas en el caso de estudio, se usaron los reportes generados por Perf para construir un nuevo tipo de visualización de llamadas a sistema: Un \emph{Call-Graph-Chart} (Ver figura \ref{fig:callgraph}), de manera de poder reconocer los bloques de funciones más repetidos en la ejecución del caso de prueba. Para esta visualización se aprovechó el script \emph{gprof2dot}\footnote{\url{https://github.com/jrfonseca/gprof2dot}}. En éste caso, resulta evidente cómo el grafo de llamadas de sistema se complejiza drásticamente al incorporar más threads, de la mano con el aumento en los porcentajes de permanencia en llamadas de sincronización.

\begin{figure}[]
	\centering
	\hspace*{\fill}
	\subfigure[Evaluando 2 Threads]{
		\includegraphics[width=.25\textwidth]{resultados/g2prof_2con.png}
		\label{fig:callgraph2}
	}\hfill
	\subfigure[Evaluando 4 Threads]{
		\includegraphics[width=.45\textwidth]{resultados/g2prof_4con.png}
		\label{fig:callgraph4}
	}
	\caption{Visualización de call-graph identificando las llamadas a sistema y sus pesos en el caso de prueba.}
	\label{fig:callgraph}
	\hspace*{\fill}
\end{figure}

Lo anterior postula una primera relación entre el número de threads en el acceso concurrente al socket con respecto al porcentaje y dinámica de llamados a funciones de bloqueo del spinlock. Sin embargo, los porcentajes de llamadas no son del todo relevantes si no se saben los tiempos efectivos que significan en la prueba. Para ello, se repasan los resultados de Ftrace.

Los resultados generales obtenidos con el estudio de FTrace disponibles en la figura \ref{fig:detalleftrace} dan cuenta de un fenómeno aún más interesante. Al igual que con los datos colectados con Perf, se ilustra una estrecha relación entre los tiempos de operación de las funciones relacionadas a spinlock a medida que se van incorporando threads, sin embargo, en éste caso, la tendencia resulta siempre creciente, plasmando cómo a medida que se van usando más threads, el sistema operativo pasa más tiempo en tareas de coordinación en su acceso.

Por otro lado, al realizar una colección de datos con FTrace a modo de recuperar el total de tiempo que se pasa en operaciones de tipo de bloqueo de spinlock se obtiene el resultado ilustrado en la figura \ref{fig:sumaFtrace}. Una característica interesante de este resultado es que, la tendencia de tiempos producida tiene un ajuste de naturaleza logarítmica, con un índice de determinación superior al 96\%. Este resultado es muy significativo, pues revela que los tiempos reales de acción de las funciones de bloqueo de spinlock siguen una tendencia de la misma naturaleza a los tiempos netos de operación del caso de estudio, estipulando una relación estrecha entre ambos y postulando como principal responsable de los tiempos finales a las funciones de coordinación en el acceso al spinlock.

\begin{figure}[!h]
	\centering
	\includegraphics[scale=.6]{resultados/sumaFtrace-crop.pdf}
	\caption{Tiempos totales de bloqueo sobre el \emph{lock} por las distintas llamadas de sistema capturadas por \emph{Ftrace} en el caso de estudio juntos con una curva de aproximación de tendencia.}
	\label{fig:sumaFtrace}
\end{figure}
Más aún, si se repasan las tendencias de cada una de las principales llamadas de bloqueo y de liberación de spinlock (ver figura \ref{fig:Ftracebloquealibera}) se puede ver como las tendencias de naturaleza logarítmica prevalecen al aumentar el número de threads en el consumo de datos. Sin perjuicio de lo anterior, otro dato interesante es que las operaciones de liberación (Fig. \ref{fig:ftracelibera}) son muchísimo más cortas que las de bloqueo (Fig. \ref{fig:ftracebloquea}) revelando el escenario de competencia al que se ve enfrentado el spinlock ante la concurrencia en el acceso.

\begin{figure}[h!]
	\centering
	\hspace*{\fill}
	\subfigure[Funciones de Bloqueo]{
		\includegraphics[width=.47\textwidth]{resultados/bloqueantesftrace-crop.pdf}
		\label{fig:ftracebloquea}
	}\hfill
	\subfigure[Funciones de Liberación]{
		\includegraphics[width=.47\textwidth]{resultados/liberadorasFtrace-crop.pdf}
		\label{fig:ftracelibera}
	}
	\caption{Gráficos con tendencias de tiempos del lock capturados con \emph{Ftrace} a lo largo del caso de estudio.}
	\label{fig:Ftracebloquealibera}
	\hspace*{\fill}
\end{figure}

\subsection{TraceDisplay}
Para poder obtener una interpretación adicional del fenómeno reconocido, se construyó una herramienta de visualización de las llamadas a sistema para funciones de sincronización que permitiese reconocer las porciones de tiempo que tomasen en cada procesador dichas funciones. Para ello, la herramienta denominada \emph{TraceDisplay} recibe un log generado con \emph{FTrace} que incluya las llamadas de sistema ya filtradas, y construye un mapa de tiempo coloreado donde se pueden apreciar las porciones de tiempo que consume cada llamada y desde cual CPU se originan. El resultado se puede apreciar en la figura \ref{fig:traceDisplay} donde se ilustra el caso de analizar los logs generados por FTrace en un escenario de acceso de 4 threads en el caso de estudio.

Éste subproducto de la investigación principal junto con su documentación de uso está publicado\footnote{Disponible en \url{https://github.com/sebablasko/TraceDisplay}} y disponible para su uso.

\begin{figure}[!h]
	\centering
	\includegraphics[scale=0.34]{imagenes/traceVisualization.png}
	\caption{Visualización de aplicación de llamadas de sistema de sincronización realizadas entre procesadores, generada con la herramienta TraceDisplay.}
	\label{fig:traceDisplay}
\end{figure}

Resultados como el mostrado en la figura \ref{fig:traceDisplay} ilustran como en la práctica, las operaciones de bloqueo se terminan ejecutando secuencialmente, aun cuando son distintos los CPU que los originen. Esto producto de que es una misma estructura la que se está compartiendo la cual no tiene un soporte adecuado para permitir modificaciones concurrentes de orígenes distintos, generando así un sobrecosto producido por dicha serialización en el acceso.


\section{Conclusiones}
A raíz del estudio de operación de primitivas de sincronización del sistema se pueden rescatar varios aspectos interesantes:
\begin{itemize}
\item La tendencia en los costos de tiempo son crecientes a medida que se agregan hilos que consumen el mismo socket. Ello tanto en porcentaje de llamadas de sistema, como también en tiempos netos en dichas llamadas.
\item Se destacan estructuras de tipo spinlock como las más reiteradas en el rastro de llamadas a sistema para la aplicación de mecanismos de protección en el caso de estudio. En particular, la llamada a sistema \textbf{\_spin\_lock\_bh} es la más significativa en la operación de los mecanismos de acceso y toma del \emph{lock} que protege al Internet socket.
\item Se reconoce una significativa diferencia entre los tiempos y porcentajes de llamadas a sistema de tipo bloqueante por sobre los de tipo liberador del \emph{lock}, escenario que da cuenta de la situación de competencia que empeora a medida que se incluyen más threads en la prueba.
\item Se destaca una tendencia de naturaleza logarítmica en el crecimiento de tiempos que toman las llamadas bloqueantes sobre el spinlock del socket, ajustada con un coeficiente de determinación superior al 96\% y que sigue la misma tendencia de los tiempos netos reconocidos en el caso de estudio.
\end{itemize}
A raíz de lo anterior, se reconoce en el spinlock de protección del socket como un punto de cuello de botella al momento de emplear accesos concurrentes a una estructura socket. Ello al actuar como un punto de bloqueo que termina serializando el acceso al consumo de datos y que, lejos de reducir los tiempos paralelizando el acceso, los aumenta, producto de la serialización y de la responsabilidad de coordinar los hilos, propios de este escenario.
\chapter{Estudio de Canales de Comunicación de Hardware}
La segunda hipótesis para explicar la mala performance del caso de estudio presentado se centra en una arista relacionada hardware más que al software mismo. Cómo ya se mencionó, la capacidad multiprocesador de que se dispone en equipos modernos no es un recurso fácil de aprovechar, de hecho, se requiere de una sofisticada operación y diseño tanto de las aplicaciones que solicitan recursos como del sistema operativo que ha de administrarlos, para conseguir las anheladas mejoras de performance. Como se repasó en secciones anteriores, la capacidad de paralelismo viene dada gracias a un conjunto de protocolos y algoritmos de muy bajo nivel que coordinan y mantienen en estado coherente las distintas componentes de datos para los diferentes procesadores disponibles en un esquema \emph{SMP} \cite{paper:MESI, paper:snoop}, sin embargo, por muy sofisticados que dichos mecanismos sean, las nuevas tecnologías de hardware que prometen velocidades de trasferencia y acceso nunca antes imaginadas podrían significar un problema para dichas componentes, sobrepasándolas de cierta forma.

Es precisamente en ésta línea que se establece la segunda hipótesis. En éste caso, se adjudican las responsabilidades por el mal rendimiento presentado en el caso de estudio a un problema de contención de recursos (nuevamente relacionado al spinlock de los Internet sockets), pero ésta vez, asociado a la persistencia y disponibilidad que se da del mismo a través de los mecanismos de coordinación antes mencionados. En las arquitecturas modernas los protocolos \emph{MESI} y de \emph{SNOOP} son cruciales en la operación de ejecuciones paralelas para garantizar integridad en los datos, pero las arquitecturas modernas proponen nuevas distribuciones de los componentes internos de hardware, brindando canales de comunicación de mayor velocidad de transmisión y reasignando los recursos físicos en modos diferentes a los usados para la concepción de los mecanismos de control ya mencionados. Ésta hipótesis plantea la posibilidad de que el defecto de performance del caso de estudio sea generado por un fenómeno de \emph{Caché Bouncing}, producto del abusivo comportamiento de dichos mecanismos de control en las arquitecturas modernas.

\begin{defn}[ver \cite{paper:cachebouncing}] \textbf{Caché Bouncing} corresponde a un fenómeno producido en entornos multiprocesador, cuando distintas CPU realizan modificaciones a una línea de caché especifica que está siendo referenciada por varios procesadores. La modificación de la línea en cuestión se propaga de caché en caché según los protocolos de consistencia del sistema, pero cuando la cardinalidad de referencias de los distintos procesadores sobre la misma línea de caché es muy alta, los mecanismos de propagación de caché pueden operar deficientemente. Éste fenómeno impone una significativa carga en el bus de memoria y los distintos canales de comunicación afectados pudiendo degenerar desde una degradación del proceso responsable hasta una degradación generalizada de la performance del sistema.
\end{defn}

En ésta línea, el fenómeno de \emph{Cache Bouncing} se podría manifestar dada la arquitectura del sistema, la que al contemplar bancos de memoria diversos --algunos compartidos y otros exclusivos para los núcleos de procesamiento-- podría estar manifestándose como resultado de las modificaciones concurrentes de los distintos procesadores sobre la estructura socket compartida, y más precisamente sobre el spinlock de protección del socket. Lo anterior combinado a la operación de los protocolos de consistencia y correctitud para las líneas de caché del sistema postulan evidencia que hace perfectamente posible el diagnóstico de que se esté generando un escenario de sobrecarga de comunicación que termine degradando los tiempos totales de ejecución del caso de estudio.

Para validar la hipótesis anterior es preciso un cabal entendimiento de la arquitectura de hardware objetivo a fin de poder localizar puntos de contención y canales afectados. Junto con lo anterior, se hace crucial una comprensión significativa del funcionamiento de la \emph{Performance Monitoring Unit} que provee el fabricante, así como lograr configurarla y aprovecharla para la recolección de datos finales. En las siguientes secciones se realiza un estudio de la arquitectura descrita en la figura \ref{fig:pc3} del equipo sobre el cual se realizan las pruebas experimentales reales para poder acotar el dominio de estudio. Posteriormente se realiza un análisis experimental de las tendencias presentes en una tarea de acceso concurrente como la descrita en el caso de estudio de ésta investigación con el fin de corroborar o descartar las sospechas ya mencionadas del efecto de contención y \emph{caché bouncing} por eventos de performance de hardware.

\section{Características de Arquitecturas de Hardware Modernas}
Cómo ya se mencionó en secciones anteriores, los fabricantes de partes y piezas de computadoras están constantemente desarrollando importantes avances, de la mano con el desarrollo técnico de piezas que brinda mejores componentes de hardware cada día, y la línea de desarrollo de infraestructura de hardware principal de los computadores no está exenta de dicha evolución. En secciones anteriores se presentó como las arquitecturas han evolucionado desde el primer esquema \emph{SMP} propuesto con la distribución \emph{FSB}, pasando luego por nuevas configuraciones como \emph{DIB} y \emph{DHSI}, entre otras. Sin embargo, el desarrollo ha sido constante y hoy las arquitecturas han degenerado en esquemas bastante más complejos en pos de aprovechar al máximo la capacidad de los procesadores en la línea del paralelismo.

\subsection{Arquitectura Intel QuickPath}
El año 2008, el fabricante de procesadores Intel®\footnote{\url{http://www.intel.com/}} lanzó al mercado una nueva tecnología denominada \emph{Intel QuickPath Architecture} \cite{paper:quickpath} la que postuló un nuevo esquema organizacional de los componentes internos de la placa principal de las computadoras, así como también un nuevo esquema de conectividad entre los componentes de la misma, prometiendo entre otras cosas, un sistema más confiable, eficiente, rápido y escalable, que podría aprovechar mejor la capacidad de los múltiples procesadores de su misma línea. Rápidamente \emph{QuickPath} se posicionó en el mercado para competir con la tecnología \emph{HyperTransport} desarrollada por \emph{AMD}\footnote{Principal competidor de Intel en la industria de la manufactura de microprocesadores.}, abriendo paso definitivamente a una era, dejando atrás el enfoque \emph{FSB}.

\begin{figure}[!h]
	\centering
	\includegraphics[scale=.5]{imagenes/quickpath2.png}
	\caption{Diseño organizacional de los componentes de sistema en una arquitectura estándar \emph{QuickPath} de Intel.}
	\label{fig:quickpath}
\end{figure}

El esquema \emph{QuickPath} postula una reformación arquitectural de los componentes principales de un sistema \ref{fig:quickpath}. En ésta propuesta, las distintas unidades de procesamiento (CPU) están interconectadas por canales de comunicación especiales denominados \emph{Intel QuickPath Interconnect - QPI} que son conexiones punto a punto entre CPU de enorme velocidad de transferencia (llegando hasta 25 GB/s en canales con variantes unidireccional o bidireccional). En línea para aprovechar esta nueva arquitectura, Intel® propone una serie de modificaciones al tradicional protocolo MESI, construyendo el denóminado protocolo \textbf{MESIF} que es una evolución del primero, flexibilizando los protocolos de coherencia de caché al incorporar un nuevo estado para las lineas de caché (\textbf{F} - \emph{Forward}) dotando así de mayor eficiencia y velocidad de acción entre procesadores al ser una comunicación directa.

Por otro lado, en la arquitectura \emph{QuickPath} cada CPU dispone de su propio controlador de memoria y de un banco de memoria de acceso próximo. Dicho diseño se denomina un nodo \textbf{NUMA} de sus siglas en inglés \emph{\textbf{N}on \textbf{U}niform \textbf{M}emory \textbf{A}ccess} [CITA A NUMA] la que permite a las CPU de cada nodo NUMA disponer de un banco de memoria con un acceso garantizado más rápido que al que se tendría acceso en una arquitectura tradicional. El enfoque \emph{NUMA} se aprovecha del principio de localidad de memoria \cite{paper:memorylocality}, por la cual postula que los datos son separables en su acceso por las distintas CPU, logrando así mayor velocidad en el acceso a la memoria, y menor problemas de coherencia de la misma por modificaciones entre CPUs.

El trabajo de Intel® va más allá. Conscientes de la necesidad de herramientas y utilidades para analizar la verdadera performance que provee ésta arquitectura, Intel® provee unidades de monitoreo de performance embebidas en sus procesadores (o \textbf{PMU}, por sus siglas en ingles \emph{\textbf{P}erformance} \textbf{M}onitoring \textbf{U}nit) que son componentes de hardware incorporado a los sistemas que permiten la inspección de operaciones a nivel de comunicación entre componentes del sistema directamente. A éste tipo de análisis se denomina \emph{estudios de Perfomance Counters} dado que para poder realizar una medición, el fabricante de la PMU provee una colección de posibles eventos a colectar, con significaciones puntuales cada uno.

\begin{defn}[ver \cite{article:performanceCounters}] \textbf{Performance Counters} son identificadores de máquina que permiten cuantificar determinados eventos a nivel de hardware, como lecturas de caché, corrección de líneas de caché, comunicación de protocolos de coherencia, etc. Usados para analizar el comportamiento de ciertas unidades de hardware y que conforman la base de las herramientas de profiling moderno para el rastreo en el comportamiento de funciones de un sistema.
\end{defn}

La Intel® \emph{QuickPath Architecture} implementa un modelo de 5 capas (Ver fig. \ref{fig:5layersqpi}) para la comunicación de datos entre los núcleos de procesamiento (Similar al espíritu del modelo OSI). De dicho modelo, para la detección de comunicación de aplicaciones son muy significativos los niveles de: \textbf{Protocolo}, al asociar tareas del paso de paquetes, aplicación del protocolo MESIF y de los mecanismos de SNOOP para control y coherencia de líneas de caché, y \textbf{Link}, para el reconocimiento de mecanismos de corrección de errores y recuperación a lo largo de la transmisión de datos entre dispositivos, y para el llamado esquema de \emph{Crédito/Débito} desarrollado por el mismo fabricante que permite trasmisiones de datos confiables entre componentes.

\begin{figure}[!h]
	\centering
	\includegraphics[scale=.5]{imagenes/5layersqpi.png}
	\caption{Diagrama de las 5 capas implementadas en la Intel® \emph{QuickPath Architecture} ilustrando los niveles de control de distintas componentes y protocolos importantes del sistema.}
	\label{fig:5layersqpi}
\end{figure}

El estudio de \emph{performance counters} corresponde a uno de los análisis de más bajo nivel realizables en pos de obtener datos que representen la forma fiel la comunicación entre componentes del sistema, cuyos resultados permiten tener pleno conocimiento de la dinámica de operación entre componentes del sistema \cite{paper:aplicacionDeHPCTOOLKIT}. Ello lo hace también un estudio dificultoso de realizar pues amerita un gran conocimiento de la arquitectura puntual sobre el sistema que se desea estudiar, pero de enorme valor para valerse de información sobre la verdadera cuota de comunicación inherente a un caso de estudio.

\section{Especificación y Captura de Eventos}
Para definir el marco conceptual de la prueba, se debe mantener presente el contexto de la hipótesis que fundamenta la misma. En éste caso, la motivación de éste estudio está en línea con entender el comportamiento de un consumo concurrente en una estructura socket, o más precisamente, ver cómo una instancia de una primitiva de sincronización --un spinlock-- se comporta a nivel de actividad de hardware en un escenario multithread. En ese escenario, se busca estudiar cómo se manifestaría la expresión de los protocolos de coherencia de cache bajo circunstancias de ejecución en un escenario mutiprocesador que podrían dar cuenta de que el mal desempeño general del caso de estudio tiene sus origines en los mismos sistemas de coordinación de bajo nivel del sistema.

\subsection{Arquitectura de la máquina para las pruebas}
Siguiendo el caso de prueba evaluado a lo largo de la investigación, se inspeccionará el caso de estudio de saturación de un socket UDP en el equipo servidor multicore para pruebas (Ver fig. \ref{fig:hwspecs}). En éste caso, se cuenta con un equipo placa Dell Inc. 00NH4P A07, provisto de dos CPU Intel Xeon 5600 2.8Ghz, dotado de 6 cores cada uno. Cada CPU dispone de hasta tres niveles de cache de 192 kb, 1536kb y 12288kb respectivamente, que combinados con tres memorias de 4096MB (DDR3 1333MHz) cada uno, conforman 2 nodos NUMA, con un monto total de 24GB de memoria. Una configuración que es precisamente de la familia \emph{Intel QuickPath Architecture} y es enfocada a servidores multicpu \cite{report:intelxeon5600, manual:intelxeon5600}.

\begin{figure}[!h]
	\centering
	\includegraphics[scale=.75]{imagenes/arch24Cores.png}
	\caption{Esquema de arquitectura interna del esquipo servidor multicore sobre el que se realizan las pruebas de \emph{performance counters}.}
	\label{fig:hwspecs}
\end{figure}

Para inspeccionar el sistema donde se evalúan las pruebas en detalle en busca de componentes de monitoreo disponibles se utilizó la utilidad \emph{libpfm4}\footnote{\url{http://www.bnikolic.co.uk/blog/hpc-prof-events.html}} que corresponde a una herramienta confeccionada para recuperar información sobre los códigos de inspección de eventos de performance counters de un sistema y de las PMU disponibles en el mismo. De ello, se da cuenta de que para el modelo de procesador presente se dispone de 5 unidades de PMU disponibles, separables en 3 grupos (Ver código \ref{code:pmuavailable}):
\begin{description}
\item[PMU Genéricas] Incluyen a \verb=perf= y  \verb=perf_raw=. Disponen de las especificaciones estándar de \emph{performance counters} de la línea del software \emph{Perf}, lo cual las hace poco exactas en los valores descritos por cada evento y no necesariamente fieles a su especificación pues dependen en gran parte de que el fabricante sea riguroso en su implementación.
\item[PMU x86] PMU generacional de Intel para la línea x86 de Intel que incluye a \verb=ix86arch=. Dispone de eventos comunes a dicha línea de procesadores por lo que no da soporte específico para la arquitectura \emph{QuickPath} ni del funcionamiento multiprocesador.
\item[PMU westmere] PMU específicas de la línea \emph{Westmere}\footnote{\url{https://en.wikipedia.org/wiki/Westmere_(microarchitecture)}} que soporta la base del desarrollo de la \emph{Intel QuickPath Architecture}, incluyendo a \verb=wsm_dp= y \verb=wsm_unc=. Es el nivel más exacto de PMU que provee el fabricante con los eventos más especificos y documentados del sistema, por lo que es la PMU más importante a la hora de colectar eventos.
\end{description}

\vspace{1pc}
\begin{lstlisting}[style=BashInputStyle, label={code:pmuavailable}, caption={Listado de \emph{PMUs} disponibles en el sistema, recuperado con la herramienta \emph{libpfm4}.}, captionpos=b]
	Detected PMU models:
[18, ix86arch, "Intel X86 architectural PMU", 6 events, 1 max encoding, 7 counters, core PMU]
[51, perf, "perf_events generic PMU", 104 events, 1 max encoding, 0 counters, OS generic PMU]
[53, wsm_dp, "Intel Westmere DP", 91 events, 2 max encoding, 7 counters, core PMU]
[54, wsm_unc, "Intel Westmere uncore", 52 events, 1 max encoding, 9 counters, uncore PMU]
[114, perf_raw, "perf_events raw PMU", 1 events, 1 max encoding, 0 counters, OS generic PMU]
\end{lstlisting}

\subsection{Metodología de captura de eventos}
Para la especificación de la captura de eventos resulta imprescindible prestar especial atención a la arquitectura interna que soporta la comunicación interprocesador del sistema. En la figura \ref{fig:hwcomm} se da cuenta de los canales de comunicación que dispone el sistema estudiado, haciendo un acercamiento a cada unidad lógica de procesamiento, núcleo de cada nodo NUMA.

\begin{figure}[!h]
	\centering
	\includegraphics[scale=.8]{imagenes/QuickPathChannels.png}
	\caption{Esquema de arquitectura interna del equipo servidor multicore estudiado, ilustrando las vías de comunicación del procesador que da cuenta de los principales puntos de alta comunicación en el escenario de \textit{cache bouncing} por contención de valores.}
	\label{fig:hwcomm}
\end{figure}

La figura \ref{fig:hwcomm} da cuenta de los distintos puntos en que se pueden suceder escenarios de congestión dados ya sea por un alto nivel de uso de los protocolos de coordinación de memoria o por un alto tráfico de comunicación entre CPUs. Dichos canales de comunicación podemos resumirlos en las siguientes funciones dedicadas de la arquitectura estudiada, que a su vez están asociadas a ciertos \emph{performance counters} del sistema:

\begin{itemize}
\item Uso de los canales \emph{Intel QuickPath QPI}
\item Acciones del protocolo \emph{SNOOP}
\item Pasos de datos entre distintos caché y memoria
\item Transiciones del protocolo \emph{MESI(F)}
\end{itemize}

Con ello en mente, se consultó el manual oficial del fabricante \cite{manual:bigbigevents} en búsqueda de documentación acerca de eventos disponible en el sistema que se relacionaran a las operaciones antes descritas. De dicha documentación, combinado con la utilidad \emph{libpfm4} para la recolección de eventos disponibles se colectaron un total de 143 eventos cada uno con hasta 6 variantes de configuración, dando en total casi 500 posibles eventos a estudiar. En este escenario es preciso acotar los eventos a considerar, de acuerdo a los 4 criterios antes descritos.


%[[[Hablar un poco de dicha busqueda y del manual]]]]

Finalmente, siguiendo los 4 criterios de puntos problemáticos a estudiar se construyó una selección de eventos a considerar resumida en la tabla \ref{table:eventos}. Los eventos están divididos en dos grupos: \textbf{QPI/GQ/Cache} para eventos relacionados con movimiento o traslación de datos entre distintas unidades de hardware, y \textbf{LinkLayer} para eventos referidos a protocolos de consistencia y sincronización que son pertinentes a la capa de corrección en el esquema de capas del \emph{Quickpath}.

\begin{table}[h!]
\centering
\begin{tabular}{l|l}
\multicolumn{1}{c|}{{\bf QPI/GQ/CACHE}} & \multicolumn{1}{c}{{\bf LinkLayer}} \\ \hline
{ UNC\_GQ\_DATA\_FROM} & SNOOPQ\_REQUESTS \\
{ UNC\_GQ\_DATA\_TO} & SNOOPQ\_REQUESTS\_OUTSTANDING \\
{ UNC\_QHL\_REQUESTS} & SNOOP\_RESPONSE \\
{ L1D} & UNC\_QPI\_RX\_NO\_PPT\_CREDIT \\
{ L2\_DATA\_RQSTS} & UNC\_QPI\_TX\_STALLED\_MULTI\_FLIT \\
{ UNC\_LLC\_HITS} & UNC\_QPI\_TX\_STALLED\_SINGLE\_FLIT \\
{ UNC\_LLC\_MISS} & UNC\_SNP\_RESP\_TO\_LOCAL\_HOME \\
{ UNC\_LLC\_LINES\_IN} & UNC\_SNP\_RESP\_TO\_REMOTE\_HOME \\
{ UNC\_LLC\_LINES\_OUT} & UNC\_IMC\_RETRY
\end{tabular}
\caption{Total de eventos inspeccionados y estudiados en el caso de estudio de consumo concurrente sobre sockets UDP.}
\label{table:eventos}
\end{table}

Con los eventos a colectar más claros, el siguiente paso consiste en conseguir los códigos de registro para la adquisición de cada evento. Un código de registro sigue una nomenclatura dada por el fabricante de la PMU y no guarda relación semántica con el valor que reporta, pero es la única referencia para indicar al software de recolección de datos el evento de interés a estudiar. Para ello, se confeccionó una herramienta\footnote{\url{https://github.com/sebablasko/libpfm4PerformanceEventParser}} para la obtención de los códigos de registros de cada evento de interés la cual trabajando en conjunto con \emph{libpfm4} es capaz de parsear datos del sistema para generar una colección de códigos de registro asociados a eventos de hardware en formato \verb=JSON=, según los cuales se tiene una correcta representación del nivel de saturación y uso de los componentes. Con el \verb=JSON= generado, se pueden hacer mediciones de forma sencilla usando la herramienta \verb=stat= de \emph{Perf} para generar un reporte de la cantidad de veces que se registre actividad en el evento estudiado, especificado en la misma herramienta.

\vspace{1pc}
\begin{lstlisting}[style=BashInputStyle, breaklines=true, captionpos=b, caption={Ejemplo de uso de Perf para colectar datos de una colección de eventos. En éste caso se configura para colectar datos de 2 eventos y dejar el reporte en un archivo de salida.}]
	# perf stat -e r53003c,r5300c0 -o resultado.txt -- ./programa
\end{lstlisting}

Finalmente, de los output de Perf se pueden estudiar los resultados finales de la comunicación efectiva generada a lo largo de la prueba.

\section{Metodología de Experimentación}
Para poder comprender mejor las tendencias de comportamiento de los distintos eventos en cada instancia de prueba con una determinada configuración de threads, se confeccionó un experimento donde se evaluaron 3 escenarios de consumo para comparar sus resultados\footnote{\url{https://github.com/sebablasko/Test_PerformanceCounters}}:

\begin{enumerate}
\item Lectura concurrente desde un dispositivo virtual como \verb=dev_null=. Para ilustrar el comportamiento en el caso de menor sobrecarga en lectura concurrente al ser un dispositivo libre de barreras de bloqueo.
\item Lectura exclusiva desde un socket UDP. En éste caso se consume la cuota definida en el caso de estudio por sockets con acceso exclusivo, distribuyendo la carga entre ellos con sólo 1 thread por socket. 
\item Lectura concurrente desde un socket UDP. Precisamente el caso de estudio evaluado a lo largo de toda la investigación.
\end{enumerate}

Así se puede tener un punto de comparación del fenómeno que se manifiesta en escenarios de lectura concurrente sobre un socket, y que no se manifiesta en otros escenarios.

Para la recolección de datos se empleó nuevamente el software \emph{Perf} para la administración y control de la PMU del sistema. Al igual que en el caso de estudio original se evalúan configuraciones de 1, 2, 4, 8, 16, 24, 36, 48, 64 y 128 threads contemplando un total de 60 repeticiones del proceso de captura para cada configuración de prueba. En la recolección de datos se define $T_i$ a la tupla que agrupa los resultados de cada una de las 60 repeticiones para una configuración de la prueba empleando $i$ hilos.

\begin{equation}
\label{eq:tupla1}
T_i = \left\{ T_{i,1},T_{i,2},T_{i,3}, \dots ,T_{i,59}, T_{i,60}\right\} 
\end{equation}

Con las muestras totales para cada configuración, se pueden determinar representantes estadísticos que nos permitan ilustrar el valor de cada configuración. Para éste propósito se emplea el calculo del promedio simple entre las muestras.

\begin{equation}
\label{eq:promedio}
\overline{T_{i}} = \frac{T_{i,1}+T_{i,2}+T_{i,3}+ \dots +T_{i,59}+ T_{i,60}}{60}
\end{equation}

Finalmente, se construye con para cada evento estudiado un set de registros que guardan el valor promedio de cada evaluación en una determinada configuración de hilos de ejecución.

\begin{equation}
\label{eq:tupla2}
Evento_j = \left(\overline{T_{1}}, \overline{T_{2}}, \overline{T_{4}}, \overline{T_{6}}, \overline{T_{8}}, \overline{T_{16}}, \overline{T_{24}}, \overline{T_{36}}, \overline{T_{48}}, \overline{T_{60}}\right)
\end{equation}

De ésta manera, se pueden hacer análisis más simples sobre las variables aleatorias $Evento_j$ que permita una simple comparación entre cada una de los 3 escenarios de evaluación estipulados.

\section{Resultados}
Dada la enorme cantidad de resultados obtenidos en el proceso de inspección de los \emph{performance counters}, se presentan sólo aquellos resultados con comportamiento interesante registrado en el caso de prueba. Los resultados se han agrupado en 8 categorías para facilitar la asociación de comportamientos y de elementos analizados: Comportamiento de caché de datos de nivel 1, caché de nivel 2, último nivel de caché y banco de memoria, fallo en predicción de procesamiento, solicitudes fuera del core, comunicación por canales de la arquitectura \emph{Quickpath} y finalmente actividad de coordinación por protocolos \emph{SNOOP} y \emph{MESIF}.

\begin{figure}[ph!]
\centering
\subfigure[]{
	\includegraphics[width=.47\textwidth]{resultados/pcounters/r530151.png}
	\label{fig:pcounterL1a}
}
\subfigure[]{
	\includegraphics[width=.47\textwidth]{resultados/pcounters/r530251.png}
	\label{fig:pcounterL1b}
}
\subfigure[]{
	\includegraphics[width=.47\textwidth]{resultados/pcounters/r530451.png}
	\label{fig:pcounterL1c}
}
\subfigure[]{
	\includegraphics[width=.47\textwidth]{resultados/pcounters/r530851.png}
	\label{fig:pcounterL1d}
}
\caption{Resultados asociados al comportamiento del caché de datos de primer nivel.}
\label{fig:pcounterL1}
\end{figure}

\begin{figure}[ph!]
\centering
\subfigure[]{
	\includegraphics[width=.47\textwidth]{resultados/pcounters/r530f28.png}
	\label{fig:pcounterMESIFa}
}
\subfigure[]{
	\includegraphics[width=.47\textwidth]{resultados/pcounters/r530128.png}
	\label{fig:pcounterMESIFb}
}
\caption{Resultados asociados al comportamiento del protocolo MESIF.}
\label{fig:pcounterMESIF}
\end{figure}

\begin{figure}[ph!]
\centering
\subfigure[]{
	\includegraphics[width=.47\textwidth]{resultados/pcounters/r537f89.png}
	\label{fig:pcounterMissBranchPredictiona}
}
\subfigure[]{
	\includegraphics[width=.47\textwidth]{resultados/pcounters/r5301c5.png}
	\label{fig:pcounterMissBranchPredictionb}
}
\caption{Resultados asociados al fenómeno de fallo en predicción de ejecución del procesador.}
\label{fig:pcounterMissBranchPrediction}
\end{figure}


\begin{figure}[ph!]
\centering
\subfigure[]{
	\includegraphics[width=.47\textwidth]{resultados/pcounters/r5301b0.png}
	\label{fig:pcounterOFFCorea}
}
\subfigure[]{
	\includegraphics[width=.47\textwidth]{resultados/pcounters/r5380b0.png}
	\label{fig:pcounterOFFCoreb}
}
\subfigure[]{
	\includegraphics[width=.47\textwidth]{resultados/pcounters/r530160.png}
	\label{fig:pcounterOFFCorec}
}
\caption{Resultados asociados al comportamiento de fallo en solicitud de datos de un core.}
\label{fig:pcounterOFFCore}
\end{figure}

\begin{figure}[ph!]
\centering
\subfigure[]{
	\includegraphics[width=.47\textwidth]{resultados/pcounters/r53aa24.png}
	\label{fig:pcounterL2a}
}
\subfigure[]{
	\includegraphics[width=.47\textwidth]{resultados/pcounters/r53e027.png}
	\label{fig:pcounterL2b}
}
\subfigure[]{
	\includegraphics[width=.47\textwidth]{resultados/pcounters/r530f26.png}
	\label{fig:pcounterL2c}
}
\subfigure[]{
	\includegraphics[width=.47\textwidth]{resultados/pcounters/r5302f1.png}
	\label{fig:pcounterL2d}
}
\subfigure[]{
	\includegraphics[width=.47\textwidth]{resultados/pcounters/r530127.png}
	\label{fig:pcounterL2e}
}
\caption{Resultados asociados al comportamiento del caché de segundo nivel.}
\label{fig:pcounterL2}
\end{figure}


\begin{figure}[ph!]
\centering
\subfigure[]{
	\includegraphics[width=.47\textwidth]{resultados/pcounters/r50010b.png}
	\label{fig:pcounterLLCa}
}
\subfigure[]{
	\includegraphics[width=.47\textwidth]{resultados/pcounters/r50012e.png}
	\label{fig:pcounterLLCb}
}
\subfigure[]{
	\includegraphics[width=.47\textwidth]{resultados/pcounters/r50022e.png}
	\label{fig:pcounterLLCc}
}
\subfigure[]{
	\includegraphics[width=.47\textwidth]{resultados/pcounters/r500109.png}
	\label{fig:pcounterLLCd}
}
\subfigure[]{
	\includegraphics[width=.47\textwidth]{resultados/pcounters/r500308.png}
	\label{fig:pcounterLLCe}
}
\subfigure[]{
	\includegraphics[width=.47\textwidth]{resultados/pcounters/r530205.png}
	\label{fig:pcounterLLCf}
}
\caption{Resultados asociados al comportamiento del último nivel de caché y memoria principal.}
\label{fig:pcounterLLC}
\end{figure}


\begin{figure}[ph!]
\centering
\subfigure[]{
	\includegraphics[width=.47\textwidth]{resultados/pcounters/r500104.png}
	\label{fig:pcounterQPIa}
}
\subfigure[]{
	\includegraphics[width=.47\textwidth]{resultados/pcounters/r500105.png}
	\label{fig:pcounterQPIb}
}
\subfigure[]{
	\includegraphics[width=.47\textwidth]{resultados/pcounters/r500204.png}
	\label{fig:pcounterQPIc}
}
\subfigure[]{
	\includegraphics[width=.47\textwidth]{resultados/pcounters/r500205.png}
	\label{fig:pcounterQPId}
}
\subfigure[]{
	\includegraphics[width=.47\textwidth]{resultados/pcounters/r500240.png}
	\label{fig:pcounterQPIe}
}
\caption{Resultados asociados al comportamiento los canales y estructuras de comunicación del Intel® \emph{QuickPath}.}
\label{fig:pcounterQPI}
\end{figure}


\begin{figure}[ph!]
\centering
\subfigure[]{
	\includegraphics[width=.47\textwidth]{resultados/pcounters/r5301b4.png}
	\label{fig:pcounterSNOOPa}
}
\subfigure[]{
	\includegraphics[width=.47\textwidth]{resultados/pcounters/r5302b4.png}
	\label{fig:pcounterSNOOPb}
}
\subfigure[]{
	\includegraphics[width=.47\textwidth]{resultados/pcounters/r500106.png}
	\label{fig:pcounterSNOOPc}
}
\subfigure[]{
	\includegraphics[width=.47\textwidth]{resultados/pcounters/r500107.png}
	\label{fig:pcounterSNOOPd}
}
\subfigure[]{
	\includegraphics[width=.47\textwidth]{resultados/pcounters/r500206.png}
	\label{fig:pcounterSNOOPe}
}
\caption{Resultados asociados al comportamiento del mecanísmo de control \emph{SNOOP}.}
\label{fig:pcounterSNOOP}
\end{figure}

\section{Análisis y Discusión de Resultados}

Los diferentes resultados presentados en la sección anterior comparten una serie de características comunes importantes de resaltar: En primer lugar, la experimentación sobre el dispositivo virtual \verb=dev_null= presentó prácticamente un nulo registro a nivel de los eventos contabilizados. Una situación esperable pues como se adelantó al momento de diseñar el experimento, éste dispositivo fue elegido como parte de la prueba por ser un dispositivo virtual libre de protecciones ante concurrencia, de manera que su registro da pistas de la correcta activación de las labores de registro de eventos.

Otro rasgo general interesante es que los comportamientos entre el caso \emph{UDPMultiThread} y \emph{UDPSingleThreadMultiSocket} presentan notables diferencia en prácticamente todos los eventos inspeccionados en el caso de estudio, lo cual, más allá de los distintos valores registrados, corrobora que existe una significativa diferencia entre los dos enfoques de consumo de datos (compartición del socket con múltiples hilos contra consumo exclusivo de socket).

En las secciones siguientes se analizan en detalle cada uno de los grupos de resultados rescatados en ésta prueba.

\subsection{Comportamiento de caché de datos de nivel 1}
Los resultados registrados con respecto al comportamiento del caché de datos de nivel 1 (Ver figura \ref{fig:pcounterL1}) evidencian un claro comportamiento desigual entre los escenarios de consumo concurrente vs. consumo exclusivo. Como se aprecia en la mayoría de los resultados de dicha prueba, las tendencias entre ambos escenarios son similares hasta la aplicación de dos hilos, pero al incorporar más hilos de ejecución rápidamente los registros de eventos contabilizados se disparan en el caso multi-hilo, evidenciando una mayor actividad a nivel de caché de datos de nivel 1.

Otro dato interesante a resaltar es el fenómeno de \emph{caché bouncing} expresado en éste resultado. Cómo se puede apreciar, los resultados de las figuras \ref{fig:pcounterL1b}) y \ref{fig:pcounterL1c}) que expresan la cantidad de líneas de caché en estado modificado (de acuerdo a la nomenclatura \emph{MESI}) aumentan progresivamente a medida que se incrementan los hilos consumiendo datos del socket compartido, apoyando la idea de que el escenario concurrente contribuye a una mayor sobre corrección de líneas de caché, lo que puede contribuir en una degradación de la performance del sistema. Lo anterior es nuevamente avalado por el resultado de la figura \ref{fig:pcounterL1d}), que ilustra la acción de los protocolos de corrección y coordinación para modificar estados de líneas de caché. Vale decir, éste resultado muestra cómo los mecanismos de coordinación del sistema están operando desde componentes ajenas a cada Core para corregir líneas que han sido modificadas en otro procesador, otro indicio de que la modificación concurrente de una estructura compartida termina impactando al sistema completo.

Una última reflexión de éste resultado da cuenta de cómo las tendencias crecientes registradas por los eventos relacionados al caché de datos de nivel 1 se ajustan a la registrada por los tiempos prácticos de ejecución del caso de estudio, dando espacio para entender una correlación entre ambos fenómenos.

\subsection{Comportamiento de protocolo MESIF}
En el caso del protocolo MESIF, los resultados registrados a lo largo del caso de estudio (Ver figura \ref{fig:pcounterMESIF})) dan cuenta de dos situaciones a destacar relacionadas a los eventos detectados por concepto de escrituras de líneas de caché de nivel 1 a nivel 2, fenómeno que indica que un valor presente en ambos niveles de caché sufrió una modificación que hace que se propague hacia los niveles de caché inferiores.
La figura \ref{fig:pcounterMESIFa}) presenta el total de sobre escrituras realizadas desde el caché de datos de nivel 1 al caché de nivel 2. Es fácil notar que las tendencias dominantes nuevamente se atribuyen al escenario multi-hilo, que sobrepasa ampliamente al caso de lectura exclusiva desde un socket. Ésta situación respalda el escenario descrito en el análisis previo sobre el comportamiento de caché de datos de nivel 1, donde las sobre correcciones ocurridas en el sistema producto de la modificación de un recurso compartido termina aumentando excesivamente la comunicación práctica entre componentes. En el caso del resultado de la figura \ref{fig:pcounterMESIFb}) la situación resulta aún más dramática. 

En ese resultado se contabilizan los eventos de sobre escritura desde nivel 1 a nivel 2 sobre líneas que presentan estado \emph{Invalid} de acuerdo al protocolo \emph{MESI}, es decir, corrección de valores que ya fueron descartados por otro componente del sistema. Ésta situación da cuenta de cómo la coordinación a nivel de las líneas de caché se va tornando caótica a medida que se intensifican los escenarios concurrentes en el acceso al socket. Cabe destacar que tal como antes, las tendencias generales de los eventos registrados en el caso concurrente se adecúan mucho con la del tiempo del caso de estudio.

\subsection{Comportamiento de Predicción de ejecución del Procesador}
Una de las características fundamentales de los equipos modernos es la capacidad de anticipación de ciertas operaciones a realizar, ya sea cómputo de ciertos valores así como modificaciones o accesos a porciones de memoria. Para ello, durante la ejecución de un programa en que pueden continuar distintas secuencias de pasos, el sistema pre calcula el valor de algunos flujos a seguir de manera de anticiparse a dicho momento. Sin embargo, existen situaciones en que tal predicción resulta errónea con el costo de, primero, descartar los datos computados y, segundo, recalcular los valores necesarios. Tal escenario es denominado \emph{Branch Misprediction} y es justamente lo retratado por los resultados de la figura \ref{fig:pcounterMissBranchPrediction}).

Como se aprecia en los resultados, la cantidad de predicciones erróneas del sistema se dispara a medida que se incorporan hilos en el consumo de datos, escenario que necesariamente termina impactando los tiempos de operación y procesamiento finales. Siguiendo la tendencia de los valores recopilados por concepto de eventos muestreados, tanto los resultados de las figuras \ref{fig:pcounterMissBranchPredictiona}) como \ref{fig:pcounterMissBranchPredictionb}) siguen la misma tendencia de los tiempos finales, contribuyendo evidenciando más pistas de que la compartición del Internet socket termina degenerando el sistema.

\subsection{Comportamiento de fallo en solicitud de datos}
El resultado de la figura \ref{fig:pcounterOFFCore} refleja la dinámica de llamadas que realizan las distintas CPU solicitando datos que no disponen en sus bancos de memoria directos. Nuevamente el caso al usar múltiples hilos domina ampliamente sobre el escenario de lectura exclusiva.

El resultado de la figura \ref{fig:pcounterOFFCorea} ilustra el registro de solicitudes para leer datos en porciones ajenas al núcleo en cuestión. Es particularmente interesante que la tendencia frente al caso concurrente es muy similar al comportamiento detectado en el estudio de porcentajes de llamadas a funciones de bloqueo, abriendo la posibilidad de que dicha similitud se deba a que como el spinlock de protección del socket compartido sea una referencia cruzada entre todos los threads, su localización exacta varíe de momento en momento, haciendo necesario éste tipo de comunicación para dicha validación.

\subsection{Comportamiento de caché de nivel 2}
Similar al caso de los registros de caché de datos de nivel 1, el nivel 2 (Ver figura \ref{fig:pcounterL2}) presenta una dinámica de modificación que se acrecienta a medida que se incorporan más threads, siguiendo lo que ha sido la tónica de los demás resultados.

Un resultado particularmente interesante es el registrado en la figura \ref{fig:pcounterL2b} y \ref{fig:pcounterL2e} que hacen referencia a modificaciones en caché de nivel 2 de tipo \emph{RFO (Request For Ownership)}. Según dicta el protocolo MESI, una línea de caché sólo puede ser escrita si presenta un estado \textbf{M - Modificada} o \textbf{E - Exclusiva}. En caso de presentar un estado \textbf{S - Compartida} las copias deben ser invalidadas antes de la modificación, acción que se realiza por medio de las solicitudes \emph{RFO}. En éste resultado, se aprecia como claramente, las solicitudes de dicho tipo para modificación de líneas de caché en segundo nivel se desata frente al aumento de hilos consumiendo el mismo socket. Más evidencia para entender que, al tener un recurso compartido en distintas porciones del sistema, las modificaciones sobre el mismo se vuelven más significativas y perjudiciales al sistema completo.

\subsection{Comportamiento del último nivel de caché}
La dinámica del último nivel de memoria resulta la menos significativa en términos del impacto para con el caso de estudio, dado que es más inusual que se lleven a éste nivel las referencias que se postulan responsables del fenómeno estudiado. Sin embargo, los resultados de éste estudio presentes en la figura \ref{fig:pcounterLLC}) no dejan de ser interesantes.

Como ha sido tendencia en todos los resultados presentados, la compartición del socket entre diferentes hilos también impacta las referencias alocadas en éste nivel de datos, aunque sin seguir exactamente las mismas tendencias reconocidas a lo largo del estudio. Aun así, éstos resultados son importantes pues en éste nivel de memoria, la latencia en el acceso a los datos es la más significativa, lo que hace que un alto tráfico de la misma impacte en los tiempos finales también del caso de estudio.

\subsection{Comportamiento de los canales de comunicación propios de la arquitectura Intel® \emph{QuickPath}}

Los resultados de la figura \ref{fig:pcounterQPI} dan cuenta del nivel de comunicación de salida y entrada sobre los distintos nodos NUMA del sistema. En éste resultado se reconocen 3 tendencias principales:
\begin{itemize}
\item Figuras \ref{fig:pcounterQPIa} y \ref{fig:pcounterQPIc} que ilustran la cantidad de ocurrencias de importación de información, vale decir, el arribo de datos desde alguna porción externa al núcleo de procesamiento. En éste caso, las tendencias son similares a lo que ha marcado la tendencia general de los demás resultados, presentando una amplia dominación de los valores del caso concurrente.
\item Figura \ref{fig:pcounterQPIb}, que da cuenta de los niveles ce comunicación producidos por concepto de exportación de datos por los canales \emph{QuickPath} a otro nodo NUMA o a los bancos de memoria. En éste caso las tendencias son similares para los casos de acceso concurrente y para el acceso exclusivo hasta los 6 hilos de consumo. En escenarios de mayor concurrencia, el caso multi-hilo prosigue su aumento de eventos de exportación de datos mientras en el caso de accesos exclusivos, se comienzan a reducir. La tendencia en éste caso se repite con respecto a una curva de naturaleza logarítmica que se adecúa a los tiempos netos conseguidos en el caso de estudio.
\item Figuras \ref{fig:pcounterQPId} y \ref{fig:pcounterQPIe}. La primera ilustra la salida o exportación de datos desde el núcleo de procesamiento al último nivel de caché. La segunda se refiere al número de ciclos de procesamiento en que los canales QPI se estancan debido a falta de crédito en los protocolos de nivel de la capa Link de la arquitectura \emph{QuickPath}. Ambos con crecimientos muy significativos a medida que se incrementa el número de hilos y que se separan rápidamente de los registros en el escenario de acceso exclusivo al socket. Se debe prestar especial atención al resultado de la figura \ref{fig:pcounterQPIe} correspondiente a los ciclos de estancamiento pues, dicha situación da cuenta de que la interacción en los canales de comunicación está siendo tan intensa que está saturando los mismos, complicando la transmisión efectiva de datos a nivel de los canales de comunicación y cayendo en los denominados “ciclos muertos” por los mismos protocolos definidos para coordinar la comunicación en ésta arquitectura. Vale decir, es un síntoma de alta comunicación por los canales QPI.
\end{itemize}
En términos generales, los resultados de ésta sección siguen avalando el diagnóstico original de \emph{caché bouncing} por compartición del socket.

\subsection{Comportamiento de mecanismos de control del protocolo \emph{SNOOP}}
El último apartado de resultados presentado en la figura \ref{fig:pcounterSNOOP} corresponde a la interacción dada por los mecanismos de coordinación de caché del protocolo \emph{SNOOP}. A pesar de que en todos los escenarios, los registros de la prueba multi-hilo dominan por sobre los registrados en la prueba de acceso exclusivo, se visualizan dos comportamientos principalmente.

Los resultados de las figuras \ref{fig:pcounterSNOOPc}, \ref{fig:pcounterSNOOPd} y \ref{fig:pcounterSNOOPe} dan cuenta del uso de mecanismos de SNOOP para correcciones al último nivel de acuerdo a los cambios de estado del protocolo MESI. En éste caso, las tendencias son más variadas pero se preserva la situación de mando del caso multi-hilo en todos los resultados. Finalmente, los resultados de las figuras \ref{fig:pcounterSNOOPa} y \ref{fig:pcounterSNOOPb} dan cuenta de la cantidad de mensajes de solicitud de datos de líneas de caché o de marcación de invalidez de líneas de caché de niveles primario o secundario. Ambas tendencias son de crecimiento acelerado y de basto dominio en el caso concurrente. Nuevamente ésta situación avala el diagnóstico de \emph{caché bouncing} en el caso multithread.

\subsection{Correlación de Eventos}
Para poder comprender mejor la tendencia de comportamiento en el experimento entre los distintos eventos capturados se repasaron posibles mecanismos de visualziación que permitieran una simple comparación entre dichas mediciones. Se optó por emplear una visualziación mediante el uso de una matriz de correlación\footnote{\url{https://en.wikipedia.org/wiki/Correlation_and_dependence#Correlation_matrices}}, de manera de poder detectar facilmente conjuntos de eventos relacionados, ello combinando algúna estratégia de clusterización en el proceso de visualizar los datos. Ésta técnica es muy práctica y se ha empleado en otros escenarios sobre el mismo kernel en otras investigaciones con buenos resultados \cite{paper:clusteringKernel}.

Los eventos sobre los que se prestan especial atención son los relacionados a los criterios de selección antes explícitos, agrupados en la tabla \ref{table:eventos}, y cuya traducción a códigos de registro viene dada por la tabla \ref{table:codigoseventos}.

\begin{table}[]
\centering
\begin{tabular}{|l|l|p{0.58\linewidth}|}
\hline
Nivel de Inspección                 & Registro & Descripción                                                \\ \hline
\multirow{8}{*}{Uso de QPI}         & r500104  & Cycles GQ data is imported from Quickpath interface        \\ \cline{2-3} 
                                    & r500204  & Cycles GQ data is imported from Quickpath memory interface \\ \cline{2-3} 
                                    & r500404  & Cycles GQ data is imported from LLC                        \\ \cline{2-3} 
                                    & r500105  & Cycles GQ data sent to the QPI or QMC                      \\ \cline{2-3} 
                                    & r500205  & Cycles GQ data sent to LLC                                 \\ \cline{2-3} 
                                    & r500405  & Cycles GQ data sent to cores                               \\ \cline{2-3} 
                                    & r500420  & Quickpath Home Logic remote read requests                  \\ \cline{2-3} 
                                    & r500820  & Quickpath Home Logic remote write requests                 \\ \hline
\multirow{3}{*}{Snoop}              & r530451  & L1D cache lines replaced in M state                        \\ \cline{2-3} 
                                    & r530251  & L1D cache lines allocated in the M state                   \\ \cline{2-3} 
                                    & r530851  & L1D snoop eviction of cache lines in M state               \\ \hline
\multirow{15}{*}{Pasos entre Cache} & r500108  & Number of LLC read hits                                    \\ \cline{2-3} 
                                    & r500208  & Number of LLC write hits                                   \\ \cline{2-3} 
                                    & r500109  & Number of LLC read misses                                  \\ \cline{2-3} 
                                    & r500209  & Number of LLC write misses                                 \\ \cline{2-3} 
                                    & r50010a  & LLC lines allocated in M state                             \\ \cline{2-3} 
                                    & r50020a  & LLC lines allocated in E state                             \\ \cline{2-3} 
                                    & r50040a  & LLC lines allocated in S state                             \\ \cline{2-3} 
                                    & r50080a  & LLC lines allocated in F state                             \\ \cline{2-3} 
                                    & r500f0a  & LLC lines allocated                                        \\ \cline{2-3} 
                                    & r50010b  & LLC lines victimized in M state                            \\ \cline{2-3} 
                                    & r50020b  & LLC lines victimized in E state                            \\ \cline{2-3} 
                                    & r50040b  & LLC lines victimized in S state                            \\ \cline{2-3} 
                                    & r50080b  & LLC lines victimized in I state                            \\ \cline{2-3} 
                                    & r50100b  & LLC lines victimized in F state                            \\ \cline{2-3} 
                                    & r501f0b  & LLC lines victimized                                       \\ \hline
\multirow{4}{*}{MESI}               & r501f0b  & L2 data demand loads in E state                            \\ \cline{2-3} 
                                    & r530126  & L2 data demand loads in I state (misses)                   \\ \cline{2-3} 
                                    & r530326  & L2 data demand loads in M state                            \\ \cline{2-3} 
                                    & r530526  & L2 data demand loads in S state                            \\ \hline
\end{tabular}
\caption{Colección de eventos resumidos para la inspección de los canales de comunicación del sistema en escenarios multithread.}
\label{table:codigoseventos}
\end{table}

\begin{figure}[h!]
	\centering
	\hspace*{\fill}
	\subfigure[]{
		\includegraphics[width=.45\textwidth]{imagenes/corrgram0.png}
		\label{fig:corrgram:a}
	}\hfill
	\subfigure[]{
		\includegraphics[width=.45\textwidth]{imagenes/corrgram1.png}
		\label{fig:corrgram:b}
	}
	\caption{Resultado de la visualización de la matriz de correlación con el software \emph{statgraphics}. En la figura \ref{fig:corrgram:a} se pueden visualizar los datos en bruto, mientras en \ref{fig:corrgram:b} se presentan los datos agrupados, tras ordenar la matriz de acuerdo al criterio de los vectores propios, logrando un efecto clusterizador.}
	\label{fig:corrmatrix}
	\hspace*{\fill}
\end{figure}


Para ésta tarea, se empleó el software de visualización \emph{statgraphics}\footnote{\url{http://www.statgraphics.com/}}. El software tiene la capacidad de generar una visualización aprovechando un ordenamiento por medio del primer vector propio de la matriz de correlación construida, de manera de generar un efecto clusterizador sobre los datos, agrupando aquellos con alta correlación.

Como se puede apreciar en su resultado, la mayoría de los eventos estudiados en la matriz de correlación presentan un enorme grado de similitud en sus tendencias (Próximos a 1), lo que combinado con los resultados anteriores de las curvas de registro colectadas con el estudio de \emph{performance counters} da que el grueso de los registros de eventos colectados sigue la misma tendencia de crecimiento del tiempo registrado por el caso de estudio, lo que contribuye nueva evidencia de que el conjunto de mecanismos de comunicación verificables por ésta vía está comprometida por efecto del acceso concurrente a una estructura única.

\section{Conclusiones}
A raíz del estudio de canales de comunicación de hardware del sistema se pueden rescatar varios aspectos interesantes:
\begin{itemize}
\item Se identificó un comportamiento creciente en el registro de eventos capturados, trasversal entre todos los eventos postulados en el estudio.
\item Se recabó evidencia de distintos niveles de comunicación cómo transferencia de valores entre niveles de caché y memoria, en conjunto con la actividad de los protocolos de coherencia y corrección de líneas de caché que apoya la hipótesis del escenario de \emph{caché bouncing} en el sistema estudiado.
\item se determinó experimentalmente que bajo el régimen de acceso concurrente a un mismo socket, se evidencian más problemas reflejados en el aumento de predicciones erróneas de los procesadores y de grandes aumentos en los canales de comunicación de datos entre componentes de procesamiento, evidencia que respalda la sospecha de la existencia de una estructura (el spinlock del socket) altamente compartida y requerida por los distintos cores, situación que degenera en degradaciones importantes del sistema.
\item Al observar en detalle la mayoría de los eventos, se da cuenta de cómo la tendencia de saturación sigue un régimen similar al de los tiempos generales del caso de estudio.
\item La colección de eventos correlacionados evidenció una tendencia generalizada y uniforme sobre los eventos involucrados, los cuales además de estar fuertemente correlacionados entre sí, siguen la misma tendencia de los tiempos del caso de estudio.
\end{itemize}
\chapter{Estudio de Distribución de Carga}
La tercera hipótesis de investigación plantea como responsable del mal rendimiento presentado en el caso de estudio al sistema de gestión y administración de tareas en el sistema operativo. En escenarios multicore como el estudiado, es normal que el sistema operativo realice como procedimiento de rutina la migración de procesos y la re-locación de recursos y datos para evitar la saturación de las componentes del mismo. Un caso práctico de ello es cuando un núcleo de procesamiento está sobre exigido y el sistema operativo redistribuye los procesos que están ejecutándose en dicho núcleo entre los otros procesadores disponibles del sistema con el costo que ello significa. Ésta práctica es conocida como \textbf{distribución de carga}, y a pesar de que existen varios mecanismos de aprovechamiento de dicho esquema como una estrategia de balanceo de carga en arquitecturas como la estudiada \cite{paper:NUMA}, en ciertos escenarios puede degradar el desempeño general del sistema.

Esta hipótesis plantea que dicho proceso de reasignación de recursos sería perjudicial en escenarios de concurrencia sobre arquitecturas como la estudiada, basándose en que mientras un proceso está en plena ejecución, al incorporar más y más tareas en el mismo núcleo de procesamiento agregando nuevos hilos de ejecución, sería el sistema operativo quien comenzaría la reasignación automática de dichos hilos entre los distintos procesadores cayendo en problemas como perdida de referencias de memoria en niveles de caché primario, yendo así en contra del principio de localidad de acceso a la memoria. En su peor escenario, ésta teoría lleva al ya mencionado problema de \emph{caché bouncing} que corresponde al fenómeno de sobre corrección de los datos a nivel de líneas de cache de un procesador, producido por constantes cambios de contexto del \emph{scheduller} que genera migración de procesos. Un problema que ya se mencionó en el estudio de la sección previa.

Una alternativa que se ha estudiado para solventar éste problema es la técnica de \emph{Processor Affinity} que consiste en la asociación de tareas o procesos en CPUs específicas, de manera de controlar la ubicación de memoria y zona real de ejecución del código en la máquina.
\begin{defn}[ver \cite{article:processoraffinity}] \textbf{Processor Affinity} es una estrategia de trabajo que consiste en la asociación de ciertos procesos o hilos de trabajo con determinados núcleos de procesamiento lógicos de un sistema. Dicha asociación restringe la capacidad de ejecución del hilo o proceso exclusivamente a su núcleo de procesamiento asignado.
\end{defn}
En otras palabras, con la aplicación de \emph{processor affinity} se remueve la utilidad del mismo scheduler del sistema operativo para coordinar la mejor operación en la asignación de los hilos de ejecución a las distintas CPU disponible, reemplazándolo por un criterio de diseño humano construido consientes de la tarea que se desea realizar. Es una técnica muy ambiciosa en el sentido de que bien empleada puede proveer muy buenos resultados en el sistema \cite{paper:cacheaffinity}, sin embargo, es muy fácil errar al interpretar el diseño de operación que se desea coordinar, llevando a una mala implementación en la asignación de recursos que termina degradando fuertemente el desempeño del sistema completo.

En ésta tercer estudio se plantea la utilización de la técnica de \emph{processor affinity} en pos de conseguir un mejor rendimiento del caso de estudio presentado, ello por medio de la reasignación de los hilos de ejecución entre los distintos cores lógicos del sistema en pos de explotar mejor la localidad de recursos en una arquitectura como la que se dispone. Para evaluar lo anterior, se plantean distintos esquemas de asignación de recursos basados en argumentos arquitecturales del escenario de prueba y se evalúan comparativamente los resultados.

\section{Esquemas de Distribución}
Se diseñaron variados esquemas de asignación a fin de evaluar distintos enfoques de aprovechamiento del principio de localidad de memoria. En total se diseñaron 6 esquemas para evaluar combinaciones dinámicas de cores lógicos reconocidos por el sistema operativo en búsqueda de mejores rendimientos. Los esquemas se detallan a continuación:
\begin{description}
\item[Sin Processor Affinity] Asignación dinámica por el scheduller del sistema operativo. En éste caso, la elección la hace el sistema de acuerdo a complejos algoritmos que consideran prioridad de proceso, carga de CPU, entre otros. Sirve como punto base de comparación con respecto a los demás esquemas.
\item[DummyAffinity] Asignación directa de hilos a ejecución en el core 0 del sistema, así, todos los hilos se delegan al mismo core. Bajo éste esquema se presume que se puede aprovechar mejor la localidad de memoria al disponer en el banco de memoria más próximo al núcleo de ejecución todas las referencias necesarias, disminuyendo el efecto de sobrecarga de los protocolos de coherencia y coordinación de cache.
\item[EquitativeAffinity] Asignación secuencial equitativa entre los cores lógicos del sistema siguiendo la numeración que el sistema operativo dispone de los núcleos mismos. Supone que la distribución completamente justa y equitativa entre cores entrega un mejor rendimiento general al reducir la carga de procesamiento de cada core.
\item[PairAffinity] Asignación secuencial de hilos a cores de numeración par. Busca descartar un escenario donde los cores duplicados por efecto de la tecnología \emph{hyperthreading} de Intel pudieran no aprovechar cores reales, generando así un mejor desempeño final.
\item[ImpairAffinity] Asignación secuencial de hilos a cores de numeración impar. Similar al anterior variando los cores elegidos.
\item[NumaPairAffinity] Asignación de los hilos a cores del segundo conjunto NUMA disponible en el sistema. Sigue la idea de aprovechar toda una unidad lógica de procesamiento según la arquitectura \emph{Quickpath}, persiguiendo mejores tiempos al tener mejor acceso a memoria desde éstos cores.
\item[SimpleCoreAffinity] Asignación de los hilos a los primetos dos cores lógicos del sistema, correspondientes al primer core real duplicado por efecto \emph{HyperThreading}.
\end{description}

\begin{table}[]
\centering
\begin{tabular}{|l|l|}
\hline
\multicolumn{1}{|c|}{\textbf{Esquema}} & \multicolumn{1}{c|}{\textbf{Fórmula de numeración de afinidad por hilos}}                                  \\ \hline
DummyAffinity                          & 

$\displaystyle 0$
                                                                                            \\ \hline
EquitativeAffinity                     & $\displaystyle i+1 \pmod \Phi $                                                                                 \\ \hline
PairAffinity                           & $\displaystyle 2i \pmod \Phi $                                                                                 \\ \hline
ImpairAffinity                         & $\displaystyle 2i + 1 \pmod \Phi $                                                                               \\ \hline
SimpleCoreAffinity                     & $\displaystyle \frac{i \pmod 2}{2} \Phi$                                                                              
\\ \hline
NumaPairAffinity                       & 
$
\begin{cases}
    \displaystyle i \pmod \Phi & \displaystyle \text{si } i \pmod 2 = 0 \\
    \displaystyle (i-1) + \frac{\Phi}{2} \pmod \Phi & \displaystyle \text{si } i \pmod 2 \neq 0
\end{cases}
$ \\ \hline
\end{tabular}
\caption{Numeración de afinidad a cada hilo en ejecución, donde $i$ representa el i-ésimo hilo. y $\Phi$ corresponde al número de procesadores lógicos reconocidos en el sistema (24 para nuestro caso).}
\label{my-label}
\end{table}

Con los esquemas antes mostrados, se pretende explotar las bondades de la arquitectura NUMA, de los principios de localidad de memoria, y de proximidad de acceso en los datos, a través de una ganancia efectiva en los tiempos de la prueba del caso de estudio.

\section{Metodología de Experimentación}
Para evaluar los distintos esquemas de afinidad ya postulados, se modificó el caso de estudio para incorporar la asociación de hilos de ejecución para con los distintos cores del sistema. Para ello se aprovechó la flexibilidad de la librería \emph{PThreads}, responsable de brindar los hilos de ejecución en nuestro caso, para usar ciertas llamadas a sistema que nos otorgasen el control de la localidad de ejecución buscada. En concreto, se empleó la función \verb=pthread_attr_setaffinity_np()= que permite especificar precisamente la asociación de un hilo para con el core de ejecución deseado. 

La prueba anterior se recopiló en un nuevo experimento\footnote{\url{https://github.com/sebablasko/Test_DifferentAffinityThreadsBySocket}} que, para efectos de validez estadística, se ejecutó un total de 60 veces, y se rescataron los valores promedios de dichas ejecuciones, ello siguiendo las mismas lógicas de colección de datos del promedio de las ecuaciones \ref{eq:tupla1} y \ref{eq:promedio}.

\newpage

\section{Resultados}

\begin{figure}[h!]
	\centering
	\includegraphics[scale=.6]{resultados/processoraffinity-crop.pdf}
	\caption{Resultados experimentales de los distintos esquemas de afinidad.}
	\label{fig:resAffinity}
\end{figure}

\section{Análisis y Discusión de Resultados}
Los resultados experimentales ilustrados en la imagen \ref{fig:resAffinity} son diversos. En efecto, ningún esquema de distribución evaluado consiguió reducir verdaderamente el tiempo de ejecución con respecto al uso de una configuración trivial de un único thread. Sin embargo los resultados obtenidos reflejan distintas tendencias interesantes en los tiempos resultantes del experimento.

Una primera tendencia evidenciada en los resultados viene dado por los esquemas \textbf{DummyAffinity} y \textbf{SimpleCoreAffinity}, donde los tiempos se mantuvieron estables a lo largo de la evaluación de las distintas configuraciones de threads. Si recordamos, ambos esquemas son rígidos en la posibilidad de locación de los hilos en la CPU, el primero los asigna todos al core \#0 mientras que el segundo los asigna todos entre el core \#0 y el core \#12, numeración que da el sistema operativo a los núcleos duplicados por efecto de la tecnología \emph{HyperThreading}. De ésta manera, ambos esquemas tienen en común que ejecutan la totalidad de los hilos en un mismo núcleo real de procesamiento. Ahora bien, ninguno de los dos consigue ganancias efectivas de tiempo con respecto al uso de un único thread, ello se explica pues, dado que los distintos hilos son asignados todos a un mismo core, la ejecución de los mismos termina degenrando a una ejecución secuencial, es decir, por medio de la afinidad de proceso hemos eliminado la capacidad de paralelismo al llevar el experimento a un escenario monocore. Esto explica también el por qué los tiempos son tan uniformes sin importar la cantidad de hilos que se usen.

Una segunda tendencia reconocible es la ilustrada por el esquema \textbf{EquitativeAffinity}. Resultante como el de peor desempeño, éste esquema que se caracterizaba por la distribución justa de los hilos entre los cores registra siempre los peores tiempos de la prueba. Ésta tendencia se explica por la naturaleza de la arquitectura que se está usando. Como se mencionó antes, la distribución del sistema corresponde a dos nodos NUMA de procesamiento. La aplicación del esquema equitativo vulnera derechamente tanto los beneficios que provee el esquema NUMA como el principio mismo de localidad en acceso a la memoria, ello pues en la práctica, termina distribuyendo los distintos hilos en las componentes más alejadas (topológicamente) de la arquitectura disponible.

Finalmente, una tercera tendencia es la de crecimiento producido por el conjunto de los demás esquemas de afinidad. La similitud entre \textbf{PairAffinity} e \textbf{ImpairAffinity} era previsible pues el modelo de distribución era equivalente para ambos. La misma naturaleza sigue el esquema \textbf{NumaPairAffinity} que, a pesar de aprovechar la distribución de componentes para la localización de hilos, no consiguió mejores tiempos que los ya representados. Ahora bien, hay que destacar que en términos prácticos, éste último grupo de esquemas de distribución tuvo un rendimiento muy similar al del mismísimo scheduller del sistema operativo, más aún, con la técnica de \emph{processor affinity}, la ejecución del programa es rígida en la localidad del proceso con respecto al procesador seleccionado, lo cual produce que en ciertas situaciones --como al tener sobrecarga de memoria-- los niveles de caché no cooperen con la tarea general, y no brinden mayor beneficio a la prueba final, una garantía que si dispone el scheduller del sistema operativo, por lo tanto su desempeño no debe ser menospreciado y sólo da cuenta de que lamentablemente, el problema de rendimiento parece ser producto de un defecto de diseño inherente a la estructura que se está compartiendo, defecto que no permite optar a mejores tiempos sin importar la estrategia de paralelismo usada.


\section{Conclusiones}
A raíz del estudio de distribución de carga se pueden rescatar varios aspectos interesantes:
\begin{itemize}
\item Las técnicas de \emph{processor affinity} empleadas no consiguieron beneficio alguno en el rendimiento de consumo del caso de estudio, ello a pesar de evaluar distintos esquemas en pos de sacar provecho de la arquitectura evaluada.
\item Los distintos esquemas, a pesar de ser rígidos en la ejecución de hilos sobre ciertos cores, fueron competitivos con respecto a la capacidad dinámica del scheduller del sistema operativo en los tiempos producidos en el experimento.
\item Se determina que es el mismo socket –o alguna componente inherente de sincronización como su spinlock-- una estructura que se vuelve un punto de contención al emplear hilos paralelos, calificando al mismo como no apto para soportar accesos concurrentes, ello dado su diseño y mecanismos de protección implementados.
\end{itemize}
%\chapter{Estudio de Reuseport}

Como ya se estudió, los problemas del escenario multihilo reconocidos en los capítulos anteriores impactan principalmente en los tiempos de lectura de interfaces de red, donde bajos tiempos son siempre un elemento prioritario a garantizar. Entendiendo éste problema es que varios gigantes de la industria han planteado mecanismos alternativos que permitan aprovechar mejor los recursos de los computadores de arquitecturas modernas al emplear programación paralela. En ésta linea, la propuesta más prometedora hasta la fecha es la brindada por Google, denominada \emph{Reuseport}.

\emph{Reuseport} \cite{slides:googleReuseport} es una de las soluciones más usadas para hacer frente al problema descrito, por su gran efectividad en la practica. Corresponde a un desarrollo de Tom Herbert --Ingeniero de Google-- desarrollado precisamente para responder a los bajos desempeños generados al emplear un esquema de consumo de sockets como el planteado hasta el momento. El trabajo de Herbert plantea que las distintas estratégias que se puedan adoptar sobre un único socket para mejorar la performance en su acceso concurrente terminan no resultando efectivas por mantener el punto de contención al compartir la misma estructura, lo que vendría siendo el origen del problema y se presenta disponible (según el mismo autor) desde la versión 3.9 del kernel de Linux\footnote{En la actualidad, la característica reuseport ha sido portada a otras versiones del kernel, estando disponible en versiones desde la 2.6 en variadas distribuciones de Linux.}.

\begin{figure}[h!]
	\centering
	\hspace*{\fill}
	\subfigure[Socket Tradicional]{
		\includegraphics[width=.45\textwidth]{imagenes/socketNormal.png}
	}\hfill
	\subfigure[Socket con opción Reuseport activada]{
		\includegraphics[width=.45\textwidth]{imagenes/socketReuseport.png}
		\label{fig:soReuseport}
	}
	\caption{Comparativo del funcionamiento de sockets tradicionales con respecto a Reuseport. En éste último, basta con que el primer socket que enlace al puerto en cuestión tenga la opción habilitada para que cualquier socket posterior se pueda sobre acomplar sobre el mismo puerto, compartiendo así el consumo de datos del mismo.}
	\label{fig:socketHandshake}
	\hspace*{\fill}
\end{figure}

Reuseport se plantea como una opción para los sockets estándar de Linux que promete una mejora en los tiempos de consumo en la atención a un determinado puerto. Dicha opción permite compartir un mismo puerto local del sistema entre múltiples sockets distintos. De ésta manera, conservando el esquema multithread, cada thread puede tener exclusividad en el consumo de un socket eliminando el punto de contención único que se ocaciona al compartir un único socket. Al usar la opción \emph{Reuseport} la tarea de distribución de paquetes entre los distintos sockets que compartan un puerto local es delegada directamente al kernel, el cual asigna aleatoreamente los paquetes recibidos entre los sockets que escuchan el mismo puerto.

\section{Implementación}
La opción está implementada íntegramente en el código fuente del kernel, distribuida entre distintos archivos que hacen uso de la misma. Por sólo mencionar algunos, los archivos responsables de los mecanímsos de conexión en capa IP\footnote{\url{http://lxr.free-electrons.com/source/net/ipv6/inet6_connection_sock.c?v=3.14}}, la implementación de los sistemas de tablas de hash para el módulo de red\footnote{\url{http://lxr.free-electrons.com/source/net/ipv6/inet6_hashtables.c}}, los mecanísmos base de la api de conexiones de los sockets de internet\footnote{\url{http://lxr.free-electrons.com/source/net/ipv4/inet_connection_sock.c?v=3.18}} y la mismísima implementación de UDP en el kernel\footnote{\url{http://lxr.free-electrons.com/source/net/ipv4/udp.c}} se han visto tocadas.

La opción está diseñada para habilitarse por medio de la llamda a sistema \verb=setsockopt()= indicando el descriptor del socket a habilitar y especificando como identificador para ésta opción el flag \verb=SO_REUSEPORT=, que es una constante incluída de los encabezados de \verb=socket.h= del kernel mismo. El socket sobre el que se habilite la opción debe ser el primero en tomar posesión del puerto en cuestión para poder compartirlo con otros sockets (ya sea que estos últimos cuenten con la funcionalidad reuseport habilitada o no).

El mecanísmo de distribución de paquetes que emplea el kernel se basa en una tabla de hash que aprovecha una 4-tupla de valores correspondientes a los mismos valores de las tuplas de direccionamiento constitutivas de un paquete según el modelo OSI. En la práctica, reuseport dispone de un mecanímso de hash que usa los identificadores de ip y puerto (de origen y destino en ambos casos) para la construcción del registro en la tabla de hash. La función de distribución del hash está confeccionada para lograr una distribución uniforme de valores \cite{article:reuseport}.


\section{Uso en la práctica}
Como se mencionó, para usar ésta opción se debe modificar la estructura socket que primero tome control del puerto local para escucharlo por medio de la llamada \verb=bind()=. Con ello, el puerto puede ser posteriormente re-acoplado por otros sockets sin la necesidad de que éstos últimos tengan la característica activada (Ver imagen \ref{fig:soReuseport}).

La adopción de ésta funcionalidad ha sido incorporada a distinto software con requerimientos común de alta disponibilidad de atención de consultas con buenos resultados. Ejemplos de lo anterior son productos como \emph{nginx}\footnote{\url{https://www.nginx.com/blog/socket-sharding-nginx-release-1-9-1/}} o \emph{Apache Web Server} \cite{paper:apache} que han incorporado ésta característica en versiones recientes con buenos resultados.

\section{Rendimiento en la Práctica}
Para evaluar el rendimiento práctico de la opción reuseport se modificó sutilmente el caso de estudio evaluado a lo largo de la investigación para hacerlo compatible con éste enfoque de trabajo. Recordémos que reuseport se basa en la acción de múltiples sockets consumiendo datos, ya no sólo uno, por lo que hay ciertas salvedades que estipular. Al igual que en el escenario original, el objetivo es calcular el tiempo de consumo de 1 millón de paquetes desde un socket UDP en escenario de saturación. La diferencia en éste caso radica en que, en pos de aprovechar las promesas de reuseport, se cambia el consumo concurrente desde el mismo socket usando múltiples hilos por un consumo de múltiple hilos que consume cada uno a un socket distinto, alocado en la misma interzaf lógica del sistema del socket saturado usando la opción reuseport. De ésta manera, el consumo por socket ahora es exclusivo para cada thread, eliminando el punto de contención detectado en la prueba original.

\begin{figure}[!h]
	\centering
	\includegraphics[scale=.3]{imagenes/fcfm}
	\caption{Nuevo esquema de la prueba UDP en escenarios donde se aproveche Reuseport.}
	\label{fig:casoPruebaReuseport}
\end{figure}

Para poder llevar a cabo ésta prueba, y siguiendo el régimen de operación que exige ésta solución, es necesario garantizar que el primer socket creado para la recepción de datos posea la opción \verb=SO_REUSEPORT= habilitada. Además la cantidad total de consumo (1 millón de paquetes) debe ser redistribuida entre los sockets que se evalúen. La medición de tiempos en éste caso debe ser en función del último socket que termine el consumo de su cuota, pues dicho tiempo representa el tiempo total en que se completa la transferencia total en relación al caso de estudio original. A modo de formalización, se estipúlan los siguientes conceptos importantes en el nuevo caso de estudio:
\begin{description}
\item[Tiempo Mínimo] Corresponde al tiempo en que el primer socket termina de consumir su cuota de consumo asignada.
\item[Tiempo Neto] Corresponde al tiempo en que el último socket termina de consumir su cuota de consumo asignada. De esta forma, corresponde al tiempo en que se completa la trasferencia total de datos.
\item[Cuota de Consumo] Corresponde a la porción de paquetes que le corresponde consumir a cada socket. Si el envío total se denomina como $N$ y se tienen $k$ sockets para emplear el consumo con reuseport, la cuota de consumo de cada uno es el cociente $\frac{N}{k}$. El valor de la cuota de consumo es una restricción de diseño que impone la implementación de reuseport al usar un hash de distribución uniforme, por lo que en éste caso, el valor es idéntico para cada instancia socket.
\end{description}

Un esquemático de ésta prueba puede apreciarse en la figura \ref{fig:casoPruebaReuseport}. La figura \ref{fig:resultadosReuseport} ilustra los tiempos obtenidos a lo largo del nuevo caso de estudio. Sorprendentemente, reuseport consigue tiempos netos gradualmente mejores a medida que se incorporan threads hasta un tope en torno a los 8 sockets donde el rendimiento se estanca para luego empeorar. Otro aspecto interesante es una característica no documentada, y es que según los resultados obtenidos, reuseport garantiza un tiempo de operación mínimo basado sólo en el consumo por socket. Vale decir, el tiempo en que el primer socket completa su cuota (tiempo minimo) está relacionado a ser directamente proporcional a la cuota de consumo. Luego, como a medida que se incorporan más y más sockets la cuota de consumo se reduce (en medida exponencial en nuestro experimento), y así también el tiempo mínimo se reduce de la misma manera.

\begin{figure}[!h]
	\centering
	\includegraphics[scale=.6]{resultados/reuseport1}
	\caption{Gráfico con tiempos de consumo de reuseport para el nuevo caso de estudio.}
	\label{fig:resultadosReuseport}
\end{figure}

\subsection{Diferencias entre arquitecturas}
[[[[[[Aun pendiente sobre si va esto]]]]]]]]]

\section{Evaluación de Optimizaciones Adicionales}
Los resultados anteriores evidencian un claro incremento en el rendimiento alcanzable por la prueba ya conocida aprovechando múltiples sockets en el consumo. Sin embargo cabe preguntarse si será posible mejorar aún más esos resultados combinando la opción de reuseport con técnicas de consumo concurrente sobre los sockets. Una hipótesis que apoya ésta idea se basa en que, al emplear la opción reuseport, los sockets tienen una menor cuota de consumo cada uno, por lo tanto, un acceso concurrente en ese escenario de bajo consumo podría impactar en una reducción de tiempos finales al generar una nula sobrecarga en cada caso.

\subsection{Incorporación de Multithreading}
En la presente sección se validará el rendimiento alcanzable por la opción \emph{Reuseport} usándola en combinación con diferentes estratégias de multithreading como las vistas en el capítulo anterior. Para éste caso, se evaluó el mismo caso de estudio anterior considerando además distintas configuraciones de consumo concurrente para cada instancia de socket, probando configuraciones con 1, 2, 4 y 8 threads en cada caso. Se limitó a 8 el máximo de threads a consumir pues en el peor caso, usando 64 sockets compartiendo el mismo puerto, y cada uno con 8 hilos de consumo se llevaría al sistema a un escenario con demasiados procesos en ejecución.

\begin{figure}[h!]
	\centering
	\hspace*{\fill}
	\subfigure[]{
		\includegraphics[width=.47\textwidth]{resultados/reuseport21}
		\label{fig:reuseport2min}
	}\hfill
	\subfigure[]{
		\includegraphics[width=.47\textwidth]{resultados/reuseport22}
		\label{fig:reuseport2neto}
	}
	\caption{Graficos del resultado del nuevo caso de estudio incorporando lectura con múltiples hilos. En la leyenda de cada gráfico se indica el número de threads que consumen concurrentemente cada socket con reuseport.}
	\label{fig:resultadosReuseport2}
	\hspace*{\fill}
\end{figure}

Los resultados están ilustrados en el gráfico de la figura \ref{fig:resultadosReuseport2}. El primer gráfico que indica los tiempos mínimos \ref{fig:reuseport2min} evidencia que la incorporación de hilos para lectura concurrente mayoritariamente empeoró el desempeño registrado sin concurrencia, sin embargo, las tendencias se conservan a lo largo de la prueba, dando siempre tiempos mínimos mejores a medida que se emplean más sockets. De ésta forma, la clave en los tiempos mínimos parece ir más relacionada a la cantidad de sockets que se empleen en cada caso. En el segundo gráfico que ilustra los tiempos netos de la prueba \ref{fig:reuseport2neto} se identifica como la ejecución con threads perjudica siempre el desempeño de reuseport, degradando los tiempos y no brindando ninguna mejora práctica. Es interesante resaltar también que a partir de valores muy altos, parece ser despreciable el sobrecosto que añade tener más threads en cada socket, comportamiento que apoya el planteamiento inicial de que, a una menor cuota de consumo, el sobrecosto disminuye.

\subsection{Incorporación de Multithreading con Processor Affinity}
Similar al estudio del capítulo anterior en la arista de distribución de carga, en ésta sección se evaluó el desempeño de Reuseport en escenarios donde múltiples hilos de consumo acceden al socket, y fuesen distribuidos lógicamente entre los distintos CPU del sistema aprovechando la técnica de \emph{Processor Affinity}. En éste caso, se evaluaron 3 configuraciones:

\begin{description}
\item[SOsched] Los hilos son distribuidos por el mismo sistema operativo a través de la oepración de su algoritmo de \emph{schedulling}. Es la base de comparación para con los demás esquemas de distribución.
\item[All0sched] Los distintos hilos de ejecución se alocan en el mismo core (core 0 por simplicidad). Persigue aprovechar localidad de datos en la ejecución simultanea de los hilos.
\item[Equitativesched] Los hilos se distribuyen equitativamente entre los cores lógicos del sistema, persiguiendo una distribución de carga entre unidades de procesamiento real.
\end{description}

\begin{figure}[h!]
	\centering
	\subfigure[]{
		\centering
		\includegraphics[width=.47\textwidth]{resultados/reuseport31}
		\label{fig:reuseportsomin}
	}
	\subfigure[]{
		\centering
		\includegraphics[width=.47\textwidth]{resultados/reuseport32}
		\label{fig:reuseportsoneto}
	}
	\subfigure[]{
		\centering
		\includegraphics[width=.47\textwidth]{resultados/reuseport33}
		\label{fig:reuseportall0min}
	}
	\subfigure[]{
		\centering
		\includegraphics[width=.47\textwidth]{resultados/reuseport34}
		\label{fig:reuseportall0neto}
	}
	\subfigure[]{
		\centering
		\includegraphics[width=.47\textwidth]{resultados/reuseport35}
		\label{fig:reuseporteqmin}
	}
	\subfigure[]{
		\centering
		\includegraphics[width=.47\textwidth]{resultados/reuseport36}
		\label{fig:reuseporteqneto}
	}
	\caption{.}
	\label{fig:resultadosReuseport3}
\end{figure}

Los resultados de esta evaluación se encuentran en la figura \ref{fig:resultadosReuseport3}. La primera observación relevante a destacar es que, en general, las tendencias de comportamiento tanto en tiempos mínimos como en tiempos netos, se preserva independiente del mecanísmo de distribución de threads que se emplee. Los registros de tiempos mínimos mantienen su tendencia de decrecimiento a medida que se incorporan más sockets y la tendencia de los tiempos netos preserva su decaimento en torno a los 8 sockets, sin variaciones importantes entre esquemas.

Sin embargo, hay ciertos elementos a destacar que se evidencian en los nuevos esquemas de distribución. Tanto en la configuración \emph{Equitativesched} (Fig. \ref{fig:reuseporteqmin} y \ref{fig:reuseporteqneto}) como en \emph{All0sched} (Fig. \ref{fig:reuseportall0min} y \ref{fig:reuseportall0neto}) ocurre un fenómeno interesante en la disperción o variación de los valores al incorporar threads. En ambos esquemas las variaciones usando distinta cantidad de threads en configuraciones con bajo número de sockets se presenta un desempeño bastante parejo, a diferencia del esquema \emph{SOsched} (Fig. \ref{fig:reuseportsomin} y \ref{fig:reuseportsoneto}). Una tendencia que se mantiene al aumentar el número de sockets consumiendo datos haciendo las difernecias de tiempo muy pequeñas entre usar uno o muchos threads por socket. Éste comportamiento refleja que las técnicas de \emph{processor affinity} tienen buenos resultados en balancear la carga cuando se emplean pocos sockets y la cuota de consumo de cada uno es más alta, permitiendo un mejor balance de tiempos entre los threads que consumen cada socket, pero aún así no consiguen batir los tiempos obtenidos por reuseport. La razón de ello apunta a ser que, antes de incorporar paralelísmo usando hilos, se llega a un límite de rendimiento del socket mismo que no puede ser superado aplicando concurrencia de ésta manera, y que es precisamente el problema detectado en un principio.

\subsection{Análisis y Discusión de Resultados}
Como se aprecia de los resultados experimentales, la incorporación de multithreading en el consumo de datos desde una estructura socket con soporte para \emph{reuseport} siempre degrada el rendimiento completo del caso de estudio, independiente de si se emplean esquemas de distribución de carga entre los núcleos de procesamiento efectivo. Resulta interesante que más allá de las estratégias que se adopten para la mejor distribución de carga, las variaciones no permiten justificar una verdadera ganancia o pérdida entre esquemas que se adopten, siendo simplemente sobrecargas con respecto al esquema de distribución propio del sistema oeprativo.

Éste resultado viene a confirmar la conclusión de los estudios del capítulo anterior que terminan por postular que es la misma estructura socket de Linux la que no cuenta con un diseño compatible con técnicas de consumo concurrente, degenerando siempre en escenarios de degradación de performance generalizada en su consumo.

\section{Aspectos Negativos}
El esquema que propone reuseport replantea uno de los principios fundamentales que postula el modelo OSI, que es la relación ''uno a uno'' entre estructuras sockets y direcciones de puertos locales. Como reuseport plantea la compartición de puertos locales se pierde la exclusividad de los mismos para cada socket. Otra caracteristica que trastoca ésta alternativa es la modificación en la manera de programación de aplicaciones. Como se mencionó, el uso de ésta técnica está condicionada a la habilitación de dicha opción a los sockets previo al momento de su acoplamiento con el puerto local a escuchar. Ésta situación significa en la práctica estar concientes de que toda aplicación que desee hacer uso de ésta característica debe modificar secciones profundas de su implementación en el código fuente para habilitarla, siempre y cuando el sistema donde se trabaje disponga de soporte para la misma.

Por otro lado, dado su funcionamiento la opción de reuseport está incorporada como código fuente del kernel de Linux, estando disponible según sus creadores en los kernels con versiones posteriores a la 3.9 (Finales del año 2013), ello no es del todo correcto. Un caso de ello son las distribuciones de Linux basadas en Debian --incluido el popular Ubuntu-- donde se han presentado varios problemas en versiones recientes las cuales, a pesar de contar con kernels 3.9+ y tener la implementación fuente de reuseport, poseen la constante de activación de la característica (el flag \verb=SO_REUSEPORT=) como código comentado y no utilizable\footnote{\url{https://lists.debian.org/debian-kernel/2015/02/msg00260.html}} u otros casos de distribuciones de linux que siguen la dinámica de actualización basadas en \emph{continuous release} que no incorporan los cambios encesarios para dar soporte a las funcionalidades de reuseport\footnote{https://github.com/circus-tent/circus/issues/699}, lo cual dificulta el uso de ésta opción al no dejarla disponible para el uso de los programadores de aplicaciones o llevando a errores de programas en el peor caso.

Una última dificultad asociada al uso de la opción \emph{Reuseport} se relaciona con su documentación. La documentación asociada a ésta característica es muy escasa, reducíendose prácticamente a la implementación misma disponible en el código fuente del propio kernel de Linux. No existe documentación a nivel de usuario que permita comprender ni mucho menos manipular las características de la opción en sí misma. En esa misma linea, aparece como un problema la implementación de ésta característica al ser una opción \emph{hardcodeada} en el kernel mismo de Linux, lo que hace inviable pensar cualquier modificación doméstica pensando en querer aprovechar su funcionamiento con ciertas modificaciones o adaptada a escenarios especiales. Ello pues toda modificación se traduce en modificar, recompilar e instalar todo un nuevo kernel, entendiendo en primer lugar lo complicado de dicha tarea, y en segunda instancia, sucitando el riesgo de posibles inestabilidades en el resultado, algo que en la práctica es inaceptable en entornos de producción. Ésto último da a entender que la característica \emph{Reuseport} no está pensada para ser modificable según escenarios especificos de operación.
\chapter{Estudio de Reuseport}

Como ya se estudió, los problemas del escenario multi-hilo reconocidos en los capítulos anteriores impactan principalmente en los tiempos de lectura de interfaces de red, donde siempre un elemento prioritario a garantizar es conseguir tiempos reducidos. Entendiendo este problema es que se han planteado mecanismos alternativos que permitan aprovechar mejor los recursos de los computadores de arquitecturas modernas al emplear programación paralela. En esta línea, la propuesta más prometedora hasta la fecha es la brindada por Google, denominada \emph{Reuseport}.

Reuseport \cite{slides:googleReuseport} es una de las soluciones más usadas para hacer frente al problema descrito, por su gran efectividad en la práctica. Corresponde a un desarrollo de Tom Herbert --Ingeniero de Google-- desarrollado precisamente para responder a los bajos desempeños generados al emplear un esquema de consumo de sockets como el planteado hasta el momento. El trabajo de Herbert plantea que las distintas estrategias que se puedan adoptar sobre un único socket para mejorar la performance en su acceso concurrente terminan no resultando efectivas por mantener el punto de contención al compartir la misma estructura, lo que vendría siendo el origen del problema y se presenta disponible (según el mismo autor) desde la versión 3.9 del kernel de Linux. Sin embargo, en la actualidad la característica reuseport ha sido portada a otras versiones del kernel, estando disponible en ediciones desde la 2.6 en variadas distribuciones de Linux.

\begin{figure}[h!]
	\centering
	\hspace*{\fill}
	\subfigure[Puerto tomado por un socket tradicional.]{
		\includegraphics[width=.45\textwidth]{imagenes/socketNormal.png}
		\label{fig:normalSocket}
	}\hfill
	\subfigure[Compartición de puerto usando reuseport.]{
		\includegraphics[width=.45\textwidth]{imagenes/socketReuseport.png}
		\label{fig:soReuseport}
	}
	\caption{Comparativo del funcionamiento de asociación sockets-puerto usando sockets tradicionales con respecto a sockets con la opción reuseport.}
	\label{fig:socketDiffereces}
	\hspace*{\fill}
\end{figure}

Reuseport se plantea como una opción para los sockets estándar de Linux que promete una mejora en los tiempos de consumo en la atención a un determinado puerto. Dicha opción permite compartir un mismo puerto local del sistema entre múltiples sockets distintos. De esta manera, conservando el esquema multithread, cada thread puede tener exclusividad en el consumo de un socket eliminando el punto de contención único que se ocasiona al compartir un único socket. Al usar la opción reuseport la tarea de distribución de paquetes entre los distintos sockets que compartan un puerto local es delegado directamente al kernel, el cual asigna aleatoriamente los paquetes recibidos entre los sockets que escuchan el mismo puerto.

\section{Implementación}
La opción está implementada íntegramente en el código fuente del kernel, distribuida entre distintos archivos que hacen uso de la misma. Por sólo mencionar algunos, los archivos responsables de los mecanismos de conexión en capa IP\footnote{\url{http://lxr.free-electrons.com/source/net/ipv6/inet6_connection_sock.c?v=3.14}}, la implementación de los sistemas de tablas de hash para el módulo de red\footnote{\url{http://lxr.free-electrons.com/source/net/ipv6/inet6_hashtables.c}}, los mecanismos base de la api de conexiones de los sockets de internet\footnote{\url{http://lxr.free-electrons.com/source/net/ipv4/inet_connection_sock.c?v=3.18}} y la mismísima implementación de UDP en el kernel\footnote{\url{http://lxr.free-electrons.com/source/net/ipv4/udp.c}} se han visto tocadas.

La opción está diseñada para habilitarse por medio de la llamada a sistema \verb=setsockopt()= indicando el descriptor del socket a habilitar y especificando como identificador para esta opción el flag \verb=SO_REUSEPORT=, que es una constante incluida de los encabezados de \verb=socket.h= del kernel mismo. El socket sobre el que se habilite la opción debe ser el primero en tomar posesión del puerto en cuestión para poder compartirlo con otros sockets (ya sea que estos últimos cuenten con la funcionalidad reuseport habilitada o no).

El mecanismo de distribución de paquetes que emplea el kernel se basa en una tabla de hash que aprovecha una 4-tupla de valores correspondientes a los mismos valores de las tuplas de direccionamiento constitutivas de un paquete según el modelo OSI. En la práctica, reuseport dispone de un mecanismo de hash que usa los identificadores de IP y puerto (de origen y destino en ambos casos) para la construcción del registro en la tabla de hash. La función de distribución del hash está confeccionada para lograr una distribución uniforme de valores \cite{article:reuseport}.


\section{Uso en la práctica}
Como se mencionó, para usar esta opción se debe modificar la estructura socket que primero tome control del puerto local para escucharlo por medio de la llamada \verb=bind()=. Con ello, el puerto puede ser posteriormente re-acoplado por otros sockets sin la necesidad de que estos últimos tengan la característica activada (Ver figura \ref{fig:soReuseport}).

La adopción de esta funcionalidad ha sido incorporada a distinto software con requerimientos común de alta disponibilidad de atención de consultas con buenos resultados. Ejemplos de lo anterior son productos como \emph{nginx}\footnote{\url{https://www.nginx.com/blog/socket-sharding-nginx-release-1-9-1/}} o \emph{Apache Web Server} \cite{paper:apache} que han incorporado esta característica en versiones recientes con buenos resultados.


\section{Rendimiento en la Práctica}
Para evaluar el rendimiento práctico de la opción reuseport se modificó sutilmente el caso de estudio evaluado a lo largo de la investigación para hacerlo compatible con este enfoque de trabajo. Recordemos que reuseport se basa en la acción de múltiples sockets consumiendo datos, ya no sólo uno, por lo que hay ciertas salvedades que estipular. Al igual que en el escenario original, el objetivo es calcular el tiempo de consumo de 1 millón de paquetes desde un socket UDP en escenario de saturación. La diferencia en este caso radica en que, en pos de aprovechar las promesas de reuseport, se cambia el consumo concurrente desde el mismo socket usando múltiples hilos, por un consumo de múltiple hilos en que cada uno lee datos de un socket distinto, asignado a la misma interfaz lógica del sistema del socket saturado usando la opción reuseport. De esta manera, el consumo por socket ahora es exclusivo para cada thread eliminando el punto de contención detectado en la prueba original. A este nuevo enfoque de evaluación que incluye una distribución de la cuota de trabajo se le denominó: \textbf{Nuevo caso de estudio}.

\begin{figure}[!h]
	\centering
	\includegraphics[scale=.5]{imagenes/nuevocaso.png}
	\caption{Nuevo esquema de la prueba UDP en escenarios donde se aproveche Reuseport correspondiente al nuevo caso de estudio.}
	\label{fig:casoPruebaReuseport}
\end{figure}

Para poder llevar a cabo esta prueba, y siguiendo el régimen de operación que exige esta solución, es necesario garantizar que el primer socket creado para la recepción de datos posea la opción \verb=SO_REUSEPORT= habilitada. Además la cantidad total de consumo (1 millón de paquetes) debe ser redistribuida entre los sockets que se evalúen. La medición de tiempos en este caso debe ser en función del último socket que termine el consumo de su cuota, pues dicho tiempo representa el tiempo total en que se completa la transferencia total en relación al caso de estudio original. A modo de formalización, se estipulan los siguientes conceptos importantes en el nuevo caso de estudio:
\begin{description}
\item[Tiempo Mínimo] Corresponde al tiempo en que el primer socket termina de consumir su cuota de consumo asignada.
\item[Tiempo Neto] Corresponde al tiempo en que el último socket termina de consumir su cuota de consumo asignada. De esta forma, corresponde al tiempo en que se completa la trasferencia total de datos.
\item[Cuota de Consumo] Corresponde a la porción de paquetes que le corresponde consumir a cada socket. Si el envío total se denomina como $N$ y se tienen $k$ sockets para emplear el consumo con reuseport, la cuota de consumo de cada uno es el cociente $\frac{N}{k}$. El valor de la cuota de consumo es una restricción de diseño que impone la implementación de reuseport al usar un hash de distribución uniforme, por lo que en éste caso, el valor es idéntico para cada instancia socket.
\end{description}

Un esquemático de esta prueba puede apreciarse en la figura \ref{fig:casoPruebaReuseport}. La figura \ref{fig:resultadosReuseport} ilustra los tiempos obtenidos a lo largo del nuevo caso de estudio evaluado sobre el servidor de 24 cores virtuales descrito en el punto \ref{24Cores}. Sorprendentemente, reuseport consigue tiempos netos gradualmente mejores a medida que se incorporan threads hasta un tope en torno a los 8 sockets donde el rendimiento se estanca para luego empeorar. Otro aspecto interesante es una característica no documentada, y es que según los resultados obtenidos, reuseport garantiza un tiempo de operación mínimo basado sólo en el consumo por socket. Vale decir, el tiempo en que el primer socket completa su cuota (tiempo mínimo) está relacionado a ser directamente proporcional a la cuota de consumo. Luego, como a medida que se incorporan más y más sockets la cuota de consumo se reduce (en medida exponencial en nuestro experimento), y así también el tiempo mínimo se reduce de la misma manera.

\begin{figure}[!h]
	\centering
	\includegraphics[scale=.6]{resultados/reuseport1-crop.pdf}
	\caption{Gráfico con tiempos de consumo de reuseport para el nuevo caso de estudio evaluado en el servidor de 24 cores virtuales.}
	\label{fig:resultadosReuseport}
\end{figure}

\section{Evaluación de Optimizaciones Adicionales}
Los resultados anteriores evidencian un claro incremento en el rendimiento alcanzable por la prueba ya conocida aprovechando múltiples sockets en el consumo. Sin embargo cabe preguntarse si será posible mejorar aún más esos resultados combinando la opción de reuseport con técnicas de consumo concurrente sobre los sockets. Una hipótesis que apoya esta idea se basa en que, al emplear la opción reuseport, los sockets tienen una menor cuota de consumo cada uno, por lo tanto, un acceso concurrente en ese escenario de bajo consumo podría impactar en una reducción de tiempos finales al generar una nula sobrecarga en cada caso.

\subsection{Incorporación de Multithreading}
En la presente sección se validará el rendimiento alcanzable por la opción reuseport usándola en combinación con diferentes estrategias de multithreading como las vistas en el capítulo anterior. Para este caso, se evaluó el mismo caso de estudio anterior considerando además distintas configuraciones de consumo concurrente para cada instancia de socket, probando configuraciones con 1, 2, 4 y 8 threads en cada caso. Se limitó a 8 el máximo de threads a consumir pues en el peor caso, usando 64 sockets compartiendo el mismo puerto, y cada uno con 8 hilos de consumo se llevaría al sistema a un escenario con demasiados procesos en ejecución.

\begin{figure}[h!]
	\centering
	\hspace*{\fill}
	\subfigure[]{
		\includegraphics[width=.47\textwidth]{resultados/reuseport21-crop.pdf}
		\label{fig:reuseport2min}
	}\hfill
	\subfigure[]{
		\includegraphics[width=.47\textwidth]{resultados/reuseport22-crop.pdf}
		\label{fig:reuseport2neto}
	}
	\caption{Graficos del resultado del nuevo caso de estudio incorporando lectura con múltiples hilos. En la leyenda de cada gráfico se indica el número de threads que consumen concurrentemente cada socket con reuseport.}
	\label{fig:resultadosReuseport2}
	\hspace*{\fill}
\end{figure}

Los resultados están ilustrados en el gráfico de la figura \ref{fig:resultadosReuseport2}. El primer gráfico que indica los tiempos mínimos \ref{fig:reuseport2min} evidencia que la incorporación de hilos para lectura concurrente mayoritariamente empeoró el desempeño registrado sin concurrencia, sin embargo, las tendencias se conservan a lo largo de la prueba, dando siempre tiempos mínimos mejores a medida que se emplean más sockets. De esta forma, la clave en los tiempos mínimos parece ir más relacionada a la cantidad de sockets que se empleen en cada caso. En el segundo gráfico que ilustra los tiempos netos de la prueba \ref{fig:reuseport2neto} se identifica como la ejecución con threads perjudica siempre el desempeño de reuseport, degradando los tiempos y no brindando ninguna mejora práctica. Es interesante resaltar también que a partir de valores muy altos, parece ser despreciable el sobrecosto que añade tener más threads en cada socket, comportamiento que apoya el planteamiento inicial de que, a una menor cuota de consumo, el sobrecosto disminuye.

\subsection{Incorporación de Multithreading con Processor Affinity}
Similar al estudio del capítulo anterior en la arista de distribución de carga, en esta sección se evaluó el desempeño de Reuseport en escenarios donde múltiples hilos de consumo acceden al socket, y fuesen distribuidos lógicamente entre los distintos CPU del sistema aprovechando la técnica de \emph{Processor Affinity}. En este caso, se evaluaron 3 configuraciones:

\begin{description}
\item[SOsched] Los hilos son distribuidos por el mismo sistema operativo a través de la oepración de su algoritmo de schedulling. Es la base de comparación para con los demás esquemas de distribución.
\item[All0sched] Los distintos hilos de ejecución se alocan en el mismo core (core 0 por simplicidad). Persigue aprovechar localidad de datos en la ejecución simultanea de los hilos.
\item[Equitativesched] Los hilos se distribuyen equitativamente entre los cores lógicos del sistema, persiguiendo una distribución de carga entre unidades de procesamiento real.
\end{description}

\begin{figure}[h!]
	\centering
	\subfigure[]{
		\centering
		\includegraphics[width=.47\textwidth]{resultados/reuseport31-crop.pdf}
		\label{fig:reuseportsomin}
	}
	\subfigure[]{
		\centering
		\includegraphics[width=.47\textwidth]{resultados/reuseport32-crop.pdf}
		\label{fig:reuseportsoneto}
	}
	\subfigure[]{
		\centering
		\includegraphics[width=.47\textwidth]{resultados/reuseport33-crop.pdf}
		\label{fig:reuseportall0min}
	}
	\subfigure[]{
		\centering
		\includegraphics[width=.47\textwidth]{resultados/reuseport34-crop.pdf}
		\label{fig:reuseportall0neto}
	}
	\subfigure[]{
		\centering
		\includegraphics[width=.47\textwidth]{resultados/reuseport35-crop.pdf}
		\label{fig:reuseporteqmin}
	}
	\subfigure[]{
		\centering
		\includegraphics[width=.47\textwidth]{resultados/reuseport36-crop.pdf}
		\label{fig:reuseporteqneto}
	}
	\caption{Graficos de los resultados del nuevo caso de estudio incorporando lectura con múltiples hilos empleando distintas estratégias de \emph{processor affinity} en los hilos.}
	\label{fig:resultadosReuseport3}
\end{figure}

Los resultados de esta evaluación se encuentran en la figura \ref{fig:resultadosReuseport3}. La primera observación relevante a destacar es que, en general, las tendencias de comportamiento tanto en tiempos mínimos como en tiempos netos, se preserva independiente del mecanismo de distribución de threads que se emplee. Los registros de tiempos mínimos mantienen su tendencia de decrecimiento a medida que se incorporan más sockets y la tendencia de los tiempos netos preserva su decaimiento en torno a los 8 sockets, sin variaciones importantes entre esquemas.

Sin embargo, hay ciertos elementos a destacar que se evidencian en los nuevos esquemas de distribución. Tanto en la configuración \emph{Equitativesched} (Fig. \ref{fig:reuseporteqmin} y \ref{fig:reuseporteqneto}) como en \emph{All0sched} (Fig. \ref{fig:reuseportall0min} y \ref{fig:reuseportall0neto}) ocurre un fenómeno interesante en la dispersión o variación de los valores al incorporar threads. En ambos esquemas las variaciones usando distinta cantidad de threads en configuraciones con bajo número de sockets se presenta un desempeño bastante parejo, a diferencia del esquema \emph{SOsched} (Fig. \ref{fig:reuseportsomin} y \ref{fig:reuseportsoneto}). Una tendencia que se mantiene al aumentar el número de sockets consumiendo datos haciendo las diferencias de tiempo muy pequeñas entre usar uno o muchos threads por socket. Este comportamiento refleja que las técnicas de \emph{processor affinity} tienen buenos resultados en balancear la carga cuando se emplean pocos sockets y la cuota de consumo de cada uno es más alta, permitiendo un mejor balance de tiempos entre los threads que consumen cada socket, pero aun así no consiguen batir los tiempos obtenidos por reuseport. La razón de ello apunta a ser que, antes de incorporar paralelismo usando hilos, se llega a un límite de rendimiento del socket mismo que no puede ser superado aplicando concurrencia de esta manera, y que es precisamente el problema detectado en un principio.

\subsection{Análisis y Discusión de Resultados}
Como se aprecia de los resultados experimentales, la incorporación de multithreading en el consumo de datos desde una estructura socket con soporte para reuseport siempre degrada el rendimiento completo del caso de estudio, independiente de si se emplean esquemas de distribución de carga entre los núcleos de procesamiento efectivo. Resulta interesante que más allá de las estrategias que se adopten para la mejor distribución de carga, las variaciones no permiten justificar una verdadera ganancia o pérdida entre esquemas que se adopten, siendo simplemente sobrecargas con respecto al esquema de distribución propio del sistema operativo.

Este resultado viene a confirmar la conclusión de los estudios del capítulo anterior que terminan por postular que es la misma estructura socket de Linux la que no cuenta con un diseño compatible con técnicas de consumo concurrente, degenerando siempre en escenarios de degradación de performance generalizada en su consumo.

\section{Aspectos Negativos}
El esquema que propone reuseport replantea uno de los principios fundamentales que postula el modelo OSI, que es la relación ''uno a uno'' entre estructuras sockets y direcciones de puertos locales. Como reuseport plantea la compartición de puertos locales se pierde la exclusividad de los mismos para cada socket. Otra característica que trastoca esta alternativa es la modificación en la manera de programación de aplicaciones. Como se mencionó, el uso de esta técnica está condicionada a la habilitación de dicha opción a los sockets previo al momento de su acoplamiento con el puerto local a escuchar. Esta situación significa en la práctica estar conscientes de que toda aplicación que desee hacer uso de esta característica debe modificar secciones profundas de su implementación en el código fuente para habilitarla, siempre y cuando el sistema donde se trabaje disponga de soporte para la misma.

Por otro lado, dado su funcionamiento la opción de reuseport está incorporada como código fuente del kernel de Linux, estando disponible según sus creadores en los kernels con versiones posteriores a la 3.9 (Finales del año 2013), ello no es del todo correcto. Un caso de ello son las distribuciones de Linux basadas en Debian --incluido el popular Ubuntu-- donde se han presentado varios problemas en versiones recientes las cuales, a pesar de contar con kernels 3.9+ y tener la implementación fuente de reuseport, poseen la constante de activación de la característica (el flag \verb=SO_REUSEPORT=) como código comentado y no utilizable\footnote{\url{https://lists.debian.org/debian-kernel/2015/02/msg00260.html}} u otros casos de distribuciones de Linux que siguen la dinámica de actualización basadas en \emph{continuous release} que no incorporan los cambios necesarios para dar soporte a las funcionalidades de reuseport\footnote{\url{https://github.com/circus-tent/circus/issues/699}}, lo cual dificulta el uso de esta opción al no dejarla disponible para el uso de los programadores de aplicaciones o llevando a errores de programas en el peor caso.

Una última dificultad asociada al uso de la opción reuseport se relaciona con su documentación. La documentación asociada a esta característica es muy escasa, reduciéndose prácticamente a la implementación misma disponible en el código fuente del propio kernel de Linux. No existe documentación a nivel de usuario que permita comprender ni mucho menos manipular las características de la opción en sí misma. En esa misma línea, aparece como un problema la implementación de esta característica al ser una opción \emph{hardcodeada} en el kernel mismo de Linux, lo que hace inviable pensar cualquier modificación doméstica pensando en querer aprovechar su funcionamiento con ciertas modificaciones o adaptada a escenarios especiales. Ello pues toda modificación se traduce en modificar, recompilar e instalar todo un nuevo kernel, entendiendo en primer lugar lo complicado de dicha tarea, y en segunda instancia, suscitando el riesgo de posibles inestabilidades en el resultado, algo que en la práctica es inaceptable en entornos de producción. Esto último da a entender que la característica reuseport no está pensada para ser modificable según escenarios específicos de operación.

%\chapter{Propuesta de Solución}
El análisis de las distintas alternativas disponibles para mejorar el rendimiento de la interfaz de red efectuado en el capítulo anterior nos motiva a diseñar e implementar una propuesta de solución que satisfaga ciertos requerimientos puntuales, en pos de ubicarla como una opción viable de uso en entornos como el propuesto.

En el presente capítulo se formalizan los requerimientos mínimos que constituyen nuestra solución ideal. A continuación se desarrolla un modelo para esquematizar el funcionamiento de la solución propuesta y se implementa la misma, para terminar con una evaluación del producto final bajo los mismos entornos que las pruebas desarrolladas a lo largo de la presente investigación.

\section{Formalización de Requerimientos}
Como se ilustró en el capítulo anterior, las distintas soluciones disponibles para mejorar la performance de las estructuras sockets en el kernel de Linux se caracterizan por ser complejas en su funcionamiento ya sea por su implementación sobre aplicaciones actuales como por quebrantar ciertos principios de programación definidos como los sugeridos por el modelo estándar OSI. El objetivo en este punto es caracterizar las especificaciones ideales que debe satisfacer un desarrollo de optimización de la operación de los sockets en el marco de la preservación de caracteristicas deseadas, dadas por dicho modelo.

A raíz de lo anterior, a continuación se especifican las propiedades estructurales que debe contemplar una solución ideal:


\begin{description}
\item[Rendimiento] El requerimiento principal para la solución objetivo constituye el garantizar un buen rendimiento. La solución debe poder brindar tiempos -a lo menos- competitivos con la mejor alternativa evaluada a lo largo de la investigación en curso, que corresponde al rendimiento obtenido con el mecanismo de \emph{ReusePort}.
\item[Bajo Overhead] A fin de lograr una solución con bajo overhead e impacto con el resto del sistema, se debe procurar considerar una alternativa que opere en los niveles más bajos del sistema operativo -A nivel de Kernel idealmente- a fin de evitar la sobrecarga efectuada por concepto de interrupciones y call-chains ???? que son propias de soluciones que operan en espacio de usuario.
\item[Modularidad] El esquema ideal debe ser modular en el sentido de garantizar una sencilla instalación y remoción de un sistema, sin necesitar significativas dependencias de otros componentes. En otra arista de este mismo requerimiento, se necesita una solución que permita modificarciones de manera sencilla, a fin de brindar extensibilidad de la misma.
\item[Adaptable] Finalmente, una propiedad que debe contemplar la solución es ser configurable y adaptable a los distintos entornos y requerimientos que se adapten correctamente a las necesidades de rendimiento que se persigan.
\end{description}

Las características antes mencionadas sirven como requerimientos estructurales en lo que se postula como nuestra propuesta de diseño de solución.

\section{Modelo de Funcionamiento de la Solución}
Tomando en consideración los requerimientos descritos en la sección anterior, se optó por una solución cuyo modelo de funcionamiento fuese como el descrito en la imagen \ref{modeloUDPRedistribuyeModule} denominado como \textbf{UDPRedistributeModule}. Consiste en un \emph{módulo del Kernel} que permite interceptar y redistribuir paquetes (de las características relacionadas al caso de estudio) empleando distintos esquemas de distribución para la redirección de los mismos. Para la versión implementada se desarrollaron 2 esquemas de distribución que se describen en apartados posteriores.

IMAGEN

La justificación de esta desición de diseño se fundamenta en que, con éste enfoque se satisfacen casi todos los requerimientos definidos en la sección anterior: En primer lugar, es un enfoque que añade \textbf{bajo overhead} al sistema, al ser un módulo que se ejecuta en el espacio del Kernel, por lo que cuenta con privilegios que evitan la sobrecarga experimentada por soluciones en espacio usuario. Segundo, siendo un módulo, al momento de instalarlo en el sistema se pueden realizar todas las \textbf{configuraciones} pertinentes, permitiendo un grado de adaptabilidad de acuerdo al entorno y resultado deseado. En tercer lugar, al ser un módulo, es facilmente \textbf{modificable} permitiendo añadir, modificar o eliminar del mismo, reglas definidas o esquemas de distribución. Basta modificarlo, recompilarlo y está listo para operar. Por lo mismo, es facilmente removible del sistema y no guarda ningúna dependencia estricta de terceros.

Las componentes que rigen el modelo en su operación ilustrados en la imagen \ref{modeloUDPRedistribuyeModule} se describen como:

\begin{description}
\item[hook\_port] Valor que determina el puerto a interceptar para la redirección de paquetes.
\item[redirect\_port] Colección de valores que indican los puertos hacia los cuales redistribuir los paquetes interceptados y modificados.
\item[verbose] Índice del grado de detalle con que se guardarán registros de acción del módulo en los mensajes del Kernel.
\end{description}


En la práctica, la operación de la solución propuesta se inspira en el mecanismo de un \emph{proxy} de redistribución de paquetes, modificando valores de las cabeceras de los paquetes intervenidos. El modelo implementa la siguiente lógica de instrucciones:

\begin{enumerate}
\item Interceptar paquetes de tipo UDP y que estén dirigidos a un determinado puerto.
\item Modificar el paquete, actualizando valores como el \emph{checksum} del mismo.
\item Redireccionar el paquete, modificando el puerto de destino de acuerdo al esquema (o algoritmo) de distribución seleccionado.
\item Reincorporación del paquete en el tránsito de distribución del kernel, por la vía ordinaria.
\end{enumerate}

Por su naturaleza de acción, la solución opera estrictamente entre las capas de Red y de Transporte en el modelo OSI, interviniendo esos niveles de abstracción, a través de la modificación de los headers correspondientes al empaquetamiento IP y UDP respectivamente.

\subsection{Esquemas de Distribución}
Como ya se mencionó, la solución desarrollada comprende una etapa de distribución de paquetes de acuerdo a distintos esquemas de distribución, que son reglas para reasignar los paquetes interceptados entre un pool de puertos previamente definidos. En la versión desarrollada del \textbf{UDPRedistributeModule} se implementaron 2 esquemas de distribución: \emph{RandomSched} y \emph{SequentialSched}.

\subsubsection{RandomSched}
El esquema de distribución \emph{RandomSched} realiza una distribución entre puertos de manera aleatoria. La aleatoriedad de éste esquema se consigue por medio de la llamada de sistema \verb=get_random_bytes()= que sugiere una distribución aleatorizada de resultados y redistribuyendo los paquetes de acuerdo a la aplicación de un módulo entre el total de puertos blanco de redistribución del módulo. La utilización de dicha función se justifica también en ser uno de los mecanismos más sencillos de obtener aleatoriedad en demanda en el espacio del Kernel.

\begin{figure}[th!]
\centering
\subfigure[text1]{
	\includegraphics[width=.3\textwidth]{imagenes/fcfm}
}
\subfigure[text2]{
	\includegraphics[width=.3\textwidth]{imagenes/fcfm}
}
\subfigure[text3]{
	\includegraphics[width=.3\textwidth]{imagenes/fcfm}
}
\caption{Evolución en la distribución de paquetes usando el esquema aleatorio por \emph{RandomSched}.}
\label{fig:RandomSched}
\end{figure}

A pesar de que el esquema en promedio promete una distribución uniforme entre los puertos, en la práctica es posible que la carga no sea en efecto perfectamente distribuida, por lo que éste esquema es práctico en escenarios donde se desee cierta entropía en la reasaignación de carga entre puertos de destino.

\subsubsection{SequentialSched}
El segundo esquema de distribución denominado \emph{SequentialSched} hace una distribución secuencial entre los distintos puertos de destino de distribución asignados en el módulo. De ésta manera, éste esquema consigue una perfecta distribución entre puertos de redirección logrando una carga equitativa entre todos ellos.

\begin{figure}[th!]
\centering
\subfigure[text1]{
	\includegraphics[width=.3\textwidth]{imagenes/fcfm}
}
\subfigure[text2]{
	\includegraphics[width=.3\textwidth]{imagenes/fcfm}
}
\subfigure[text3]{
	\includegraphics[width=.3\textwidth]{imagenes/fcfm}
}
\caption{Evolución en la distribución de paquetes usando el esquema secuencial por \emph{SequentialSched}.}
\label{fig:SequentialSched}
\end{figure}

\section{Implementación}
Al ser un módulo del kernel, existen diversas guias técnicas para respaldar la construcción del mismo \cite{algunawea}. Para poder implementar las operaciones de interferencia de paquetes en módulo, se utilizó el framework de Linux \emph{Hook NetFilters}.

\subsection{NetFilters Framework}
Aka, una descripción de que son los hookNetfilters

\

\section{Instalación y Utilización}
Manteniendo la dinámica de un módulo de kernel, la solución requiere ser primero compilada y luego instalada en el sistema de acuerdo a los comandos tradicionales para dicho proposito de que dispone un sistema Linux.

\subsection{Compilación}
Como con todo módulo, se requiere disponer tanto de los encabezados o cabeceras del Kernel, además de su código fuente. Las distribuciones actuales incorporan mecanismos muy simples para poder obtener esos recursos sin mayores problemas.

\begin{listing}
\caption{main}
\begin{minted}[frame=lines,bgcolor=LightGray]{bash}
 >>> a = 1 
 >>> b = 2
 >>> a b +
\end{minted}
\end{listing}


\subsection{Instalación y Configuración}
\subsection{Utilización}

\section{Rendimiento en la Práctica}

\begin{defn}[ver \cite{KAR00}] Definición definitiva $$\frac{d}{dx}\int_a^xf(y)dy=f(x).$$\end{defn}

\begin{teo}[ver \cite{KAR00}] Definición definitiva $$\frac{d}{dx}\int_a^xf(y)dy=f(x).$$\end{teo}

\begin{prop}[ver \cite{KAR00}] Definición definitiva $$\frac{d}{dx}\int_a^xf(y)dy=f(x).$$\end{prop}

\begin{obs}[ver \cite{KAR00}] Definición definitiva $$\frac{d}{dx}\int_a^xf(y)dy=f(x).$$\end{obs}

\begin{ej}[ver \cite{KAR00}] Definición definitiva $$\frac{d}{dx}\int_a^xf(y)dy=f(x).$$\end{ej}
\chapter{Propuesta de Solución}
El análisis de la opción \emph{reuseport} efectuado en el capítulo anterior reveló que es posible obtener mejores tiempos en el consumo de datos desde una interfaz de red con respecto al consumo tradicional. Ahora bien, los aspectos negativos reconocidos en el uso de esta opción nos motivan a diseñar e implementar una propuesta de solución que satisfaga ciertos requerimientos puntuales, en pos de ubicarla como una opción viable --y preferible por sobre reuseport-- en un uso en entornos como el propuesto.

En el presente capítulo se formalizan los requerimientos mínimos que constituyen nuestra solución ideal, los cuales servirán como lineamientos mínimos a satisfacer por nuestra solución. Con dicho marco de requisitos, se construye un modelo para esquematizar el funcionamiento de la solución propuesta. Posteriormente se trabaja en la especificación de la implementación de la solución, para terminar con una evaluación final de performance bajo los mismos entornos que las pruebas desarrolladas a lo largo de la investigación y concluir determinando si existen beneficios reales asociados a nuestra propuesta con respecto a otras alternativas.

\section{Formalización de Requerimientos}
Como se expuso en el capítulo anterior, la alternativa de \emph{reuseport}, desarrollada para mejorar la performance de las estructuras sockets en el kernel de Linux se caracteriza por ser complejas en su funcionamiento ya sea por su implementación para funcionar sobre aplicaciones ya desplegadas en un sistema, o por quebrantar ciertos principios de programación definidos --como los sugeridos por el modelo estándar OSI-- con consecuencias como tener que modificar el código de programas ya operativos bajo el modelo tradicional. Existen otras alternativas para emular un comportamiento similar a la técnica de reuseport como lo es la utilidad de \textbf{IPTables} \cite{book:iptables} la cual se caracteriza por su dificultad de uso y configuración, haciéndola difícil de sugerir en entornos delicados de producción. El objetivo de la presente sección es caracterizar las especificaciones ideales que debe satisfacer un desarrollo enfocado en brindar una optimización de la operación de los sockets de Linux, en el marco de la preservación de características deseadas especificadas por el ya mencionado modelo OSI.

En función de lo anterior, a continuación se especifican las propiedades estructurales que debe contemplar una solución ideal:

\begin{description}
\item[Rendimiento] El requerimiento principal para la solución objetivo constituye el garantizar un buen rendimiento. La solución debe poder brindar mejores tiempos de operación que el enfoque tradicional de consumo de datos desde un socket, y han de ser tiempos --a lo menos-- competitivos con la mejor alternativa evaluada a lo largo de la investigación en curso, que corresponde al rendimiento obtenido con el mecanismo de \emph{Reuseport}.
\item[Bajo Overhead] A fin de lograr una solución con bajo  nivel de sobrecarga e impacto con el resto del sistema se debe procurar considerar una alternativa que opere en los niveles más bajos del sistema operativo, a fin de evitar la sobrecarga efectuada por concepto de interrupciones o intervenciones que son propias de soluciones que operan en espacio de usuario.
\item[Modularidad] El esquema ideal debe ser modular, en el sentido de garantizar una sencilla instalación y remoción de la misma en un sistema sin necesitar significativas dependencias para con otros componentes a modo de garantizar amplia compatibilidad entre sistemas. En otra arista de este mismo requerimiento, se busca desarrollar una solución que permita modificaciones en sí misma de manera sencilla a fin de garantizar \textbf{extensibilidad}.
\item[Adaptabilidad] Finalmente, una propiedad que debe contemplar la solución es ser fácilmente configurable y adaptable a los distintos entornos y requerimientos que se presenten en línea con las necesidades de rendimiento que se persigan.
\end{description}

Las características antes mencionadas se utilizaron como requerimientos estructurales en lo que se postula como nuestra propuesta de diseño de solución.

\section{Modelo de Funcionamiento de la Solución}
Tomando en consideración los requerimientos descritos en la sección anterior, se optó por una solución cuyo modelo de funcionamiento fuese como el descrito en la figura \ref{fig:modeloUDPRedistribuyeModule}, construido como un \textbf{módulo del kernel} capaz de intervenir paquetes de Internet de las características que atañen a este estudio, y los redistribuyan entre un pool de nuevos puertos para luego ser atendidos por interfaces sockets exclusivas en cada caso. Para una primera versión de esta solución se postularon 2 esquemas de distribución que se describen en apartados posteriores.

	\begin{figure}[!h]
		\centering
		\includegraphics[scale=.55]{imagenes/udpredistributemoduleDiagram.png}
		\caption{Diagrama esquemático de la operación de la solución propuesta.}
		\label{fig:modeloUDPRedistribuyeModule}
	\end{figure}

Esta decisión de diseño se fundamenta en que con este enfoque se satisfacen prácticamente todos los requerimientos definidos en la sección anterior: En primer lugar es un enfoque que añade \textbf{bajo overhead} al sistema. Al ser un módulo, la solución se ejecuta en el espacio del kernel por lo que cuenta con privilegios que evitan la sobrecarga experimentada por soluciones en espacio usuario, además de tener un bajo impacto para otras aplicaciones por no intervenirlas en su operación. Segundo, siendo un módulo, al momento de instalarlo en el sistema se pueden especificar distintas configuraciones pertinentes al modo de funcionamiento deseado, permitiendo un amplio grado de \textbf{adaptabilidad} de acuerdo al entorno y resultado deseado. En tercer lugar, al ser un módulo es fácilmente \textbf{modificable} permitiendo añadir, modificar o eliminar del mismo toda la lógica de programación implementada en su interior. Para ello sólo basta modificarlo, recompilarlo e instalarlo nuevamente. De la mano con lo anterior, es fácilmente removible del sistema.

Las componentes que rigen el modelo en su operación ilustrados en la figura \ref{fig:modeloUDPRedistribuyeModule} se describen acuerdo a las siguientes definiciones:


\begin{description}
\item[hook\_port] Valor que determina el puerto a interceptar para la redirección de paquetes.
\item[redirect\_port] Colección de valores que indican los puertos hacia los cuales redistribuir los paquetes interceptados y modificados.
\item[verbose] Índice del grado de detalle con que se guardarán registros de acción del módulo en los mensajes del kernel.
\end{description}

En la práctica, la operación de la solución propuesta se inspira en el mecanismo de un \emph{proxy}, modificando solamente los valores en las cabeceras de los paquetes intervenidos. El modelo implementa la siguiente lógica de instrucciones:

\begin{enumerate}
\item Interceptar paquetes de tipo UDP y que estén dirigidos a un determinado puerto (\textbf{hook\_port}).
\item Modificar el paquete, actualizando valores como el \emph{checksum} del mismo.
\item Redireccionar el paquete, modificando el puerto de destino del paquete (\textbf{redirect\_port}) de acuerdo al esquema de distribución operativo.
\item Reincorporación del paquete modificado en el tránsito de distribución del kernel por la vía ordinaria.
\end{enumerate}

Por su naturaleza de acción, la solución opera estrictamente entre las capas de Red y de Transporte en el modelo OSI, interviniendo esos niveles de abstracción a través de la modificación de los encabezados correspondientes al empaquetamiento por los protocolos IP y UDP respectivamente.


\subsection{Esquemas de Distribución}
Como ya se mencionó, la solución desarrollada comprende una etapa de reparto de paquetes de acuerdo a distintos esquemas de distribución. En este contexto, un esquema de distribución se define como un conjunto de reglas para reasignar los paquetes interceptados entre un pool de puertos previamente definidos. En la versión desarrollada de la solución se implementaron 2 esquemas de distribución: \emph{RandomSched} y \emph{SequentialSched}.

\subsubsection{RandomSched}
El esquema de distribución \emph{RandomSched} realiza una distribución entre puertos de manera aleatoria. La aleatoriedad de este esquema se consigue por medio de la llamada de sistema \verb=get_random_bytes()= que sugiere una entrega aleatoria de resultados en valores de bytes. Dicho resultado se emplea para la redistribución de los paquetes de acuerdo a la aplicación de una operación de módulo sobre el valor aleatorio obtenido, por la cantidad total de puertos objetivos de redistribución. El valor resultante indica el índice del puerto (entre los puertos de redirección) a utilizar. La utilización de dicha función se justifica en ser uno de los mecanismos más sencillos de obtener aleatoriedad en demanda en el espacio del kernel y que, según su documentación, sus valores generados siguen una distribución uniforme.


\begin{figure}[th!]
\centering
\subfigure[]{
	\includegraphics[width=.3\textwidth]{imagenes/rand1.png}
}
\subfigure[]{
	\includegraphics[width=.3\textwidth]{imagenes/rand2.png}
}
\subfigure[]{
	\includegraphics[width=.3\textwidth]{imagenes/rand3.png}
}
\caption{Evolución en la distribución de paquetes usando el esquema aleatorio por \emph{RandomSched}.}
\label{fig:RandomSched}
\end{figure}

A pesar de que en teoría este esquema en promedio promete una distribución uniforme, en la práctica es posible que la carga no sea en efecto perfectamente distribuida por lo que este esquema es práctico en escenarios donde se permita cierta entropía en la reasignación de carga entre puertos de destino.

\subsubsection{SequentialSched}
El segundo esquema de distribución diseñado es denominado \emph{SequentialSched}. En este caso, se hace una distribución secuencial entre los distintos puertos de destino de distribución asignados en el módulo. De esta manera, este esquema consigue una perfecta distribución entre puertos de redirección logrando una carga equitativa entre todos ellos.

\begin{figure}[th!]
\centering
\subfigure[]{
	\includegraphics[width=.3\textwidth]{imagenes/seq1.png}
}
\subfigure[]{
	\includegraphics[width=.3\textwidth]{imagenes/seq2.png}
}
\subfigure[]{
	\includegraphics[width=.3\textwidth]{imagenes/seq3.png}
}
\caption{Evolución en la distribución de paquetes usando el esquema secuencial por \emph{SequentialSched}.}
\label{fig:SequentialSched}
\end{figure}

\section{Implementación}
El mecanismo de funcionamiento planteado en la sección anterior postula modificar el destino final de los paquetes, a través de la modificación de los encabezados de los paquetes interceptados cambiando el valor del puerto de destino de los mismos. Dicha operación implica dos características a considerar:

\begin{enumerate}
\item Interceptar paquetes en el momento apropiado. A priori dicho momento sería idealmente apenas el sistema reconozca el arribo de un paquete y sin intervención de otros mecanismos del sistema.
\item Modificar paquetes para su redirección. La información de puerto de destino es una especificación propia de la capa de transporte según el modelo OSI por lo que dicha información es inherente a los encabezados de dicha capa.
\end{enumerate}

Dados los requerimientos anteriores, para la implementación de la solución se optó por el diseño de un módulo del kernel operacional sobre el framework de \emph{Netfilters} del kernel, el cual permite intervenir paquetes de las características relacionadas al caso de estudio y brinda la flexibilidad para modificar los encabezados de los mismos, logrando el efecto de redistribución entre distintos puertos, y siendo lo suficientemente flexible como para emplear distintos esquemas de distribución para la redirección de los mismos. Además, al ser un módulo del kernel, existen diversas guías técnicas que profundizan los aspectos de diseño a considerar en construcción del mismo y al ser parte del módulo de red del kernel mismo, apoya el requerimiento de \textbf{adaptabilidad} solicitado en la sección de requisitos de la solución.

\subsection{NetFilter Framework}
Netfilter \cite{report:netfilterModule} es un framework disponible en el núcleo de sistemas Linux que permite la manipulación de paquetes en distintos niveles del tránsito a través del \emph{stack} de protocolos de red del kernel. Parte del módulo de red del mismo kernel de Linux, Netfilters provee una interfaz para implementar los denominados \emph{hooks}, funciones de intervención de paquetes donde se da la libertad de programar libremente operaciones sobre cada estructura de paquete intervenido. Los \emph{hooks} implementados deben ser registrados en el sistema en un \emph{punto de intercepción} y es el mismo kernel el encargado de incluirlos en su rutina de inspección de cada paquetes a medida que los mismos van llegando al sistema, aplicándolos de acuerdo a reglas de prioridad y puntos de intervención bien definidos. Netfilters es una de las herramientas más potentes para la manipulación de paquetes, siendo la base de distintas herramientas que operan en espacio usuario como por ejemplo IPTables.

Las capacidades de manipulación que provee Netfilters son sumamente potentes abarcando un amplio espacio de acción en el sistema, postulándose así como una interesante herramienta que contemplar en el desarrollo de la solución al desafío planteado.

\subsubsection{Arquitectura de Interrupción en Netfilters}
El framework provee una arquitectura de seguimiento de los distintos paquetes para interceptar que se modela en un esquema definido en la figura \ref{fig:netfilterArchitecture}. Esta arquitectura es lo suficientemente amplia como para intervenir paquetes en distintos niveles de comunicación y tránsito a través del kernel.

\begin{figure}[!h]
	\centering
	\includegraphics[scale=.65]{imagenes/netfilterArchitecture}
	\caption{Esquema de la arquitectura de interrupción disponible para la intervención de paquetes provisto por el Netfilters Framework.}
	\label{fig:netfilterArchitecture}
\end{figure}

El esquema ilustrado en la figura \ref{fig:netfilterArchitecture} da cuenta del flujo que recorre un paquete en el kernel en consideración de los modos de intervención que brinda el framework de Netfilters. Cada uno de los puntos ilustrados en dicho diagrama es un potencial punto de intervención, y da cuenta de alguna etapa en el arribo o salida de un paquete. Los distintos puntos de intervención disponibles en \emph{IPv4} que soporta Netfilters se detallan a continuación:

\begin{description}
\item[NF\_IP\_PRE\_ROUTING] Todos los paquetes entrantes pasan a través de este hook, al momento de ejecutarse la llamada a sistema \verb=ip_rcv()=, antes de cualquier proceso de ruteo.
\item[NF\_IP\_INPUT] Todo paquetes destinado al equipo local pasa por este hook, al ejecutarse la llamada \verb=ip_local_deliver()=.
\item[NF\_IP\_FORWARD] Todo paquete entrante que esté en tránsito a su destino final (es decir, que su destino final no sea el equipo local y sólo deba ser redirigido) pasa por este hook, en la llamada a sistema \verb=ip_forward()=.
\item[NF\_IP\_OUTPUT] Todo paquete saliente creado en el equipo local, pasa por este hook, en la llamada a sistema \verb=ip_build_and_send_pkt()=.
\item[NF\_IP\_POST\_ROUTING] Todo paquete saliente, haya sido creado en este equipo o solo redirigido por él, pasa por este hook, en la llamada a sistema \verb=ip_finish_output()=.
\end{description}

\subsubsection{Implementación de funciones de Interrupción usando Netfilters}
El framework en cuestión está diseñado para poder aprovechar sus funcionalidades fácilmente a partir de un módulo adicionado al kernel. Para ello, basta incluir las directivas de encabezados del framework en el módulo implementado (\verb=#include <linux/netfilter.h>=) e implementar los mecanismos de registro del \emph{hook} en el sistema, especificando sus opciones como: \textbf{prioridad}, \textbf{familia} y \textbf{punto de interrupción}. Para esta tarea, se emplean valores constantes provistos por el mismo framework. Además de lo anterior, en el módulo se ha de implementar la función a aplicar sobre cada paquete intervenido, para lo cual es necesario seguir la convención que especifica el mismo framework, esto es, implementar un prototipo de función que propone el framework que se ilustra en el segmento de código \ref{code:netfilter}.

\vspace{1pc}
\begin{lstlisting}[style=CInputStyle, label=code:netfilter, captionpos=b, caption={Prototipo de la función de interrupción a ser definida en un \emph{hook} de Netfilters.}]
static unsigned int hook_func(
            		unsigned int hooknum,
            		struct sk_buff *skb, 
            		const struct net_device *in, 
            		const struct net_device *out, 
            		int (*okfn)(struct sk_buff *));
\end{lstlisting}

Dicho prototipo de función comprende 5 valores a recibir que contienen información relacionada al paquete que se ha de intervenir: \textbf{hooknum} que corresponde al valor del mismo nombre que se asigna al momento de registrar la función de \emph{hook} y que hace referencia al punto de interrupción en que inspeccionará paquetes nuestro \emph{hook}. \textbf{skb} que es un puntero a una estructura tipo \verb=sk_buff= que almacena el paquete interceptado propiamente tal, y del cual se pueden extraer y modificar los encabezados de las capas de transporte y red para modificar el tránsito del paquete. Los punteros \textbf{in} y \textbf{out} de estructuras \verb=net_device= que apuntan a las interfaces de los dispositivos de red de procedencia y destino del paquete, registradas al momento de su interceptación, y \textbf{*okfn} (\emph{okay function}) que corresponde a una función que es invocada cuando todas las funciones registradas con este hook retornan \verb=NF_ACCEPT=, significando que el paquete sigue en tránsito por el sistema.

Finalmente, las funciones de \emph{hook} deben especificar un valor de retorno que determina el camino a seguir por el paquete intervenido. En éste sentido, el framework especifica 5 posibles valores de retorno para las funciones de interrupción especificadas a continuación:
\begin{description}
\item[NF\_ACCEPT:] Para que el paquete prosiga su entrega, es decir, promueve el paquete a la siguiente etapa de distribución según módulo de red del kernel.
\item[NF\_DROP:] Para descartar el paquete, cancelando su entrega.
\item[NF\_STOLEN:] Para adueñarse del paquete. Es decir, sacar el paquete del entorno de Netfilters y brindarle su propiedad a la misma función de \emph{hook}.
\item[NF\_QUEUE:] Para encolar el paquete y poder manipularlo en espacio usuario.
\item[NF\_REPEAT:] Para llamar a la función de \emph{hook} en cuestión, otra vez.
\end{description}

\subsection{Implementación de la solución usando Netfilters}
Para el proceso de registro del \emph{hook} se tomaron ciertas consideraciones que permitieran garantizar la plena operatividad de la solución en el entorno de trabajo.

En primer lugar, se seleccionó como punto de interrupción del \emph{hook} a \verb=NF_INET_PRE_ROUTING=, ello pues corresponde al nivel más temprano de interrupción desde el arribo del paquete por la interfaz del sistema que admite el framework y básicamente captura todo tránsito de paquetes. Por la naturaleza de paquetes a trabajar se configuró la familia del \emph{hook} como \verb=PF_INET= que abarca a la familia de los protocolos de internet. Para el apartado de prioridad del \emph{hook} se seleccionó como el valor constante \verb=NF_IP_PRI_FIRST= provisto por el framework mismo y que garantiza la más pronta aplicación posible de nuestro \emph{hook} a los paquetes interceptados. Esta decisión se apoya en que se desea el menor impacto para con otras componentes del sistema, por ello la intervención debe ser temprana y poco invasiva. Finalmente, se implementó el cuerpo de la función para tratar cada paquete interceptado (\verb=hook_func=) en la cual se realiza una inspección para reconocer los paquetes de interés en nuestro caso de estudio (paquetes de tipo UDP dirigidos al puerto configurado a interceptar) para luego modificar los encabezados y permitir al paquete su tránsito en el sistema.

\subsection{Código Fuente}
La solución\footnote{\url{https://github.com/sebablasko/UDPRedistributeModule}} fue elaborada siguiendo los principios de construcción de un módulo estándar de Linux \cite{report:netfilterModule}. Se implementaron los métodos \verb=init_module= y \verb=cleanup_module= para la instalación y eliminación del módulo respectivamente, es en el primer método donde se gatilla la construcción y el registro del \emph{hook} definido para la interrupción de los paquetes. Para los procesos de registro y eliminación del \emph{hook} del sistema se usan respectivamente las llamadas \verb=nf_register_hook= y \verb=nf_unregister_hook= provistas por el framework de Netfilters.

\section{Instalación y Utilización}
Preservando la dinámica de un módulo de kernel, la solución requiere ser primero compilada y luego instalada en el sistema de acuerdo a los comandos tradicionales para dicho propósito de que dispone un sistema Linux.

\subsection{Compilación}
Para el proceso de compilación se requiere disponer tanto de los encabezados del kernel activo en la máquina a instalar, además de su código fuente. La gran mayoría de las distribuciones modernas de Linux incorporan mecanismos muy simples para poder obtener esos recursos sin mayores problemas.

La solución desarrollada incorpora un archivo \verb=Makefile= que resume las instrucciones para la compilación y correcta limpieza del mismo, utilizable en la mayoría de los escenarios sin necesidad de ningún cambio.

\subsection{Instalación y Configuración}
Una vez compilado, para la instalación en el sistema se deben emplear los comandos tradicionales de Linux para agregar módulos al kernel, específicamente \verb=insmod= para instalar. Es importante recordar que ésta es una operación que requiere privilegios en el sistema por lo que debe ser ejecutada en modo administrador.

Al momento de realizar la instalación se debe proveer la configuración que regirá la operación de la solución. La configuración consta de la incorporación de las siguientes opciones:

\begin{description}
\item[verbose] Para seleccionar un nivel de detalle en los mensajes que se dejarán en el registro del kernel sobre la actividad del módulo. Los niveles disponibles de verbosidad son:
\begin{description}
\item[verbose=0] Sólo registra la configuración de instalación y de operación del módulo.
\item[verbose=1] Registra lo mismo que el nivel de verbosidad 0, además de registrar para cada paquete correctamente interceptado detalles de la nueva dirección de puerto de destino que se ha asignado.
\item[verbose=2] Registra lo mismo que los niveles 0 y 1, además de informar para cada paquete capturado detalles de sus construcción como direcciones \verb=IP= y puertos de origen y destino, largo de encabezados y todos los pasos de modificación del paquete.
\end{description}
\item[hook\_port] Para configurar el puerto desde el cual se interceptarán los paquetes. Es decir, todo paquete que venga dirigido a éste puerto, pasará por el proceso de redistribución del módulo.
\item[start\_redirect\_port] Para configurar el puerto inicial que servirá para redirigir los paquetes interceptados.
\item[number\_redirect\_ports] Para determinar cuántos serán los puertos de redirección que se emplearán en el esquema de distribución del módulo.
\item[port\_sched] Para seleccionar el esquema de distribución a utilizar en la reasignación de puertos por paquete. Las opciones en éste caso son: 1 para \textbf{RandomSched} y 2 para \textbf{SequentialSched}.
\end{description}

En la práctica, los puertos de redirección quedan determinados a partir del valor de \verb=start_redirect_port= y hasta \verb=start_redirect_port + number_redirect_ports=.

\vspace{1pc}
\begin{lstlisting}[style=BashInputStyle,	breaklines=true, caption=Ejemplo de instalación de la solución, captionpos=b]
    # sudo insmod UDPRedistributeModule.ko verbose=2 hook_port=13131 start_redirect_port=1820 number_redirect_ports=4 port_sched={1,2}
\end{lstlisting}


\subsection{Utilización}
Una vez instalada, la solución queda plenamente operativa en el sistema llevando un registro de sus operaciones dependiendo del nivel de verbosidad definido al instante de la instalación. Es importante resaltar que es responsabilidad del usuario tomar posesión del conjunto de puertos de redirección definido en la solución, usándolos para los fines que él mismo desee.

\subsection{Eliminación}
La remoción de la solución desde el sistema es muy simple y se basa en que la misma es sólo un módulo del kernel, por lo tanto basta aplicar el comando \verb=rmmod= del sistema acompañado del nombre del módulo. Para una completa limpieza del módulo, es recomendable también limpiar los registros generados al momento de la compilación del módulo, para lo cual se puede ejecutar la directiva \verb=make clean= del archivo \verb=Makefile= provisto en la solución.

\section{Evaluación de Rendimiento de la Solución}

A fin de poder estudiar el funcionamiento y determinar el rendimiento de la solución propuesta, se construyó un experimento siguiendo la misma dinámica con que se valuaron los experimentos anteriores (Tanto el caso de estudio original como el nuevo caso de estudio de \emph{reuseport}). De esta manera, se construyó un nuevo caso de estudio (Ver figura \ref{fig:5layersqpi}) basado en el mismo experimento de \emph{reuseport} que consiste en el cálculo de tiempo promedio de consumo de un total de un millón de paquetes distribuidos en una cantidad de sockets pre configurada según la instalación de nuestra solución planteada \footnote{\url{ https://github.com/sebablasko/Test_SaturationReusePortUDPSockets}}. Al igual que en el caso anterior, el experimento contempla los conceptos de \textbf{tiempo mínimo}, \textbf{tiempo neto} y \textbf{cuota de consumo}. Finalmente, y al igual que en los experimentos anteriores, se evaluó un total de 60 repeticiones de éste experimento para validez estadístico de los resultados y para tener un resultado comparable con el obtenido en el estudio de \emph{reuseport}, el experimento se desarrolló en el servidor de 24 cores virtuales descrito en el punto \ref{24Cores}.

\begin{figure}[!h]
	\centering
	\includegraphics[scale=.5]{imagenes/adaptacioncasoestudio.png}
	\caption{Adaptación del esquema de la prueba UDP en un sistema con la solución desarrollada habilitada, correspondiente al nuevo caso de estudio.}
	\label{fig:casoPruebaModulo}
\end{figure}

Al igual que en el experimento con \emph{reuseport} se formalizaron los siguientes conceptos para cuantificar los tiempos de las distintas operaciones involucradas en la prueba:

\begin{description}
\item[Tiempo Mínimo] Corresponde al tiempo en que el primer socket termina de consumir su cuota de consumo asignada.
\item[Tiempo Neto] Corresponde al tiempo en que el último socket termina de consumir su cuota de consumo asignada. De esta forma, corresponde al tiempo en que se completa la trasferencia total de datos.
\item[Cuota de Consumo] Corresponde a la porción de paquetes que le corresponde consumir a cada socket y depende directamente del esquema de distribución de paquetes que se esté empleando.
\end{description}

\subsection{Resultados}
A continuación se presentan los resultados de los tiempos registrados en el experimento correspondiente al nuevo caso de estudio.

\begin{figure}[h!]
	\centering
	\hspace*{\fill}
	\subfigure[]{
		\includegraphics[width=.47\textwidth]{resultados/modulomin-crop.pdf}
		\label{fig:mimodulomin}
	}\hfill
	\subfigure[]{
		\includegraphics[width=.47\textwidth]{resultados/modulomax-crop.pdf}
		\label{fig:mimoduloneto}
	}
	\caption{Gráfico con tiempos de consumo del módulo solución \textbf{UDPRedistributeModule} para el nuevo caso de estudio.}
	\label{fig:resultadoMiModuloMinMax}
	\hspace*{\fill}
\end{figure}

En primera instancia, la solución consigue tanto en sus tiempos mínimos como netos una ganancia en el tiempo de la prueba usando múltiples sockets con respecto al caso de uso de un solo socket, algo nunca conseguido usando el enfoque de consumo concurrente sobre un socket usando múltiples hilos. Otro aspecto interesante es que el rendimiento de la solución usando el mecanismo de distribución \emph{SequentialSched} es más eficiente que el correspondiente con \emph{RandomSched}, situación comprensible entendiendo que por su naturaleza, el esquema secuencial satisface la cuota de los sockets alimentados sin desperdiciar ningún paquete, esto producto de tener una distribución totalmente justa entre sockets. Diferencia crucial con respecto al esquema de distribución aleatorio donde se puede dar el caso de que los sockets no reciban una cuota idéntica, pudiendo así cumplir algunos su cuota primero y causando la perdida paquetes al redirigidos a sockets satisfechos en la prueba.

Por otro lado, los tiempos netos de transferencia registran una variación significativa usando el esquema \emph{SequentialSched} a medida que se usan más sockets, lo cual sugiere un comportamiento posiblemente competitivo con \emph{reuseport}. Sin embargo, para evaluar el verdadero desempeño de la solución, es preciso comparar su rendimiento en detalle con respecto a las distintas alternativas disponibles.

\section{Comparación de Rendimiento}

Para evaluar el verdadero despeño de la solución propuesta, se evaluaron los distintos enfoques de consumo de datos desde un socket estudiados a lo largo de esta investigación. Tal y como se adelantó, los resultados de la figura \ref{fig:resultadoMiModulovsTodos} ilustran como la solución (en cualquiera de sus dos mecanismos de distribución) al igual que reuseport, genera ganancias de tiempo para completar el consumo del millón de paquetes a medida que se emplean más sockets con respecto al escenario original del uso multi-hilo para consumo concurrente. De la misma forma, se aprecia cómo el mejor desempeño es dado generalmente (salvo excepciones como el escenario de 2 sockets) por la alternativa reuseport, seguido por nuestra solución en esquema secuencial, y posteriormente el esquema aleatorio, con una diferencia entre tiempos que se amortiza a medida que se van adicionando más sockets.

\begin{figure}[!h]
	\centering
	\includegraphics[scale=.6]{resultados/modulovstodos1-crop.pdf}
	\caption{Comparación de tiempos netos de consumo para el nuevo caso de estudio de todas las opciones de consumo estudiadas. Experimento evaluado sobre el servidor de 24 cores virtuales descrito en el punto \ref{24Cores}.}
	\label{fig:resultadoMiModulovsTodos}
\end{figure}

No obstante el resultado anterior, si rememoramos los resultados registrados por cada mecanismo notamos que existe un valor que diferencia notablemente a reuseport con respecto a nuestra solución (en cualquiera de sus dos sabores). Dicha diferencia viene dada por la variación que presenta cada mecanismo entre los tiempos mínimos y tiempos netos que registra. Para ilustrar mejor lo anterior definimos el tiempo de coordinación como el tiempo transcurrido entre el tiempo neto y el tiempo mínimo registrado en la prueba. De la misma manera, definimos un proporcional porcentual del tiempo de coordinación con respecto al tiempo neto de la prueba como \textbf{índice de coordinación} con el simbolo $\eta$, de acuerdo a la siguiente ecuación:

\begin{equation}
\eta = \frac{\left(Tº Neto - Tº Mínimo\right)}{Tº Neto}
\end{equation}

De esta forma, el índice de coordinación ilustra la variación entre la finalización de trabajo de cada socket con respecto al tiempo total de la prueba misma. 

En la práctica, cuando $\eta\rightarrow 0$ indica que los tiempos de los distintos sockets trabajando son equitativos, y la prueba la completan todos de forma simultánea. Al contrario, cuando ocurre que $\eta \rightarrow 1$ indica que los distintos sockets terminan su trabajo en tiempos muy diferentes entre sí, más aún, indica que todo el tiempo neto de la prueba se está debiendo a la actividad del último socket en terminar, habiendo finalizado los demás hace mucho.

A continuación se presentan los resultados del indice de coordinación para los distintos mecanismos estudiados en esta investigación.

\begin{figure}[!h]
	\centering
	\includegraphics[scale=.6]{resultados/tiempodelta-crop.pdf}
	\caption{Comparativa de índices de coordinación entre reuseport y la solución propuesta en sus dos variantes de distribución.}
	\label{fig:tiemposdelta}
\end{figure}

De los resultados de la figura \ref{fig:tiemposdelta} se extrae un importante resultado asociado a la equitatividad de trabajo en cada escenario. En la práctica, los tiempos de reuseport vienen dados siempre por el rendimiento de consumo del último socket que consuma su cuota para completar el consumo neto, lo cual genera una gran disparidad en el ritmo de trabajo de los distintos sockets coordinados en un mismo puerto. El resultado sugiere que al usar reuseport, el consumo de paquetes no es exactamente equitativo entre los sockets asociados al mismo puerto, más aún, garantiza un menor tiempo con decrecimiento proporcional al número de sockets trabajando, pero no da garantías sobre el tiempo neto, que como se evidenció en esta prueba, es aquel que termina determinando el tiempo real de trasferencia de la prueba.

Contrario a lo anterior, el resultado del índice de coordinación para nuestra solución es bastante alentador. Con un valor $\eta$ cercano a cero, los esquemas de distribución implementados en el módulo desarrollado garantizan un tiempo de trabajo similar para todos los sockets usados en el experimento. De hecho, es con el esquema de distribución secuencial \emph{SequentialSched} que el valor de $\eta$ es prácticamente 0, con lo que el tiempo de trabajo de cada socket asociado es idéntico para con los demás en esta prueba.


\section{Aciertos de la solución}

A raíz de los análisis anteriores, se desprenden varios aspectos rescatables de la solución implementada, obtenidos como consecuencia de satisfacer los requerimientos de operación planteados al comienzo del presente capítulo, y que diferencian nuestra solución de las opciones disponibles.

Un primer elemento a destacar es el \textbf{rendimiento}. En las pruebas desarrolladas en secciones anteriores se corroboró la efectividad de la solución propuesta, comparándola con la mejor alternativa disponible: \emph{Reuseport}, de donde se concluyó que nuestra propuesta logra un comportamiento similar al de \emph{Reuseport} en términos nominales, conservando un margen de diferencias que mantiene a \emph{Reuseport} como la opción más rápida, pero postulando nuestro desarrollo como competitivo en esa línea.

Un segundo punto a destacar de la solución es su \textbf{equitatividad de tiempos}, una característica que lo diferencia de \emph{Reuseport} pues, nuestro diseño provee un rendimiento más homogéneo entre múltiples procesos. Esta característica es muy interesante de resaltar pues hay aplicaciones donde es necesario garantizar un escenario altamente \emph{Fairness} y experimentalmente nuestra solución luce prometedora en ese sentido.

Otro aspecto a destacar de nuestra solución es que logra buenos rendimientos preservando los \textbf{principios de diseño del modelo OSI}. Esta característica era fundamental de proveer pues da garantías de fácil adopción de la solución en entornos con aplicaciones ya operativas sin mayores modificaciones de los mismos.

Otro importante acierto a reconocer de neustra solución consiste en su \textbf{configurabilidad} y \textbf{extensibilidad}. El primero pues, como se explicó en secciones anteriores, la solución admite distintos parámetros de configuración que permiten, sin cambios del código fuente, proveer distintos modos de funcionamiento según el requerimiento de consumo y distribución que se busque. El segundo dado por la implementación misma de la solución, ello pues al ser un módulo del kernel, es fácilmente modificable para añadir nuevo código fuente que permita modificar los esquemas de distribución, etapas de interceptación de paquetes, funcionamiento del framework de \emph{Netfilters}, etc.

Finalmente, una característica diferenciadora de nuestra solución con respecto a alternativas disponibles como reuseport viene dada por el denominado \textbf{índice de coordinación}. Como se ilustro en el experimento de la sección previa, la solución implementada consigue un valor de $\eta$ cercano a 0, garantizando así una solución justa en los tiempos de trabajo entre los sockets involucrados en la redistribución de paquetes. Este resultado es relevante pues, para la gran cantidad de aplicaciones relacionadas al consumo constante de datos y de distribución de trabajo –como en el escenario original de distribución de carga DNS estudiado--, una distribución justa y equitativa en el consumo de paquetes es crucial para dar verdaderas optimizaciones reales.

\section{Proyecciones}
La solución planteada en la presente investigación cumple correctamente para los fines con que se postuló originalmente, basada en su satisfacción de los requerimientos especificados en la seccion previa. Sin embargo, el mismo modelo de operación que se planteó en esta solución puede ser aprovechado y extendido para ser usado en otros escenarios o contextos.

Un caso interesante consiste en extender la solución para funcionar con otros protocolos orientados a la conexión, como por ejemplo TCP. El gran problema de la solución desarrollada para con el caso de TCP es salvar correctamente la etapa de negociación inicial para establecer la conexión -El denominado \emph{handshake de 3 pasos} de TCP-. En este contexto, la solución desarrollada debe poder contemplar un mecanismo para poder hacer una correcta redistribución de paquetes para respetar la asignación apropiada entre los paquetes recibidos y las conexiones que se están llevando a cabo.

Gracias a su buen diseño, suplir el requerimiento anterior es sencillo con nuestra solución. Basta implementar un nuevo esquema de distribución que sea conciente de la distribución de paquetes para, usando las tuplas de datos de origen/destino de cada paquete, hacer una distribución correcta entre los sockets de redirección, para que las conexiones sean correctas. Para ello, una solución interesante es en lugar de aplicar un criterio aleatorio o secuencial de distribución como se usó en los esquemas provistos en este trabajo, emplear una \emph{función de Hash} que use las tuplas de origen/destino de cada paquete y así, lleve un registro lógico correcto de las conexiones que están operativas sobre un mismo socket.

\begin{figure}[!h]
	\centering
	\includegraphics[scale=.5]{imagenes/socketMultiplexed.png}
	\caption{Capacidad de multiplexar un único puerto lógico del sistema, permitiendo comunicación diversa en una única interfaz de red.}
	\label{fig:multiplexarPuerto}
\end{figure}

El enfoque antes planteado es sumamente poderoso y brinda una interesante oportunidad pues postula la capacidad de \textbf{multiplexar} un único puerto del sistema, haciendo posible la recepción de paquetes genéricos en una misma interfaz y reasignar los mismos a un pool de otros sockets atendidos por aplicaciones independientes. En otras palabras, se podrían atender diversos requerimientos en una misma interfaz lógica del sistema asociada a un puerto para atender una variedad de servicios, rigiendo la distribución según las características mismas de un paquete (Ver figura \ref{fig:multiplexarPuerto}).

Tan flexible es este enfoque que permite soportar una diversidad de servicios (ya sean orientados a la conexión o a la mensajería) sobre una única interfaz lógica de conexión según las capas 3 y 4 del modelo OSI. Esto en la práctica permite postular a mejoras en aspectos como seguridad, permitiendo que sólo los servicios reconocidos consiguen realizar conexiones correctas. Un ejemplo práctico de ello anterior se ilustra en la figura \ref{fig:multiplexarPuertoEjemplo} donde se muestran distintas aplicaciones que operan trabajando con paquetes de diversa naturaleza (tanto por sus características, como por sus orígenes) pudiendo reasignar, clonar o simplemente descartar paquetes, según la naturaleza de las reglas definidas en el módulo.

\begin{figure}[!h]
	\centering
	\includegraphics[scale=.6]{imagenes/udpredistributeapplications.png}
	\caption{Ejemplo de multiplexación de un único puerto con varias aplicaciones diferentes. Es el módulo el encargado de la resolución de la entrega de los paquetes según sus características internas.}
	\label{fig:multiplexarPuertoEjemplo}
\end{figure}

\begin{conclusion}
Como se expuso en el presente trabajo, el inminente crecimiento de las redes de computadoras y el explosivo aumento en el uso de distintos dispositivos que demandan conectividad hacen que volver la Internet más robusta, disponible y eficiente sea una prioridad. Entendiendo como parte crucial de dicha labor el trabajo del servicio DNS, postular mejoras que optimicen su operación es un ejercicio plenamente justificado en pos de brindar una mejor performance sobre la red. Es tras ese afán --y apoyado en un fundamento teórico probado-- que se han realizado distintos esfuerzos por mejorar el rendimiento en el consumo de datos desde una interfaz lógica de red incorporando paralelismo, pero sin los resultados esperados. 
 
 
A lo largo de esta investigación se repasaron las principales sospechas vigentes que tratan de dar explicación al mal rendimiento presentado en el consumo concurrente de datos desde una interfaz de red, dividiendo el problema en 3 estudios: 
 
 
A través del \textbf{estudio de operación de las primitivas de sincronización del sistema} se corroboró que frente a condiciones de concurrencia, la actividad de sincronización dada por las llamadas a funciones de dicho tipo se incrementa a medida que se usan más hilos en el consumo de datos, volviéndose cada vez más significativa este tipo de funciones tanto en términos porcentuales como en términos del tiempo de ejecución total. Se determinó además que el protagonismo en dichas llamadas está asociado al control de bloqueo de spinlocks que protegen los sockets y que la tendencia en la duración de estas llamadas se ajusta de manera muy exacta a una curva de tipo logarítmico, algo interesante pues se rige por la misma tendencia que se determinó dominaba en el registro de tiempos netos de consumo de datos del caso de estudio, marcando una correlación directa entre ambos fenómenos y dando a entender que el control del spinlock termina impactando directamente en los tiempos totales de operación del socket. 
 
 
Con respecto al \textbf{estudio de canales de comunicación de hardware}, centrado en la dinámica de los \emph{performance counters} disponibles en el sistema, se logró corroborar el comportamiento entre componentes de hardware y cómo dicha dinámica se vuelve caótica a medida que se emplea una solución basada en concurrencia sobre un mismo socket, ello apoyado en el explosivo incremento de eventos como correcciones de caché, predicciones de procesamiento erróneas y rescate de datos de distintos niveles de memoria, entre otros. Más aún, este estudio demostró que el sistema --como un conjunto-- presenta un comportamiento explosivo en todas las aristas estudiadas, evidenciado en una alta correlación entre los eventos registrados al evaluarlos en un régimen concurrente. Nuevamente, se obtuvieron resultados que avalaban la sospecha de que la estructura socket no presenta un rendimiento escalable a medida que se incorporan operaciones concurrentes, reafirmando la hipótesis de que dicha estructura no tendría un diseño compatible con estrategias de consumo paralelo como las evaluadas. 
 
 
Por otro lado, nuestro \textbf{estudio de distribución de carga} usando \emph{processor affinity} demostró que aún cuando la arquitectura NUMA evaluada está optimizada para un mejor rendimiento basada en un procesamiento con datos distribuidos, las estrategias usadas para la asociación de hilos de ejecución a distintos cores no mostró ningún beneficio práctico. Más aún, se concluyó por una parte que al intentar aprovechar los distintos núcleos de procesamiento se cae en la misma situación detectada al usar concurrencia, ello pues los recursos terminan dispersándose entre los nodos NUMA y sus subestructuras de almacenamiento de datos, perdiendo el beneficio dado por el principio de localidad de memoria, y haciendo más dificil la coordinación en el acceso al socket mismo. Y por otro lado, que la aplicación de esquemas que reúnen los hilos en una misma componente de procesamiento termina por serialziar los accesos concurrentes, brindando un rendimiento equivalente a usar un sólo hilo. El comportamiento anterior concluye que las estratégias de procesamiento distribuido entre cores no brindan beneficio alguno pues a pesar de estar distribuyendo la carga de procesamiento se mantiene un mismo cuello de botella en el acceso por concepto del mecanismo de protección del socket compartido (nuevamente, el spinlock del socket). 
 
 
De esta forma, se pudo corroborar responsabilidades transversales y cruzadas entre las sospechas vigentes del problema estudiado. Por un lado transversales, pues todas las sospechas resultaron ser corroboradas experimentalmente, dando a entender que --en mayor o menor grado-- la operación de las interfaces de red de Linux no incorpora en su diseño una aplicación que admita concurrencia en el consumo de información, limitando así la capacidad de optimizaciones usando una estrategia de paralelismo a secas. Y por otro lado cruzadas, en el sentido de que los factores responsables de cada sospecha resultan repetirse entre cada caso, siendo el conjunto de variables que constituyen un Internet socket --en particular, los spinlock de protección del mismo-- elementos que al ser compartidos, sobrecargan los protocolos de coherencia y consistencia de datos al incurrir en estratégias concurrentes.
 
 
Como diagnóstico final reuniendo las principales conclusiones de los tres estudios contemplados en la presente investigación se concluye que, tal y como se postuló como hipótesis inicial, la estructura socket implementada a nivel del kernel de Linux no es apta para operar correctamente en condiciones de concurrencia, ello debido a que la misma no estaría diseñada para soportar dinámicas de acceso paralelo. Por otro lado, se concluye también que la arquitectura NUMA evaluada no puede sacar verdadero provecho del paralelismo sobre una estructura socket pues, por su diseño, los mecanismos de protección de spinlocks asociados a los sockets producen efectos de cuello de botella que terminan impactando negativamente al sistema al sobrecargar los mecanismos de coherencia y corrección de caché. En este escenario, resulta incompatible la aplicación de verdadera distribución de trabajo entre distintas componentes de procesamiento cuando se tiene un punto de contención detectado que fuerza una dinámica de serialización en el acceso al socket, entorpeciendo la coordinación de los distintos hilos que acceden al socket y terminando por degradar los tiempos netos de operación. 
 
 
Resulta interesante el que, al día de hoy, el problema estudiado sigue vigente y abierto a nuevas propuestas de soluciones. En esta misma línea, recientemente Facebook, a través de su equipo de desarrollo de kernel ha postulado modificaciones al núcleo de Linux que permitan hacer un uso más flexible en el consumo concurrente de datos desde estructuras protegidas por medio de estructuras denominadas \emph{Blk-mq} \cite{post:facebookFin}, que van en busca de aprovechar consumo distribuidos de colas usando lecturas exclusivas por cada CPU. Un enfoque que aprovecharía de mejor manera las interacciones de hardware del sistema al tener accesos más localizados que lo detectado en nuestro estudio de \emph{performance counters} e inspirado sobre la misma idea que nuestro estudio de \emph{processor affinity} pero implementado directamente a nivel del kernel. Por otro lado, una reciente publicacion de la revista \emph{;login;} de Usenix \cite{magazine:login} presentó un artículo sobre cómo aprovechar arquitecturas NUMA a modo de incrementar el rendimiento de operaciones sobre estructuras compartidas apoyándose en un estudio basado en performance counters, un enfoque muy similar al que se repasó en nuestra investigación combinando los enfoques de los estudios de \emph{performance counters} y \emph{processor affinity}, mostrando que las estrategias de análisis aplicadas en el presente trabajo están plenamente vigentes y siendo empleadas en investigaciones de alto nivel para sistemas y problemas actuales. 
 
 
En respuesta al problema, se postuló una solución basada en un módulo del kernel para Linux que, bajo un escenario de operación fiel al caso de estudio presentado, se presenta plenamente comparable con reuseport --la mejor opción disponible para paliar el problema estudiado en cuestión-- siendo competitivo en rendimiento con este último, brindando en la práctica un desempeño muy similar. La solución desarrollada cuenta con varios aciertos que la distinguen de reuseport: Una característica de balanceo de trabajo que ni el mismo reuseport incorpora, la que brinda una distribución de tiempos de trabajo más equitativa entre los sockets que trabajan conjuntamente, con respecto a reuseport. Por otro lado, ser una solución extensible y modificable, lo que garantiza su fácil adaptación para escenarios de operación específicos de distribución de paquetes entre sockets e incluso la capacidad de portar la solución misma a otras plataformas. Finalmente, el ser un desarrollo modular (Al ser un módulo del kernel) es una característica que permite postular la solución como una alternativa válida sobre cualquier entorno Linux y la hace invasiva con respecto al resto del sistema.  
 
 
Así también, esta investigación develó que el desempeño de las estrategias que involucran paralización no producen las ganancias de tiempo que se esperarían de dichas técnicas de procesamiento en el escenario estudiado --apoyando el diagnóstico de que los sockets no estarían diseñados para soportar dicha forma de operación--, lo cual hace pensar que, a menos de que se enfrente un escenario muy restrictivo que obligue a emplear técnicas de compartición de interfaces de red, o que ameriten mecanismos inteligentes de distribución de carga de datos, la alternativa de emplear un único socket resulta acertada y más que suficiente para obtener un buen rendimiento. 


Los resultados obtenidos con la presente investigación plantean distintas líneas de desarrollo que seguir como trabajo futuro. Un camino interesante corresponde al desarrollo de un modelo de costos basado en todas las variables recopiladas a lo largo de la presente investigación, el cual permita modelar la interacción de las distintas componentes estudiadas de manera de reconocer operaciones de alto costo computacional en la practica, y así poder optar a optimizaciones del sistema para lograr mejoras en los tiempos de ejecución. Un análisis de ésta naturaleza se puede hacer a partir del trabajo desarrollado en los estudios de operación de las primitivas de sincronización del sistema y de canales de comunicación de hardware desarrollados en esta tesis, los cuales ya proveen abundante información individual de las componentes del sistema frente a la ejecución de una tarea especifica. Otro camino en pos de profundizar esta investigación va de la mano del estudio de distribución de carga desarrollado en la misma, del cual se pueden evaluar mejores técnicas de distribución de trabajo según la arquitectura del sistema huésped y de los recursos disponibles, a fin de conseguir mejoras significativas al emplear estrategias de ejecución concurrente. Finalmente, una labor más a considerar como trabajo futuro es la relacionada con el desarrollo más acabado de la solución planteada en esta tesis, en el sentido de ampliar su operación diseñando y evaluando nuevos esquemas de distribución que se adapten mejor a distintos requerimientos de aplicaciones o sistemas de arquitecturas especificas.
\end{conclusion}

% \input{glosario.tex} % opcional

\bibliographystyle{plain}
\bibliography{bibliografia}

% \input{anexo_apendices.tex} % opcionales

\end{document}

%%% Local Variables:
%%% mode: latex
%%% TeX-master: t
%%% End:
