\begin{thanks} % opcional

Me resulta extraño escribir estas últimas palabras, especialmente en un trabajo que significa la culminación de 7 años de esfuerzo, aventuras, emociones, rabias, alegrías y --lo más importante-- personas. En este momento me resulta inevitable pensar en proponerme nuevos desafíos, metas que me permitan mantener viva mis ansias de crecer y que me hagan revivir esa sensación de vértigo a la que me acostumbró la vida en la universidad. Sin embargo, tanto o más fuerte que el anhelo de buscar nuevos horizontes es mi deseo de agradecer a quienes han sido parte importante de mi caminar. La presente página es para todos aquellos quienes fueron parte de mi proceso formativo profesional, y a quienes de una u otra manera les debo la persona quien soy hoy día. Aquí vamos: 
 
A mis amigos del colegio: Gauda, Cristian, Camila, Palacios, Pablo, Nico, Pau, Tomasin, y especialmente, al Maxi y al Ortiz. Mi pandilla, una familia con la que crecí y con quienes tengo la dicha de mantener contacto. Guardianes de una confianza especial y un lazo desde niños. A ustedes, que fueron mi nexo a mi pasado y mi cable a tierra a lo largo de este proceso, les agradezco de corazón. 
 
A mis amigos de plan común: Consuelo, Carmen, Peet, Pipo, Leo, Ismael, Samir, Sergio e Ivan. Compañeros en la lucha en plan común, con ustedes guardo los recuerdos de las aventuras más geniales y locas de mi paso por la universidad, bañados de nuestra vitalidad de los primeros años como mechones, y aún siendo más viejones. Con ustedes aprendí a vivir una nueva vida, y a creer en la nueva amistad. Muchas gracias. 
 
A mis partners de computación: Omar, Jazmine, Lehmann, Mario, Boris, Pancho, Nico, Camilo, Tito, Kevin, Max Y Giselle. Colegas en la aventura más larga de mi vida hasta ahora. Por cada paso que dimos juntos, desde pregrado al magister, por la aventura en el CaDCC, por los paseos adonde fuere, por las historias que solo nuestros corazones conocen y por la confianza y la amistad. Muchas gracias amigos, que la ñoñez nos una por siempre.  
 
A Javier Bustos, por darme la oportunidad de participar en el NIC Labs, por las oportunidades para representar al laboratorio, y por las anécdotas y enseñanzas, gracias.
 
Finalmente, a mi familia, por apoyarme a lo largo de todos estos años, impulsándome a dar lo mejor de mi mismo, y confiando siempre en mis capacidades para salir adelante. Siento su orgullo en cada paso que doy y me hace fuerte para proponerme siempre metas mejores. Mamá, Papá, Jesus, Marcos, gracias por estar ahí siempre, los amo.



\end{thanks}