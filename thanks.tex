\begin{thanks} % opcional
Al final de cualquier camino importante, creo es importante hacer un balance de las cosas que vivimos. El trabajo que tienes en tus manos estimado lector, representa mucho más que sólo una investigación de casi un año... Es más bien, la culminación de un proceso formativo que me tomó algo asi como 7 años en las aulas de la facultad de ciencias físicas y matemáticas de la universidad de chile, y del cual, más que las herramientas formales que me proveieron en ese tiempo, creo que es el momento de destacar a las personas que me dieron una parte de ellas en mi camino profesional.

Mis primeros agradecimientos van a mis amados compañeros del colegio... Gauda, Cristian, Camila, Pablo, Nico, Palacios y muy especialmente al Maxi y al Ignacio... Mi pandilla, mis complices... mis amigos de confianza completa... Siempre me reí de las tallas por estudiar Ganadería Nuclear o por trabajar con la Soledad Onneto de la tele --Como el nico siempre decía--. Ustedes, amigos, fueron mi cable a tierra por todos estos años, el lazo con mi pasado, con mi familia del nido del Alcántara, con quienes me formé y de quienes aprendí cómo debían ser las personas para darles mi confianza y aceptar la suya de vuelta. Por todo, aún por lo que crean no importante, muchas gracias.

Mi siguiente parrafo va pensado en mis primeros años en la universidad de Chile, y mi paso por el célebre plan común de ingeniería y ciencias. No puedo recordar mis primeros días en ese mundo sino con mucha ansiedad y un fuerte vacío tras dejar atras a mi familia del colegio.... Pero como dicen, no hay mal que por bien no venga. En esta etapa de mi vida conocí a un grupo de personas excepcionales, con quienes me abri paso por el plan comun --Etapa que considero, la más dificil de mi carrera--, En este periodo aparecen nombres de personas que considero grandes amigos: Carmen, Peet, Leo, Ismael, Sergio, Samir, Consuelola, Ernesto Pipo Perez, Gotshclish, Ivan, Ale Madrazo, Mono, Joako y Diego Floyd. Con ustedes guardo las memorias de las aventuras más entretenidas que pasé por la universidad, Como olvidar los carretes donde Ismael (Un trompo y una piscina me recuerdan dos anecdotas, alomenos dignas de recordar), las escapadas al depa de la Ale Madrazo (Con el mejor resultado en tareas de computa), Las maratones de problemas DIM (Con consuelo y pipo 20 unidades adelante de los mortales), Los asados donde Joako (Camino al fin del mundo), Las idas a la celebre concha acústica (Mal que mal, había que visitar la PUC de vez en cuando), las juntas donde pelao-peluo (¿Cuantos ingenieros se necesitan para cambiar un neumático?), las marchas por la educacion (y las buenas pizzas al final de cada una), y más de algún paseo a la playa (Que con polémicas y todo, se pasaba a todo dar). Amigos soldados, sobrevivientes de esta historia, gracias por ser mis compañeros en un camino duro.


\end{thanks}